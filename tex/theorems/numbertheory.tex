\field{Number Theory}
\begin{technique}[Modular Arithmetic Basics]
    \label{teq: modulo}
    Fix a positive integer $n$. For any
    two integers $a$ and $b$, we say that ``$a$ and $b$ are equivalent
    \emph{modulo} $n$'' if and only if $a - b$ is divisible by $n$. In other
    words, $a$ and $b$ are equivalent mod $n$ if and only if they leave the
    same remainder when divided by $n$. We write
    \[ a \equiv b \pmod{n} \]
    to symbolize this equivalence. For example,
    \[ a \equiv 0 \pmod{n} \]
    means that $a$ is divisible by $n$.

    This equivalence symbol $\equiv$ really behaves like the equality symbol $=$
    that we use all the time. In particular, it respects all of the arithmetic
    operations except for division. The latter requires special care; we can
    only divide both sides of the equivalence by an integer $k$ if and only if
    $k$ is relatively prime to $n$. (Think about why this is true!) If $a \equiv
    b \pmod{n}$ and $c \equiv d \pmod{n}$, then we have
    \[ a + b \equiv c + d \pmod{n}, \quad ab \equiv cd \pmod{n},
    \quad\text{and}\quad a^k \equiv b^k \pmod{n}\,\text{ for all $k\in\mathbb{N}$}. \]
    This is a very useful notation for dealing with divisibility problems, and
    it will be used freely in the solutions.
\end{technique}

\begin{theorem}[Fermat's Little Theorem]
    \label{thm: FLT}
    For any prime $p$ and positive integer $a$,
    \[ a^p \equiv a \pmod{p}.\] 
    If $a$ and $p$ are relatively prime in addition,
    \[ a^{p - 1} \equiv 1 \pmod{p}.\]
    This is because we can divide both sides of the equivalence by $a$.
\end{theorem}

\begin{lemma}
    \label{lem: weakCRT} 
    Let $a$, $b$ and $n$ be integers. Then
    $a$ and $b$ both divide $n$ if and only if their least common multiple also
    divides $n$. A useful case of this is when $a$ and $b$ are relatively prime.
    In this case, their least common multiple is $ab$, so $a$ and $b$ both
    divide $n$ if and only if $ab$ divides $n$.

    More generally, if we have $k$ integers $a_1, a_2, \ldots,
    a_n$, then they all divide $n$ if and only if their least common multiple
    also divides $n$.
\end{lemma}

\begin{lemma}
    \label{lem: divisibilityGP}
    Let $a$ and $b$ be integers, and $n$ be a positive integer. Then 
    \[ \frac{a^n - b^n}{a - b} = a^{n - 1} + a^{n - 2}b + a^{n - 3}b^2 + \cdots + b^{n - 1}. \]
    In particular, $a^n - b^n$ is divisible by $a - b$. Think about how this is
    related to the last identity in the equation in Technique~\ref{teq: modulo}.
\end{lemma}

% \begin{theorem}[Legendre's formula]
    % \label{thm: legendre}
    % Proof: Consider a table with a column for each number containing a $\nu_p(n)$ amount of dots vertically. Then count the number of dots in each layer and sum up. 
% \end{theorem}

\begin{lemma}[Euclid]
    \label{lem: euclid}
    Let $a$, $b$ and $c$ be positive integers and suppose that $a$ divides $bc$.
    If $a$ and $b$ are relatively prime, then $a$ divides $c$. This innocent
    looking lemma is surprisingly hard to prove without being circular!
\end{lemma}

