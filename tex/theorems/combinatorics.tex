\field{Combinatorics}
\begin{theorem}[Inclusion-Exclusion Principle]
    \label{thm: incluexclu}
    The inclusion-exclusion principle is a useful technique to count the number
    of elements in a union of sets, which may not be disjoint. In its
    simplest form, it states that given two sets $A$ and $B$, the number of
    elements in the set $A \cup B$ is
    \[ |A \cup B| = |A| + |B| - |A \cap B|. \]
    This can be proven easily by using a Venn diagram. More generally,
    suppose that $A_1$, $A_2$, $\ldots$, $A_n$ are $n$ sets. Then the number of
    elements in the union $A_1 \cup A_2 \cup \cdots \cup A_n$ is the alternating
    sum
    \[ |A_1 \cup A_2 \cup \cdots \cup A_n| = \sum_{i = 1}^n |A_i| - \sum_{1 \leq
        i < j \leq n} |A_i \cap A_j| + \cdots + (-1)^{n + 1} |A_1 \cap A_2 \cap
    \cdots \cap A_n|. \]
    For example, the above formula for $n = 3$ would be
    \[ |A_1 \cup A_2 \cup A_3| = |A_1| + |A_2| + |A_3| - |A_1 \cap A_2| - |A_2
    \cap A_3| - |A_1 \cap A_3| + |A_1 \cap A_2 \cap A_3|. \]
\end{theorem}
\begin{theorem}[Pigeonhole Principle]
    \label{thm: pigeonhole}
    Suppose that you have $n$ objects and $k$ buckets, where $k < n$. The
    pigeonhole principle states that no matter how you distribute the objects
    into the buckets, there will at least be one bucket with more than one
    object in it. This is quite a trivial statement but it can be used to prove
    pretty counterintuitive stuff.
\end{theorem}
\begin{technique}[Stars and Bars]
    \label{teq: starsandbars}
    Suppose that we have $n$ identical stars and $k$ buckets, and we want to
    count the number of ways to distribute the stars into buckets such that
    \emph{each bucket contains at least one star}. One way to decide how to
    distribute the stars is by laying all the stars in a line, and dividing them
    into $k$ parts by inserting $k - 1$ bars in the gaps between them. Note that
    there must be at least one star between consecutive bars; this means that
    only one bar can be inserted into one gap. There are $n - 1$ gaps, and the
    number of ways to insert $k - 1$ bars in $n - 1$ gaps is
    \[ \binom{n - 1}{k - 1}. \]

    Now suppose that we allow the buckets to be empty. i.e., we no longer need
    the each bucket to contain at least one star. In this case, there can be
    multiple bars in each gap, and it is no longer easy to count the number of
    ways to insert the bars. Instead, we can view each configuration as a
    permutation of stars and bars \emph{combined}. Since there are $n$ identical
    stars and $k - 1$ identical bars, the number of ways to permute them is
    \[ \binom{n + k - 1}{k - 1}. \]
\end{technique}
\begin{definition}[Graph]
    \label{def: graph}
    A graph is a structure consisting of a set of \emph{vertices}, and a set of
    \emph{edges} connecting those vertices. Usually, the vertices represent
    objects that we are interested in a problem (e.g., people), and the edges represent the
    relations between the objects (e.g., two people are adjacent). Below is an
    example of a graph. The dots are vertices and the line segments are edges.
    \begin{center}
        \begin{asy}
            size(6cm);
            draw((0, 0)--(1, 2)--(4, 4)--(7, 2)--(8, 0));
            draw((1, 2)--(2, 0)--(2.5, -2));
            draw((2, 0)--(1.5, -2));
            draw((6, 0)--(7, 2));
            dot((0, 0));
            dot((1, 2));
            dot((4, 4));
            dot((7, 2));
            dot((8, 0));
            dot((2, 0));
            dot((6, 0));
            dot((1.5, -2));
            dot((2.5, -2));
        \end{asy}
    \end{center}

    For example,
    there is the path graph with $n$ vertices, denoted as $P_n$, which is just a
    chain. Here is a (boring) diagram of $P_1$ (which is just a point), $P_2$,
    $P_3$ and $P_4$.
    \begin{center}
        \begin{asy}
            size(8cm);
            pair P11 = (0, 0);
            pair P21 = (1, 0);
            pair P22 = (2, 0);
            pair P31 = (3, 0);
            pair P32 = (4, 0);
            pair P33 = (5, 0);
            pair P41 = (6, 0);
            pair P42 = (7, 0);
            pair P43 = (8, 0);
            pair P44 = (9, 0);
            draw(P21--P22);
            draw(P31--P33);
            draw(P41--P44);
            dot(P11);
            dot(P21);
            dot(P22);
            dot(P31);
            dot(P32);
            dot(P33);
            dot(P41);
            dot(P42);
            dot(P43);
            dot(P44);
        \end{asy}
    \end{center}
    There's also the cycle graph with $n$ vertices, denoted as $C_n$, which as
    the name suggests is a cycle with $n$ vertices. Here are the diagrams of
    $C_3$, $C_4$ and $C_5$ respectively.
    \begin{center}
        \begin{asy}
            size(8cm);
            path t = dir(90)--dir(210)--dir(330)--cycle;
            path s = dir(45)--dir(135)--dir(225)--dir(315)--cycle;
            path p = dir(72)--dir(144)--dir(216)--dir(288)--dir(360)--cycle;
            draw(t);
            draw(shift((3, 0))*s);
            draw(shift((6, 0))*p);
            dot(dir(90));
            dot(dir(210));
            dot(dir(330));
            dot(dir(45) + (3, 0));
            dot(dir(135) + (3, 0));
            dot(dir(225) + (3, 0));
            dot(dir(315) + (3, 0));
            dot(dir(72) + (6, 0));
            dot(dir(144) + (6, 0));
            dot(dir(216) + (6, 0));
            dot(dir(288) + (6, 0));
            dot(dir(360) + (6, 0));
        \end{asy}
    \end{center}
\end{definition}
\begin{definition}[Chromatic Polynomial]
    \label{def: chromaticpoly}
    Let $G$ be a graph and $k$ be any positive integer. A $k$-colouring of $G$
    is a colouring of the vertices of $G$, such that any two vertices which are
    connected by an edge have different colours. The \emph{chromatic polynomial}
    of $G$, often denoted as $P(G, k)$, is the number of $k$-colourings of
    $G$. (The fact that this function is a polynomial in $k$ is
    not obvious; however, it can be proven using the next technique.)
\end{definition}
\begin{technique}[Deletion-Contraction]
    \label{teq: DC}
    Suppose that we have a set of objects in a specific configuration (e.g., in a
    circle) and we want to count the number of ways to colour them such that no
    two adjacent objects have the same colour. Such a problem can be
    quite annoying to tackle as you can miss a case or two and get the wrong
    answer, but it becomes a lot easier once you translate it into a graph
    colouring problem. First, start by drawing a vertex for each object, and
    then draw an edge between any two objects that are adjacent to each other.
    Then the task of finding the number of colourings merely becomes the task of
    finding the chromatic polynomial of the resulting graph.

    Deletion-contraction gives a way to recursively find the chromatic
    polynomial of any graph. Let $G$ be a graph and $k$ be a positive integer.
    Pick any edge $e$ with endpoints $u$ and $v$. Let $G - e$ be the graph
    obtained by removing the edge $e$. Then the set of $k$-colourings of $G'$
    can be divided as follows:
    \begin{multline*}
        \{\text{$k$-colourings of $G-e$}\}
        = \{\text{$k$-colourings of $G-e$ where $u$ and $v$ have the same
        colour}\}\\
        \cup \{\text{$k$-colourings of $G-e$ where $u$ and $v$ have different
        colours}\}
    \end{multline*}

    Note that the number of elements in the first set is just the number of
    $k$-colourings of $G/e$, which is the graph obtained by deleting the edge
    $e$ of $G$ and merging $u$ and $v$ together. Also, the number of elements in the
    second set is the number of $k$-colourings of the original graph $G$. In
    other words, we have the following equality:
    \[ P(G - e, k) = P(G/e, k) + P(G, k), \]
    which can be rearranged to form
    \[ P(G, k) = P(G - e, k) - P(G/e, k). \]
    In this way, we can recursively find $P(G, k)$ for any $k$.

    For an example, see the second solution of \hyperref[sol: 2016 Regional
    Round P11]{2016 Regional Round problem 11}.
\end{technique}
