\field{Algebra}
\begin{theorem}[Remainder Theorem]
    \label{thm: remainder}
    Let $f(x)$ be a polynomial. Then the remainder of $f$ when divided by $x -
    r$ is $f(r)$.

    A special case of this theorem when the remainder is 0 is called the factor
    theorem. It says that $x - r$ is a factor of $f$ if and only if $f(r) = 0$.
\end{theorem}
\begin{theorem}[RMS-AM-GM-HM Inequality]
    \label{thm: amgm}
    For any positive real numbers $a_1, a_2, \ldots, a_n$,
    \[\sqrt{\frac{x_1^2 + x_2^2 + \cdots + x_n^2}{n}} \geq \frac{x_1 + x_2 + \cdots + x_n}{n} \geq \sqrt[n]{x_1 x_2 \cdots x_n} \geq \frac{n}{1/x_1 + 1/x_2 + \cdots + 1/x_n}\]
    with equality if and only if $a_1 = a_2 = \cdots = a_n$.
\end{theorem}

\begin{theorem}[Cauchy-Schwarz Inequality]
    \label{thm: cs}
    For any two sequences of real numbers\\ $a_1, a_2, \ldots, a_n$ and $b_1, b_2, \ldots, b_n$, the following inequality holds:
    \[ (a_1 b_1 + a_2 b_2 + \cdots + a_n b_n)^2 \leq (a_1^2 + a_2^2 + \cdots + a_n^2)(b_1^2 + b_2^2 + \cdots +b_n^2) \]
    with equality if and only if $a_i = kb_i$ for some real constant $k$.
\end{theorem}

\begin{theorem}[Vieta's Formulas]
    \label{thm: vieta}
    Let $a_{n}x^{n}+a_{n-1}x^{n-1}+\cdots + a_{0}$ be a polynomial with roots $r_{1},r_{2},\ldots,r_{n}$. Then the roots and coefficients are related as follows:
    \begin{align*}
        r_{1}+r_{2}+\cdots+r_{n}&=-\frac{a_{n-1}}{a_{n}}\\
        (r_{1}r_{2}+r_{1}r_{3}+\cdots+r_{1}r_{n})+(r_{2}r_{3}+\cdots+r_{2}r_{n})+\cdots+r_{n-1}r_{n}&=\frac{a_{n-2}}{a_{n}}\\
        &\vdotswithin{=}\\
        r_{1}r_{2}\cdot \ldots \cdot r_{n}&=(-1)^{n}\frac{a_{0}}{a_{n}}.
    \end{align*}
\end{theorem}

\begin{theorem}[Descartes' Rule of Signs]
    \label{thm: ruleofsigns}
    Let $f(x)$ be a polynomial with real coefficients, and let $k$ be the number
    of sign changes in the sequence of coefficients. Then the number of positive
    real roots of $f$ is either equal to $k$, or less than it by an even number.

    For example, the polynomial $x^2 + 1$ has no sign changes since its sequence
    of coefficients is $1, 1$, so it has no positive real roots. On the other
    hand, the polynomial $x^3 + 4x^2 - 5x - 6$ has exactly one sign change (its
    sequence of coefficients is $1, 4, -5, -6$ so the sign changes from $+$ to
    $-$ between $4$ and $-5$), and so it has exactly one positive real root.

    As a corollary, if we let $\ell$ be the number of sign changes in the
    coefficients of $f(-x)$, then the number of negative real roots of $f$ is
    either equal to $\ell$, or less than it by an even number.
\end{theorem}

\begin{theorem}[Fundamental Theorem of Algebra]
    \label{thm: fundamentalthmofalg}
    Let $f(x)$ be a polynomial of degree $n$ with complex coefficients. Then
    $f$ splits into $n$ linear factors with complex coefficients. i.e., there
    exist complex numbers $\alpha_1, \alpha_2, \ldots, \alpha_n$ (not
    necessarily distinct) such that
    \[ f(x) = (x - \alpha_1)(x - \alpha_2)\cdot\ldots\cdot(x - \alpha_n). \]
    In other words, $f$ has exactly $n$ roots in complex numbers when counted
    with multiplicity.
    
    For example, the polynomial $x^2 + 1$ has no real roots, but it has 2 roots,
    namely $i$ and $-i$, in complex numbers\footnote{In fact, this is one way to
    define complex numbers, by adjoining the roots of $x^2 + 1$ to
    $\mathbb{R}$.}. The polynomial $x^3 - 1$ has only one real root, 1, but it
    has two complex roots, which are $(-1 + \sqrt{3}i)/2$ and $(-1 -
    \sqrt{3}i)/2$ respectively.
\end{theorem}

\begin{technique}[Some Useful Identities]
    \label{teq: algidentities}
    For any real numbers $a$, $b$ and $c$, the following identities hold.
    \begin{enumerate}
        \item $(a + b + c)^2 = a^2 + b^2 + c^2 + 2ab + 2bc + 2ca$.
        \item $(a + 1)(b + 1)(c + 1) = abc + ab + bc + ca + a + b + c + 1$.
        \item $a^3 + b^3 + c^3 - 3abc = (a + b + c)(a^2 + b^2 + c^2 - ab - bc - ca)$. 
    \end{enumerate}
\end{technique}

\begin{technique}[Mathematical Induction]
    \label{teq: induction}
    % Insert falling dominoes
    Sometimes, while proving something, we may notice that a problem can be reduced to a smaller version of the same problem. In those cases, we can use mathematical induction. A proof using induction generally consists of two parts. Suppose that we are trying to prove a statement $P$.
    \begin{enumerate}
        \item We first show that $P(i)$ is true for some natural number $i$, usually 0 or 1. This is called the base case. 
        \item We then show that if $P(n)$ is true, than $P(n + 1)$ must also be true. This is called the inductive step.
    \end{enumerate}
    If we can do both of these steps, then we can conclude that the statement $P$ is true for all natural numbers $n \geq i$. It is easy to see why this holds: from step 1, $P(i)$ is true, and from step 2, $P(i + 1)$ must be true. Since $P(i + 1)$ is true, by step 2 again, $P(i + 2)$ is also true, and so on. This is often illustrated by the classic example of falling dominoes. Usually, the second step is a lot more important and difficult to prove than the first one. 

    There are also many other variants of induction. One important and generally more useful version is `strong induction', where we assume in step 2 that $P(k)$ is true for all $i \leq k \leq n$, rather than just $P(n)$. Another version which is used less often is `forward - backward induction' which is described below as technique \ref{teq: f.b.induction}.
\end{technique}

\begin{technique}[Forward-Backward Induction]
    \label{teq: f.b.induction}
    This is a rarely used variant of induction, but it can handle some problems where normal induction fails. The procedure is generally like this:
    \begin{enumerate}
        \item Similarly to usual induction, we first show that $P(i)$ is true for some natural number $i$, usually 0 or 1.

        \item We then show that if $P(n)$ is true, $P(2n)$ must also be true.

        \item Finally we show that if $P(n)$ is true, $P(n - 1)$ must also be true.
    \end{enumerate}
    You can see why this technique is called forward - backward induction. We use step 2 to skip over a chunk of numbers, and then use step 3 to show that $P$ is also true for the numbers that we skipped. You can read more about this technique \href{https://brilliant.org/wiki/forward-backwards-induction/}{here}. 
\end{technique}
