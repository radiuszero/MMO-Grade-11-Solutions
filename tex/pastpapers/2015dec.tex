\paperheader{2015 Regional Round}
\begin{problems}
    \problem The operation $\triangle$ is defined as $a \otriangle b = ab + 2a +
    b$. Find $x$, if $\frac{12}{6 - x} \otriangle 2 = 3 \otriangle 5$. 

    \problem If $x + \frac{1}{x} = 7$, find the value of $x^3 + \frac{1}{x^3}$. 
    
    \problem Find the value of 
    \[
    \frac{\frac{1}{2} - \frac{1}{3}}{\frac{1}{3} - \frac{1}{4}} \cdot
    \frac{\frac{1}{4} - \frac{1}{5}}{\frac{1}{5} - \frac{1}{6}} \cdot
    \frac{\frac{1}{6} - \frac{1}{7}}{\frac{1}{7} - \frac{1}{8}} \cdot \ldots \cdot
    \frac{\frac{1}{2014} - \frac{1}{2015}}{\frac{1}{2015} - \frac{1}{2016}}.
    \]

    \problem Let $p$ and $q$ be the remainders when the polynomials $f(x) = x^3
    + 2x^2 - 5ax - 7$ and $g(x) = x^3 + ax^2 - 12x + 6$ are divided by $x + 1$
    and $x - 2$ respecively. If $2p + q = 6$, find the value of $a$.

    \problem Prove that $(a + b)(b + c)(c + a) \geq 8abc$ for any $a, b, c \geq
    0$. 

    \problem The point $P$ lies on a circle with radius $r$ which has $AB$ as a
    diameter. Show that $PA \cdot  PB \leq 2r^2$ and $PA + PB \leq 2\sqrt{2}r$.
    \begin{center}
        \begin{asy}
            import olympiad;
            size(5cm);
            defaultpen(fontsize(11pt));
            pen mydash = linetype(new real[] {5,5});
            pair A = dir(180);
            pair B = dir(0);
            pair P = dir(60);
            pair O = (0, 0);
            draw(A--B--P--cycle, black+1);
            draw(circumcircle(A, B, P));
            draw(rightanglemark(A, P, B, 4));
            dot("$A$", A, dir(A));
            dot("$B$", B, dir(B));
            dot("$P$", P, dir(P));
            dot("$O$", O, dir(270));
            label("$r$", (-.5, 0), dir(270));
            label("$r$", (.5, 0), dir(270));
        \end{asy}
    \end{center}


    \problem In a sequence, $u_{1} = 1$, $u_{2} = 2$ and $u_{3} = 3$. For $n
    \geq 4$, the $n$th term $u_{n}$ is calculated from the previous three terms
    as $u_{n} = u_{n - 3} + u_{n - 2} - u_{n - 1}$. For example, $u_{4} = u_{1}
    + u_{2} - u_{3} = 0$. Write down the first 9 terms. What is the 2015th term
    of the sequence?

    \problem In the figure, a square $ABCD$ of side length 6 is given. Two
    circles with diameters $AD$ and $CD$ are drawn. Determine the combined area
    of two shaded regions.
    \begin{center}
        \begin{asy}
            import olympiad;
            size(5cm);
            defaultpen(fontsize(11pt));
            pen mydash = linetype(new real[] {5,5});
            pair A = dir(225);
            pair B = dir(135);
            pair C = dir(45);
            pair D = dir(315);
            pair O = (0, 0);
            pair M = midpoint(D--C);
            pair N = midpoint(D--A);
            filldraw(B--arc(N, abs(N-A), 180, 90)--arc(M, abs(M-C), 180, 90)--cycle, .6*gray+.4*white);
            filldraw(arc(M, abs(M-C), 180, 270)--arc(N, abs(N-A), 90, 0)--cycle, .6*gray+.4*white);
            draw(circle(M, abs(M-C)));
            draw(circle(N, abs(N-A)));
            draw(A--B--C--D--cycle, black+1);
            draw(A--C);
            dot("$A$", A, dir(A));
            dot("$B$", B, dir(B));
            dot("$C$", C, dir(C));
            dot("$D$", D, dir(D));
            dot("$O$", O, dir(135));
        \end{asy}
    \end{center}


    \problem In $\triangle ABC$, $\angle A = 90^\circ$. The point $P$ lies
    inside $\triangle ABC$ with distances from $AB$, $BC$ and $CA$ equal to $x$,
    $y$ and $z$ respectively. If we denote $a = BC$, $b = CA$ and $c = AB$, show
    that $z = \frac{bc - cx - ay}{b}$.
    \begin{center}
        \begin{asy}
            import olympiad;
            size(6cm);
            defaultpen(fontsize(11pt));
            pen mydash = linetype(new real[] {5,5});
            usepackage("contour", "outline");
            texpreamble("\contourlength{1pt}");
            pair A = dir(70);
            pair B = dir(180);
            pair C = dir(0);
            pair P = (.1, .3);
            pair X = foot(P, A, B);
            pair Y = foot(P, B, C);
            pair Z = foot(P, C, A);
            draw(A--B--C--cycle, black+1);
            draw(P--X);
            draw(P--Y);
            draw(P--Z);
            draw(rightanglemark(P, Y, C, 3));
            draw(rightanglemark(P, X, B, 3));
            draw(rightanglemark(P, Z, C, 3));
            draw(rightanglemark(B, A, C, 3));
            label("$x$", midpoint(P--X), dir(35));
            label("$y$", midpoint(P--Y), dir(180));
            label("$z$", midpoint(P--Z), dir(-55));
            label("$a$", midpoint(B--C), dir(270));
            label("$b$", midpoint(C--A), dir(35));
            label("$c$", midpoint(A--B), dir(125));
            dot("$A$", A, dir(A));
            dot("$B$", B, dir(B));
            dot("$C$", C, dir(C));
            dot("\contour{white}{$P$}", P, dir(340));
        \end{asy}
    \end{center}


    \problem An integer is chosen from the set $\{1, 2, 3, \ldots, 100\}$. Find
    the probability that the integer is divisible by 3 or 7.

    \problem Aung Aung says to Bo Bo, ``I am 5 times what you were when I was
    your age''.  The sum of their current ages is 64. Find their ages.

    \problem The sum of squares of three consecutive positive integers is 2 more
    than 100 times the sum of the numbers itself. Find the largest of the three
    numbers.

    \problem $P$ and $Q$ are two points on $AB$ and $AC$ respectively, of
    $\triangle ABC$. If $PQ$ is parallel to $BC$, and bisects $\triangle ABC$,
    find $AP : PB$.

    \problem Find the remainder when $(x + 1)^{2016} + (x + 2)^{2016}$ is
    divided by $x^2 + 3x + 2$.

    \problem If $a$, $b$, $c$, $d$ are in harmonic progression, prove that $ab +
    bc + cd = 3ad$.

    \problem Using mathematical induction, prove that 
    \[1^2 + 2^2 + 3^2 + \cdots + n^2 = \frac{n(n + 1)(2n + 1)}{6}.\]

    \problem Show that $n^7 - n$ is divisible by 42, for all positive integers
    $n$.

    \problem If $\alpha$, $\beta$ and $\gamma$ are roots of the equation $x^3 +
    px^2 + qx + k = 0$, show that $\alpha^2 + \beta^2 + \gamma^2 = p^2 - 2q$.
\end{problems}
