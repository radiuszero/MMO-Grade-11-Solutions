\paperheader{2017 National Round}
\begin{problems}
    \problem $x = \overline{ABCDE}$ is a five digit number. If $(A + C + E) -
    (B + D) = 11k$ where $k = -1, 0, 1, 2$, prove that $x$ is divisible by 11. 
    
    \problem A palindrome is a number that remains the same when its digits are
    reversed. For example, 252 is a three-digit palindrome and 3663 is a
    four-digit palindrome. If the numbers $x - 22$ and $x$ are three-digit and
    four-digit palindromes, respectively, find the value of $x$.
    
    \problem 
    \begin{enumerate}
        \item To find the exact value of $\sqrt{4 + 2\sqrt{3}}$, let $\sqrt{4 +
            2\sqrt{3}} = a + b\sqrt{3}$, where $a$ and $b$ are integers and $a
            > b\sqrt{3} > 0$, and compute the exact values of $a$ and
            $b$.\footnote{The original problem asked to find the value of
            $\sqrt{24 - 2\sqrt{3}}$.}

        \item Length of a side of an equilateral triangle $ABC$ is 2. If $AD
            \perp BC$, and the angle bisector of $\angle BAD$ meets $BC$ at
            $E$, show that $\angle AED = 75^\circ$. Using the angle bisector
            theorem, find the exact length of $ED$. Using the diagram, compute
            exact values of $\sin 75^\circ$ and $\cos 75^\circ$.
    \end{enumerate}
    
    \problem Find the $n$th term of a harmonic progression whose first two
    terms are $a$ and $b$. 
    
    \problem Find the relationship between $a$, $b$ and $c$ if the system 
    \begin{align*}
        x + y &= a\\
        x^2 + y^2 &= b\\
        x^3 + y^3 &= c
    \end{align*}
    has solutions. 
    
    \problem A function $f$ is defined on the positive integers, $f(1) = 1009$
    and 
    \[f(1) + f(2) + \cdots + f(n) = n^2 f(n).\]
    \begin{enumerate}
        \item By expressing $f(1) + f(2) + \cdots + f(n - 1)$ in two ways, find
            $\frac{f(n)}{f(n - 1)}$. 
        
        \item By using the result in part(a), find the formula for $f(n)$.
            Calculate $f(2018)$. 
    \end{enumerate}
    
    \problem \begin{enumerate}
        \item $f(x) = \frac{1}{(2x - 1)(2x - 3)}$ can be expressed as $f(x) =
            \frac{A}{2x - 1}+\frac{B}{2x - 3}$, where $A$ and $B$ are
            constants. Find the values of $A$ and $B$. 
        
        \item If $f(3) + f(4) + f(5) + \cdots + f(n) = c - g(n)$, where $c$ is
            a constant and $g(n)$ is a function, determine $c$ and $g(n)$. 
    \end{enumerate}
    
    \problem Find the number of ways in which 5 men, 3 women and 2 children can
    sit at a round table, if 
    \begin{enumerate}
        \item there are no restrictions,
        
        \item each child is seated between 2 women.
    \end{enumerate}
    
    \problem $ABCD$ is a rectangle with $AB = x$, $AD = y$ and $y > x$, and
    $AXYZ$ is a square. If the area of $AXYZ$ is the same as the area of
    $ABCD$, show that $\sqrt{xy} - x\leq BX\leq \sqrt{xy} + x$ and $y -
    \sqrt{xy} \leq DZ\leq y + \sqrt{xy}$. 
    
    \problem The average of the numbers $1, 2, 3, \ldots, 99$ and $x$ is
    $100x$. Find the value of $x$. 
    
    \problem Each of 240 boxes in a line contains a single red marble, and for
    $1 \leq k \leq 240$, the box in the $k$th position also contains $k$ white
    marbles. Phyu Phyu begins at the first box and successively draws a single
    marble at random from each box. She stops when she first draws a red
    marble. Let $\mathbb{P}(n)$ be the probability that Phyu Phyu stops after
    drawing exactly $n$ marbles. What is the smallest value of $n$ for which
    $\mathbb{P}(n) < \frac{1}{240}$. 
    
    \problem For all positive integers $n$, define 
    \[f(n) = 1 - \frac{1}{2} + \frac{1}{3} - \cdots + \frac{1}{2n-1} -
    \frac{1}{2n},\]
    \[g(n) = \frac{1}{n + 1} + \frac{1}{n + 2} + \frac{1}{n + 3} + \cdots +
    \frac{1}{2n}.\]
    By using mathematical induction, prove that $f(n) = g(n)$ for all positive
    integers $n$. 
    
    \problem $ABCD$ is a cyclic quadrilateral with $AB = AC$. The line $PQ$ is
    tangent to the circle at the point $C$, and is parallel to $BD$. Diagonals
    $BD$ and $AC$ intersect at $E$. If $AB = 18$ and $BC = 12$, find the length
    of $AE$. 
    
    \problem Let $a > b > 0$. Define two sequences $a_{n}$ and $b_{n}$ as
    follows:
    \[a_{1} = a,\mathspace b_{1} = b,\mathspace a_{n + 1} = \frac{a_{n} +
    b_{n}}{2},\mathspace b_{n + 1} = \sqrt{a_{n}b_{n}}.\]
    \begin{enumerate}
        \item Prove that $a_{n + 1} < a_{n}$ and $b_{n + 1} > b_{n}$ for $n >
            1$. 
        
        \item Prove that $a_{n + 1} - b_{n + 1} = \frac{(a_{n} -
            b_{n})^2}{8a_{n + 2}}$. 
        
        \item If $a = 4$ and $b = 1$, find the first four terms of each
            sequence of $a_{n}$ and $b_{n}$.
    \end{enumerate}
    
    \problem If $\cos\alpha, \cos\beta$ and $\cos\gamma$ are the roots of the
    equation $x^3 + ax^2 + bx + c = 0$, where $\alpha$, $\beta$ and $\gamma$
    are angles of a triangle, prove that $a^2 = 2b + 2c + 1$. 
    
    \problem The two circles $C_{1}$ and $C_{2}$ intersect at the points $A$
    and $B$. The tangent to $C_{1}$ at $A$ intersects $C_{2}$ at $P$ and the
    line $PB$ intersects $C_{1}$ at $Q$. The tangent to $C_{2}$ drawn from $Q$
    intersects $C_{1}$ and $C_{2}$ at the points $X$ and $Y$ respectively. The
    points $A$ and $Y$ lie on the different sides of $PQ$. Show that $AY$
    bisects $\angle XAP$. 
\end{problems}
