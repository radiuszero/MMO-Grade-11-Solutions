\paperheader{2019 Regional Round}
\begin{problems}
    \problem 
    \begin{enumerate}
        \item Show that $x^2 + y^2 \geq 2xy$ for any two real numbers $x$ and
            $y$, and that the sign of the equality holds if and only if $x=y$. 
        
        \item By using similar triangles, prove that $h = \frac{ab}{c}$ for the
            given figure.
        
        \item By using (1) and (2), show that $h \leq \frac{c}{2}$, and that the
            sign of equality only holds if and only if $a = b$. 
        
        \item Show that of all right triangles having the same length of
            hypotenuse, the isosceles right triangle maximizes the area. 
    \end{enumerate}
    \begin{center}
        \begin{asy}
            import olympiad;
            size(6cm);
            defaultpen(fontsize(11pt));
            pen mydash = linetype(new real[] {5,5});
            real s = -.16;
            pair A = dir(180);
            pair B = dir(0);
            pair A1 = (xpart(A), ypart(A)+s);
            pair B1 = (xpart(B), ypart(B)+s);
            pair M = midpoint(A1--B1);
            real k = .1;
            pair MA = M + k*unit((-1, 0));
            pair MB = M + k*unit((1, 0));
            pair C = dir(110);
            pair D = foot(C, A, B);
            draw(A--B--C--cycle, black+1);
            draw(C--D);
            draw(MA--A1, arrow=Arrow(size=5));
            draw(MB--B1, arrow=Arrow(size=5));
            draw(rightanglemark(A, C, B, 3));
            draw(rightanglemark(B, D, C, 3));
            dot("$A$", A, dir(225));
            dot("$B$", B, dir(315));
            dot("$C$", C, dir(90));
            dot("$D$", D, dir(270));
            label("$b$", midpoint(C--A), dir(135));
            label("$a$", midpoint(C--B), dir(45));
            label("$h$", midpoint(C--D), dir(0));
            label("$c$", M);
            \end{asy}
    \end{center}

    \problem Find all functions over the reals such that $f(x) + 2f(1 - x) =
    x(1 - x)$.
    
    \problem In the given figure, $ABCD$ is a rectangle. $P$ and $Q$ are points
    on $BC$ such that $AB = BP = PQ = QC$. Find two similar but not congruent
    triangles and prove their similarity. 
    \begin{center}
        \begin{asy}
            import olympiad;
            size(7cm);
            defaultpen(fontsize(11pt));
            pen mydash = linetype(new real[] {5,5});
            pair A = (0, 1);
            pair B = (0, 0);
            pair P = (1, 0);
            pair Q = (2, 0);
            pair C = (3, 0);
            pair D = (3, 1);
            draw(A--B--C--D--cycle, black+1);
            draw(A--P);
            draw(A--Q);
            draw(A--C);
            add(pathticks(A--B, s=4));
            add(pathticks(B--P, s=4));
            add(pathticks(P--Q, s=4));
            add(pathticks(Q--C, s=4));
            dot("$A$", A, dir(135));
            dot("$B$", B, dir(225));
            dot("$C$", C, dir(315));
            dot("$D$", D, dir(45));
            dot("$P$", P, dir(270));
            dot("$Q$", Q, dir(270));
        \end{asy}
    \end{center}
    
    \problem Let $f(x) = ax^2 + bx + c$, $a \ne 0$. 
    \begin{enumerate}
        \item Fill in the following steps ($\square$) to show that $f(x) = a(x
            + \frac{b}{2a})^2 - \frac{b^2 - 4ac}{2a}$. 
        \begin{align*}
            f(x) &= a(x^2 + \square x) + c\\
            &= a\left(x^2 + \square x + \square - \left(\frac{b}{2a}\right)^2\right) + c\\
            &= a(x^2 + \square x + \square)- \frac{b^2}{\square} + c\\
            &= a(x + \square)^2 - \frac{b^2}{\square} + c\\
            &= a\left(x + \frac{b}{2a}\right)^2 - \frac{b^2 - 4ac}{4a}
        \end{align*}
        
        \item Hence show that if $a > 0$ and $b^2 - 4ac < 0$, then $f(x) > 0$
            for all $x \in \mathbb{R}$. 
        
        \item Show that if $a < 0$ and $b^2 - 4ac < 0$, then $f(x) < 0$ for all
            $x \in \mathbb{R}$.
        
        \item If $b^2 - 4ac > 0$, find the roots of the equation $f(x) = 0$ in
            terms of $a$, $b$ and $c$. 
    \end{enumerate}
    
    \problem In the given figure, $DC = PB$ and $DC \parallel PB$. $M$ and $N$
    are midpoints of $AC$ and $BD$ respectively. 
    \begin{enumerate}
        \item Prove that the points $P$, $N$, $C$ are collinear. 
        
        \item Prove that $AP \parallel MN$ and $AP = 2MN$. 
    \end{enumerate}
    \begin{center}
        \begin{asy}
            import olympiad;
            size(6cm);
            defaultpen(fontsize(11pt));
            pen mydash = linetype(new real[] {5,5});
            pair B = (2, -.5);
            pair D = -B;
            pair C = (.5, 1.5);
            pair P = -C;
            pair N = (0, 0);
            pair A = (-2.5, -.7);
            pair M = midpoint(A--C);
            draw(D--C--B--P);
            draw(D--A);
            draw(A--P);
            draw(A--C);
            draw(D--B--A);
            draw(N--M);
            dot("$A$", A, dir(180));
            dot("$B$", B, dir(0));
            dot("$C$", C, dir(90));
            dot("$D$", D, dir(135));
            dot("$P$", P, dir(270));
            dot("$M$", M, dir(135));
            dot("$N$", N, dir(45));
        \end{asy}
    \end{center}
    
    \problem In the given figure, $\angle BAC = 60^\circ$, $AB = 24$ cm, $BD
    \perp AC$ and $DC = 3$ cm. Find the diameter of the circle. 
    \begin{center}
        \begin{asy}
            import olympiad;
            size(6cm);
            defaultpen(fontsize(11pt));
            pen mydash = linetype(new real[] {5,5});
            real s = 30;
            pair A = dir(170);
            pair B = dir(s);
            pair C = dir(s-120);
            pair D = foot(B, C, A);
            pair O = (0, 0);
            pair E = -C;
            draw(B--A--C);
            draw(B--D);
            draw(circumcircle(A, B, C));
            draw(rightanglemark(B, D, A, 2.5));
            dot("$A$", A, dir(A));
            dot("$B$", B, dir(B));
            dot("$C$", C, dir(C));
            dot("$D$", D, dir(225));
            label("24", midpoint(A--B), dir(100));
            label("3", midpoint(D--C), dir(225));
        \end{asy}
    \end{center}
    
    \problem In an AP, the $k$th term is 11. The sum of the first $k$ terms is
    26. The sum of the next $k$ terms is 74. Find the first term and the common
    difference. 
    
    \problem What is the radius of the inscribed circle of a 3-4-5 right
    triangle? 
    \begin{center}
        \begin{asy}
            size(5cm);
            import olympiad;
            pair A = (-3, 0);
            pair B = (0, 4);
            pair C = (0, 0);
            pair I = incenter(A, B, C);
            pair P = foot(I, A, C);
            real r = abs(I - P);
            draw(A--B--C--cycle);
            draw(rightanglemark(B, C, A, 5));
            draw(circle(I, r));
            label("$3$", midpoint(A--C), dir(-90));
            label("$4$", midpoint(B--C), dir(0));
            label("$5$", midpoint(A--B), dir(135));
        \end{asy}
    \end{center}
    
    \problem A bag contains 3 red balls and 2 green balls. Balls are drawn at
    random, one at a time but not replaced, until all 3 of red balls are drawn
    or until both green balls are drawn. What is the probability that the 3
    reds are drawn? 
    
    \problem If 75! is divisible by $5^n$, find the maximum value of $n$. 
    
    \problem If $L$, $M$, $N$ are the midpoints of the sides of $\triangle
    ABC$, and $P$ is the foot of perpendicular from $A$ to $BC$, prove that
    $L$, $P$, $M$, $N$ are concyclic.
    \begin{center}
        \begin{asy}
            import olympiad;
            size(6cm);
            defaultpen(fontsize(11pt));
            pen mydash = linetype(new real[] {5,5});
            pair A = dir(120);
            pair B = dir(210);
            pair C = dir(330);
            pair M = midpoint(B--C);
            pair N = midpoint(C--A);
            pair L = midpoint(A--B);
            pair P = foot(A, B, C);
            draw(A--B--C--cycle, black+1);
            draw(A--P);
            draw(rightanglemark(A, P, C, 2.5));
            dot("$A$", A, dir(A));
            dot("$B$", B, dir(225));
            dot("$C$", C, dir(315));
            dot("$M$", M, dir(270));
            dot("$N$", N, dir(45));
            dot("$L$", L, dir(135));
            dot("$P$", P, dir(270));
        \end{asy}
    \end{center}
    
    \problem In a GP, the $k$th term is 864. The sum of the first $k$ terms is
    2080. The sum of the first $2k$ terms is 12610. Find the first term and the
    common ratio.
    
    \problem Prove that $\binom{n}{m}\binom{m}{k} = \binom{n}{k}\binom{n - k}{m
    - k}$ if all variables are integers and $n \geq m \geq k \geq 0$.
    
    \problem
    \begin{enumerate}
        \item Prove that $n \leq -k^2 + nk + k \leq \frac{(n + 1)^2}{4}$ for $1
            \leq k \leq n$. 
        
        \item Consider $k(n + 1 - k) = -k^2 + nk + k$ and by using the
            inequalities in (1), prove that \[n^n \leq (n!)^2 \leq \frac{(n +
            1)^{2n}}{4^n}.\]
        
        \item Hence prove that \[n^{\frac{n}{2}} \leq n! \leq \frac{(n +
            1)^n}{2^n}.\]
    \end{enumerate}
\end{problems}
