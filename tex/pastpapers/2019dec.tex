\paperheader{2019 National Round}
\begin{problems}
    \problem $\triangle ABC$ is an equilateral triangle. 
    \begin{enumerate}
        \item Prove that there are points $D$, $E$ and $F$ on $AB$, $BC$ and
            $CA$ respectively such that $DE \perp BC$, $EF\perp CA$ and $FD
            \perp AB$. 

        \item Prove that $\triangle DEF$ is also an equilateral triangle.

        \item Find the ratio of the perimeters of $\triangle DEF$ and
            $\triangle ABC$.
        
        \item Find the ratio of the areas of $\triangle DEF$ and $\triangle
            ABC$. 
    \end{enumerate}
    \begin{center}
        \begin{asy}
            import olympiad;
            size(5cm);
            defaultpen(fontsize(11pt));
            pen mydash = linetype(new real[] {5,5});
            pair A = dir(90);
            pair B = dir(210);
            pair C = dir(330);
            pair D = (2/3)*A+(1/3)*B;
            pair E = foot(D, B, C);
            pair F = foot(E, A, C);
            pair M = midpoint(B--C);
            draw(A--B--C--cycle, black+1);
            draw(D--E--F--cycle);
            dot("$A$", A, dir(A));
            dot("$B$", B, dir(225));
            dot("$C$", C, dir(315));
            dot("$E$", E, dir(270));
            dot("$F$", F, dir(30));
            dot("$D$", D, dir(120));
        \end{asy}
    \end{center}
    
    \problem
    \begin{enumerate}
        \item If $abc = 2ab + 2bc +2ca$ where $a$, $b$ and $c$ are integers and
            $1 \leq a \leq b \leq c$, then show that $3 \leq a \leq 6$.
        
        \item Find all possible ordered triples $(a,b,c)$ such that \[abc = 2ab
            + 2bc + 2ca\] where $a$, $b$ and $c$ are integers and $1 \leq a
            \leq b \leq c$. 
    \end{enumerate}
    
    \problem Let $\mathbb{Z}$ be the set of integers. Let $f:\mathbb{Z}
    \rightarrow \mathbb{Z}$ be a function from $\mathbb{Z}$ to $\mathbb{Z}$
    such that 
    \[f(x + y) + f(x - y) = 2f(x) + 2f(y),\]
    for all integers $x$ and $y$. 
    \begin{enumerate}
        \item Show that $f(0) = 0$ and $f(2x) = 4f(x)$. 
        
        \item Show that $f(nx) = n^2f(x)$ for every positive integer $n$. 
        
        \item Show that $f(y) = f(-y)$. 
        
        \item Determine all functions $f:\mathbb{Z} \rightarrow \mathbb{Z}$
            such that 
            \[f(x + y) + f(x - y) = 2f(x) + 2f(y)\] 
            for all integers $x$ and $y$. 
    \end{enumerate}
    
    \problem Seven integers are written around a circle in a way that no two or
    three adjacent numbers have a sum divisible by 3. How many of these seven
    numbers are divisible by 3? 
    
    \problem 
    \begin{enumerate}
        \item Find an ordered pair $(x,y)$ such that \[2019x + 2021y = 1\]
            where $x$ and $y$ are integers. 

        \item By using the ordered pair obtained in question 1, find all
            solutions of 
            \[2019x + 2021y = 1\] 
            where $x$ and $y$ are integers. 
    \end{enumerate}
    
    \problem In $\triangle ABC$, altitudes $AD$, $BE$ and $CF$ pass through the
    point $H$. Points $A'$, $B'$ and $C'$ are midpoints of $BC$, $CA$, $AB$
    respectively. Points $A''$, $B''$ and $C''$ are midpoints of $AH$, $BH$,
    $CH$ respectively. 
    \begin{enumerate}
        \item Prove that $B'C'B''C''$ is a rectangle. 
        
        \item Prove that $C'A'C''A''$ is a rectangle.
        
        \item Prove that the six points $A'$, $B'$, $C'$,$A''$, $B''$, $C''$
            are concyclic. 
        
        \item Prove that the nine points $A',B',C',A'',B'',C'',D,E,F$ are
            concyclic. 
    \end{enumerate}
    \begin{center}
        \begin{asy}
            import olympiad;
            size(6cm);
            defaultpen(fontsize(11pt));
            pen mydash = linetype(new real[] {5,5});
            usepackage("contour", "outline");
            texpreamble("\contourlength{1pt}");
            pair A = dir(120);
            pair B = dir(210);
            pair C = dir(330);
            pair D = foot(A, B, C);
            pair E = foot(B, C, A);
            pair F = foot(C, A, B);
            pair H = A+B+C;
            pair A1 = midpoint(B--C);
            pair B1 = midpoint(C--A);
            pair C1 = midpoint(A--B);
            pair A2 = midpoint(A--H);
            pair B2 = midpoint(B--H);
            pair C2 = midpoint(C--H);
            draw(A--B--C--cycle, black+1);
            draw(A--D);
            draw(B--E);
            draw(C--F);
            dot("$A$", A, dir(A));
            dot("$B$", B, dir(225));
            dot("$C$", C, dir(315));
            dot("$A'$", A1, dir(270));
            dot("$B'$", B1, dir(45));
            dot("$C'$", C1, dir(165));
            dot("\contour{white}{$A''$}", A2, dir(45));
            dot("\contour{white}{$B''$}", B2, dir(270));
            dot("\contour{white}{$C''$}", C2, dir(270));
            dot("$H$", H, dir(305));
            dot("$D$", D, dir(270));
            dot("$E$", E, dir(45));
            dot("$F$", F, dir(165));
        \end{asy}
    \end{center}
    
    \problem Let $f:\mathbb{R} \rightarrow \mathbb{R}$ be a function such that 
    \[f(x + y) + f(x - y) = 2f(x)\cos y,\mathspace x, y \in \mathbb{R}.\]
    \begin{enumerate}
        \item Show that $f(\theta) + f(-\theta) = 2a\cos \theta$, where $a =
            f(0)$. 
        
        \item Show that $f(\theta + \pi) + f(\theta) = 0$.
        
        \item Show that $f(\theta + \pi) + f(-\theta) = -2b\sin\theta$, where
            $b = f\left(\frac{\pi}{2}\right)$.
        
        \item Find all functions $f:\mathbb{R} \rightarrow \mathbb{R}$ such
            that $$f(x + y) + f(x - y) = 2f(x)\cos y$$ where $x,y \in
            \mathbb{R}$.
    \end{enumerate}
    
    \problem A student council must select a two-person welcoming committee and
    a three-person planning committee from its members. There are exactly 15
    ways to select a two-person team for the welcoming committee. It is
    possible for students to serve on both committees. In how many different
    ways can a three-person planning committee be selected? 
    
    \problem $AB$ is a diameter of a circle $O$ with radius 10 cm. $OQ$ is a
    radius of a circle $O$ such that $QO \perp AB$. A point $P$ is on $OQ$.
    Draw a semicircle centered at $P$ with diameter $CD$ where $CD$ is the
    chord of circle $O$ and $CD \perp PQ$. $PQ$ produced meets the semicircle
    at $R$. Find the maximum possible length of $QR$. 
    \begin{center}
        \begin{asy}
            import olympiad;
            size(6cm);
            defaultpen(fontsize(11pt));
            pen mydash = linetype(new real[] {5,5});
            pair A = dir(180);
            pair B = dir(0);
            pair Q = dir(90);
            real s = 50;
            pair D = dir(s);
            pair C = dir(180-s);
            pair P = midpoint(C--D);
            pair O = (0, 0);
            pair R = P+abs(P-D)*unit(P-O);
            draw(circle(O, 1));
            draw(A--B);
            draw(O--R);
            draw(C--D);
            draw(arc(P, abs(P-D), 0, 180));
            draw(rightanglemark(R, P, D, 2.5));
            draw(rightanglemark(R, O, B, 2.5));
            dot("$A$", A, dir(180));
            dot("$B$", B, dir(0));
            dot("$O$", O, dir(270));
            dot("$P$", P, dir(225));
            dot("$C$", C, dir(165));
            dot("$D$", D, dir(15));
            dot("$Q$", Q, dir(135));
            dot("$R$", R, dir(90));
        \end{asy}
    \end{center}
    
    \problem Find all positive integers $s$ such that $\ceil*{\frac{s}{3}} - 21
    = \ceil*{\frac{s}{5}}$ where $\ceil*{x}$ is the smallest integer greater
    than or equal to $x$. For example, $\ceil*{3.7} = 4$, $\ceil*{3} = 3$ and
    $\ceil*{3.2} = 4$. 
    
    \problem In $\triangle ABC$, $AB = 94$ and $AC = 107$. A circle with center
    $A$ and radius $AB$ intersects $BC$ at points $B$ and $X$. Moreover, $BX$
    and $CX$ have integer lengths. What is $BC$? 
    
    \problem Seven people are sitting around a circular table, each holding a
    fair coin. All seven people flip their coins and those who flip heads stand
    while those flip tails seated. What is the probability that no two people
    adjacent will stand? 
    
    \problem
    \begin{enumerate}
        \item Let $a$ and $b$ be positive integers. If there are integers
            $x_{0}$, $y_{0}$ such that $ax_{0} + by_{0} = 1$, then prove that
            the greatest common divisor of $a$ and $b$ is 1. 
        
        \item Prove that the fraction $\frac{12n + 5}{14n + 6}$ is in lowest
            terms for every positive integer $n$. 
    \end{enumerate}
    
    \problem $\triangle ABC$ is inscribed in a circle. Altitudes $AD$, $BE$ and
    $CF$ pass through the point $H$. $EF$ produced meets the circle at $P$.
    $BP$ produced and $DF$ produced meet at the point $Q$. 
    \begin{enumerate}
        \item Show that $\angle ACF = \angle ADF = \angle ABE$.
        
        \item Show that $\angle AFQ = \angle ACD$. 
        
        \item Show that $AP = AQ$. 
    \end{enumerate}
    \begin{center}
        \begin{asy}
            import olympiad;
            size(6cm);
            defaultpen(fontsize(11pt));
            pen mydash = linetype(new real[] {5,5});
            usepackage("contour", "outline");
            texpreamble("\contourlength{1pt}");
            pair A = dir(60);
            pair B = dir(210);
            pair C = dir(330);
            pair H = orthocenter(A, B, C);
            pair D = foot(A, B, C);
            pair E = foot(B, C, A);
            pair F = foot(C, A, B);
            pair O = (0, 0);
            pair P[] = intersectionpoints(E--(F+unit(F-E)), circle(O, 1));
            pair P = P[0];
            pair Q = extension(B, P, D, F);
            draw(A--B--C--cycle, black+1);
            draw(A--D);
            draw(B--E);
            draw(C--F);
            draw(Q--B);
            draw(Q--D);
            draw(P--E);
            draw(P--A);
            draw(Q--A);
            draw(rightanglemark(A, D, B, 2.5));
            draw(rightanglemark(B, E, C, 2.5));
            draw(rightanglemark(C, F, A, 2.5));
            draw(circle(O, 1));
            dot("$A$", A, dir(45));
            dot("$B$", B, dir(225));
            dot("$C$", C, dir(315));
            dot("$D$", D, dir(270));
            dot("$E$", E, dir(45));
            dot("\contour{white}{$F$}", F, dir(80));
            dot("$P$", P, dir(180));
            dot("$Q$", Q, dir(135));
            dot("\contour{white}{$H$}", H, dir(315));
        \end{asy}
    \end{center}
    
    \problem[fixed] A car drives from town $A$ to $B$ at the average speed of 30 km/h,
    from town $B$ to town $C$ at average speed of 60 km/h; and on the way back,
    the car drives from $C$ to $B$ at average speed of 30 km/h\footnote{The
    original problem set this to be 20 km/h, but in that case the total
    distance is not unique anymore.}, from town $B$ to $A$ at average speed of 60
    km/h. The whole trip takes 6 hours. What is the total distance of the round
    trip?
    
    \problem 
    \begin{enumerate}
        \item Prove that the square of an odd number gives the remainder 1 upon
            dividing by 8. 
        
        \item Prove that if $k$ is odd and $n$ is a positive integer, then
            $k^{2^n} - 1$ is divisible by $2^{n + 2}$. 
    \end{enumerate}
\end{problems}
