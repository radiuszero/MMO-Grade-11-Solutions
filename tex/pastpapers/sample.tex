\paperheader{Sample Problems}
\begin{problems}
    \problem Given the sequence $u_{n} = n^2 - 6n + 13$, what is the smallest
    term in the sequence?

    \problem A function $f(x) = \frac{p}{x - q}$, $q > 0$, $q \ne x$ is such
    that $f(p) = p$ and $f(2q) = 2q$. Find the values of $p$ and $q$. If $(f
    \circ f)(x) = x$, show that $x^2 - x - 2 = 0$.  \footnote{The problem
    originally stated $f(q) = q$, but this does not give $x^2 - x - 2 = 0$ when
    $(f \circ f)(x) = x$.} 
    
    \problem Let $f(x) = 1 + x + x^2 + \cdots + x^n$. The remainder when $f(x)$
    is divided by $2x - 1$ is $\frac{341}{256}$ more than the remainder when
    $f(x)$ is divided by $2x + 1$. Find the value of $n$.

    \problem[error] Show that 
    $\matr{A} = \begin{bmatrix}
                    \cos^2\theta & \sin\theta \\
                    \sin\theta   & \cos\theta
                \end{bmatrix}
    $
    satisfies $\matr{A}^2 + \matr{I} = 2\matr{A}\cos \theta$, where 
    $\matr{I} = \begin{bmatrix}
                    1 & 0 \\ 
                    0 & 1
                \end{bmatrix}.
    $ 
    If $\theta = 30^\circ$, show that $\matr{A}' - \matr{A}^{-1} = \matr{0}$.

    \problem In the binomial expansion of $(1 + x)^n$, the coefficients of 5th,
    6th and 7th terms are consecutive terms of an AP. Find the first three terms
    of the binomial expansion.

    \problem Find the number of integers between 100 and 999 such that the sum
    of the three digits is 12.

    \problem Determine the solution set for which 
    \[ \frac{(x + 1)(x - 2)}{1 - 2x} > 0. \]

    \problem If $a_{1}$, $a_{2}$, $a_{3}$ are in AP, $a_{2}$, $a_{3}$, $a_{4}$
    are in GP and $a_{3}$, $a_{4}$, $a_{5}$ are in HP, then prove $a_{1}$,
    $a_{3}$, $a_{5}$ are in GP.

    \problem Prove by mathematical induction that 
    \[\sum_{r = 1}^n r^{4} = \frac{n(n + 1)(6n^3 + 9n^2 + n - 1)}{30}.\]

    \problem[error] If the equation $x^4 - 4cx^3 + 6x^2 + x + 1 = 0$ has a
    repeated root $p$, show that $3c = \frac{p^2 + 3}{p}$. Hence or otherwise,
    prove that there is only one positive value $c$ giving a repeated root, and
    that this value of $c$ is $(\frac{4}{3})^{\frac{3}{4}}$.

    \problem If the base $BC$ of $\triangle ABC$ is trisected at $P$ and $Q$,
    show that 
    \[AB^2 + AC^2 = AP^2 + AQ^2 + 4PQ^2.\]

    \problem From a point $O$, two straight lines are at any angle. On one of
    these lines, points $A$ and $B$ are taken such that $OA = \frac{5}{2}$ inch
    and $AB = \frac{3}{2}$ inch. Find the point on the other line at which $AB$
    subtends the greatest angle.

    \problem Given any seven distinct real numbers $x_{1}$, $x_{2}$, $\ldots$,
    $x_{7}$, prove that we can always find the numbers $x_{i}, 1\leq i\leq 7$
    and $x_{j},1\leq j\leq 7$ such that 
    \[ 0 < \frac{x_{i} - x_{j}}{1 + x_{i}x_{j}} < \frac{1}{\sqrt{3}}. \]

    \problem Prove that $6 \mid n^3 - n$ for all integers $n$.

    \problem[error] $ABCD$ is a semicircle, $A$ and $B$ being the extremities of
    the diameter and $C$ and $D$ being points in the arc in which the ratio of
    arcs $AB : BC : CD = 2 : 3 : 4$. Find $\angle ADC$.

    \problem The 5-digit number $\overline{A986B}$ is divisible by 72. What is
    the value of $A + B$?

    \problem $AB$ is a chord of a circle and $P$ is any point on the arc of one
    of the segments cut off. Prove that the bisection of the $\angle APB$ passes
    through a fixed point on the circumference.

    \problem Two circles touch each other internally at $A$. Through $B$, a
    point on the circumference of the inner circle, a tangent is drawn which
    meets the circumference of the outer at $P$ and $Q$. Show that \mbox{$AP :
    AQ = BP : BQ$}.
\end{problems}
