\paperheader{2018 Regional Round}
\begin{problems}
    \problem In the figure, $AP$, $BQ$, $CR$ are perpendicular to the straight
    line $ABC$. Prove that 
    \begin{enumerate}
        \item $\triangle PAB \sim \triangle RCB$
        
        \item $\frac{1}{BQ} = \frac{1}{AP} + \frac{1}{CR}$. 
    \end{enumerate}
    \begin{center}
        \begin{asy}
            import olympiad;
            size(6cm);
            defaultpen(fontsize(11pt));
            pen mydash = linetype(new real[] {5,5});
            pair A = (-1, 0);
            pair B = (0, 0);
            pair C = (1.5, 0);
            pair P = (-1, 2/3);
            pair Q = extension(B, rotate(90, B)*A, P, C);
            pair R = extension(A, Q, rotate(90, C)*A, C);
            draw(A--R--C--P--cycle);
            draw(A--C);
            draw(P--B--R);
            draw(Q--B);
            dot("$A$", A, dir(225));
            dot("$C$", C, dir(315));
            dot("$B$", B, dir(270));
            dot("$P$", P, dir(135));
            dot("$Q$", Q, dir(90));
            dot("$R$", R, dir(45));
        \end{asy}
    \end{center}
    
    \problem $PA$ and $PB$ are the tangent segments at $A$ and $B$ to a circle
    whose center is $O$. $AB$ and $OP$ are intercept at $Q$. Prove that $AB
    \perp OP$. Hence show that $OQ : QP = AO^2 : AP^2$. Hence also show that
    $\alpha(\triangle OAB) : \alpha(\triangle PAB)=AO^2 : AP^2$. 
    
    \problem If $T_{1}$, $T_{2}$, $T_{3}$ are the sums of $n$ terms of three
    series in AP, the first term of each being $a$ and the respective common
    differences being $d$, $2d$, $3d$, then show that $T_{1} + T_{3} = 2T_{2}$. 
    
    \problem The positive difference between the zeros of the quadratic
    expression $x^2 + kx + 3$ is $\sqrt{69}$. Find the possible values of $k$. 
    
    \problem In $\triangle PQR$, $\angle Q = 90^\circ$ and $S$ is a point on
    $PR$ such that $QS \perp PR$. If $PR = kQR$, then show that $PS = (k^2 -
    1)RS$. 
    
    \problem Prove the following theorem:
    
    If a ray from the vertex of an angle of a triangle divides the opposite
    side into segments that have the same ratio as the other two sides, then it
    bisects the angle.

    \problem The sum to $k$ terms of an AP is 21. The sum to $2k$ terms is 78.
    The $k$th term is 11. Find the first term and the common difference. 
    
    \problem Prove that if $a$, $b$, $c$ and $d$ are positive, the equation
    $x^4 + bx^2 + cx - d = 0$ has one positive, one negative and two imaginary
    roots. 
    
    \problem Show that the sum of the squares of the first $n$ odd numbers is
    $\frac{1}{3}n(4n^2 - 1)$. 
    
    \problem If $100!$ is divisible by $7^{n}$, find the maximum value of $n$. 
    
    \problem Show that $x = 10^\circ$ is a solution of $2\sin x=\frac{1 +
    \tan^2x}{3 - \tan^2x}$. 
    
    \problem Show that $\frac{1}{\sqrt{1}} + \frac{1}{\sqrt{2}} + \cdots +
    \frac{1}{\sqrt{n}} > \sqrt{n}$ for all $n > 1$. 
    
    \problem 
    \begin{enumerate}
        \item Prove the following Cauchy inequality:
            \par For any real numbers $a_1, \ldots, a_n$ and $b_1, \ldots, b_n$,
            \[(a_{1}b_{1} + a_{2}b_{2} + \cdots +a_{n}b_{n})^2\leq (a_{1}^2 +
            a_{2}^2 + \cdots + a_{n}^2)(b_{1}^2 + b_{2}^2 + \cdots +
            b_{n}^2).\]
        
        \item For a set of positive real numbers $x_1, \ldots, x_n$, the
            Root-Mean Square RMS is defined by the formula 
            \[\text{RMS} = \sqrt{\frac{x_{1}^2 + \cdots + x_{n}^2}{n}}\] 
            and the Arithmetic Mean AM is defined by the formula 
            \[\text{AM} = \frac{x_{1} + \cdots + x_{n}}{n}.\] 
            Prove that $\text{RMS} \geq \text{AM}$.
    \end{enumerate}
    
    \problem 
    \begin{enumerate}
        \item Prove that if $p$ is a prime number, then the coefficient of
            every term in the expansion of $(a + b)^p$ except the first and
            last is divisible by $p$. 
        
        \item Hence show that if $p$ is a prime number and $N$ is a positive
            integer, then $N^{p} - N$ is a multiple of $p$. 
        
        \item Hence also show that if $p$ is a prime number, then $10^p - 7^p -
            3$ is divisible by $p$. 
    \end{enumerate}
\end{problems}
