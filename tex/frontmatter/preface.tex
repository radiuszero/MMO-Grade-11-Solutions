\chapter{Preface}
This is a compilation of solutions of all the problems that have appeared
previously in the Grade-11 (Tenth Standard) level mathematical olympiads
organized by the Mathematical Society of Myanmar (MSM). Most of the solutions
here are my own work, while the others were either suggested to me by my
friends or found online. I've tried my best to make these solutions as
accessible and self-contained as possible. However, you should ideally learn
the theory from other more comprehensive textbooks; what I've provided here is
just a brief review.

\addsec{Document Properties}
\subsection*{Format}
The first part of the document contains all the theorems that I've
used in the solutions that may not be part of the standard
curriculum. %\footnote{which, of course, is changing all the time.}.
The latter two parts contains the question papers and solutions from 2015 to
2019. As a side note, the IMO team members for year $N$ were chosen from
students who had passed the Regional Round and the National Round of year $N -
1$; the team selection results were usually announced around the start of May of
year $N$. As of 2023, this system has 
\href{https://www.facebook.com/groups/194801247376848/posts/2211275779062708/}%
{changed completely}.

\subsection*{Colors}
You may notice that some of the problems are colored red or yellow. The problems
colored in red contain major typos that cannot be fixed reasonably, whereas the
ones colored in yellow contain only minor ones that have been fixed. In such
cases, I will usually include the original statement as well.

\subsection*{Navigation}
For any problem in the question papers, you can go straight to its solution by
clicking the boxed number beside it. Similarly, you can go back to the questions
by clicking the problem number of the solution. Here's a diagram in case you are
not sure about what to click:
\[
    \fcolorbox{black}{white}{\makebox[0.45cm]{\normalfont\large 1}}
    \qquad
    \longleftrightarrow
    \qquad
    \text{\sffamily\bfseries\color{RoyalBlue} Problem 1.}
\]

\addsec{Further Reading}
If you are a complete beginner at math olympiads and you want to learn more
theory before solving problems, I highly recommend taking a look at Evan's
recommendations at
\url{https://web.evanchen.cc/wherestart.html}.
\href{https://drive.google.com/drive/folders/1p4-khER6GDk9UJ7AWCp-sSLoB_TCruyV?usp=drive_link}{Here}
is a collection of all the past question papers by ko Phyoe Min Khant.
I've also collected a bunch of handouts and books specifically on geometry that
you can find
\href{https://drive.google.com/drive/folders/1ABGT3_tQgPxxL_Ri81CVpBOYeMzgH-Wt?usp=drive_link}{here}.
Of course, you need to be somewhat familiar with English to make use of these,
but this is inevitable since most math books that you can buy at bookstores in
Myanmar are written in English anyway. However, I don't recommend buying most of them
since they are usually badly written and contain a lot of typos, and you can
easily find much higher quality books and resources online without paying a
dime.

\addsec{Personal Notes}
I started writing these solutions in the summer of 2021. At the time, there was
no concise compilation of solutions for past mathematical olympiads, and the only way
to find the question papers and solutions was by scrolling through countless
Facebook posts. Since one of my friends had already started writing the
solutions for the Grade-10 level olympiads, I decided to take on the Grade-11
ones instead.

In the next few months, I spend countless hours finding the shortest solutions,
deciding on the best ways to present them, and tinkering with \LaTeX/Asymptote.
The progress was slow but steady, and I was feeling that I could release the
notes by the end of August. However, in the middle of July, I got an offer from
the only university to which I had applied. Suddenly, I had a long list in my
hand of tasks to do, and this project ended up at the very bottom of that list.

Fast forward a few years to this summer, I suddenly remembered about
the project while browsing through GitHub. Even though two years have passed, the
situation in the olympiad community here remains almost unchanged: resources are
scarce, hard to find, and often locked behind paywalls. This gave me an
incentive to finish what I had started writing long ago. Competing in math
olympiads changed my life, and even though I am no longer an active participant,
I still wanted to give something back to the community. These notes are the
realization of those aspirations.

\addsec{Acknowledgements}
I would like to give special thanks to Hein Thant for finding mistakes in
several problems and suggesting alternate solutions, ko Hein Thant Aung for
teaching me some graph theory and in particular the method of \hyperref[teq:
DC]{deletion-contraction}, and ko Phyoe Min Khant for compiling all the question
papers which made my life much easier.
