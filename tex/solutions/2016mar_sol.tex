\solutionheader{2015 National Round}
\begin{question}
    If $x^{n} + py^{n} + qz^{n}$ is divisible by $x^2 - (ay + bz)x + abyz$ and
    $y, z \ne 0$, show that 
    \[\frac{p}{a^n} + \frac{q}{b^n} + 1 = 0.\] 
\end{question}
\begin{solution}
    Since $x^2 - (ay + bz)x + abyz = (x - ay)(x - bz)$, it follows that both $x
    - ay$ and $x - bz$ divide $f(x) = x^n + py^n + qz^n$. Therefore, by the
    \hyperref[thm: remainder]{factor theorem},
    \[ f(ay) = a^ny^n + py^n + qz^n = 0, \] 
    and 
    \[ f(bz) = b^nz^n + py^n + qz^n = 0.\]
    From the above two equations, we have
    \[ a^ny^n = b^nz^n = -py^n - qz^n.\]
    Then the identity we want to prove is just
    \begin{align*}
        \frac{p}{a^n} + \frac{q}{b^n} + 1 &= \frac{py^n}{a^ny^n} + \frac{qz^n}{b^nz^n} + 1\\
        &= \frac{py^n + qz^n}{-(py^n + qz^n)} + 1\\
        &= -1 + 1 = 0. \qedhere
    \end{align*}
\end{solution}

\begin{question}
    A sequence is defined by $u_{1} = 1$, $u_{n + 1} = u_{n}^2 - ku_{n}$, where
    $k \ne 0$ is a constant. If $u_{3} = 1$, calculate the value of $k$ and
    find the value of
    \[u_{1} + u_{2} + u_{3} + \cdots + u_{100}.\]
\end{question}
\begin{solution}
    First, let's find the value of $k$.
    \[ u_1 = 1, \quad u_2 = 1 - k \quad\text{and}\quad u_3 = (1 - k)^2 - k(1 - k) = 1 - 3k + 2k^2 = 1. \]
    Since $k \ne 0$, this implies that $k = \frac{3}{2}$, and hence $u_2 = 1 -
    \frac{3}{2} = -\frac{1}{2}$. Now comes the key fact. Notice that the value
    of $u_{n + 1}$ is determined entirely by $u_n$, i.e. if $u_a = u_b$, $u_{a
    + 1} = u_{b + 1}$ as well (Check why this is true.) Therefore, we can
    deduce that $u_4 = -\frac{1}{2}$, $u_5 = 1$, and so on. In general, terms
    with odd indices are equal to 1 and those with even indices are equal to
    $-\frac{1}{2}$. Therefore,
    \[ u_1 + u_2 + u_3 + \cdots + u_{100} = 1 - \frac{1}{2} + 1 - \cdots -
    \frac{1}{2} = 50 - \frac{50}{2} = \frac{50}{2} = 25. \qedhere \]
\end{solution}

\begin{question}
    A rectangular room has a width of $x$ yards. The length of the room is 4
    yards longer than its width. Given that the perimeter of the room is
    greater than 19.2 yards and the area of the room is less than 21 square
    yards, find the set of possible values of $x$.
\end{question}
\begin{solution}
    The length of the room is $x + 4$ yards, so the problem conditions give
    \[ 2x + 2(x + 4) > 19.2 \quad\text{and}\quad x(x + 4) < 21. \]
    Solving the first inequality gives $x > 2.8$. Now let's solve the second
    inequality. It is equivalent to
    \[ x^2 + 4x - 21 < 0 \Longleftrightarrow (x + 7)(x - 3) < 0.\]
    Therefore, $x > 3$ and $x < -7$ or $x < 3$ and $x > -7$. The former is
    absurd, so it follows that $-7 < x < 3$. Since $x > 2.8$ by the first
    inequality, the solution set is $\{ x \text{ yards} \mid x \in \mathbb{R}
    \text{ and } 2.8 < x < 3 \}$.
\end{solution}

\begin{question}
    Two dice are thrown. Event $A$ is that the sum of the numbers on the dice
    is 7. Event $B$ is that at least one number on the die is 6. Find 
    \begin{enumerate}
        \item $\mathbb{P}(A)$, 
        
        \item $\mathbb{P}(B)$,
        
        \item $\mathbb{P}(A \cap B)$, 
        
        \item $\mathbb{P}(A) \cdot \mathbb{P}(B)$.
    \end{enumerate}
    Are $A$ and $B$ independent? 
\end{question}
\begin{solution}
    Let $S = \{ (x, y) \mid 1 \leq x, y \leq 6 \}$ be the set where $x$ is the
    number on the first die and $y$ is the number on the second die. 
    \begin{enumerate}
        \item First, let's calculate the number of elements of $S$ where $x + y
            = 7$. As $x$ ranges from 1 to 6, the value of $y$ must be $7 - x$,
            so there are 6 such pairs. Meanwhile, the total number of pairs in
            $S$ is $6 \times 6 = 36$. Therefore, $\mathbb{P}(A) = \frac{6}{36}
            = \frac{1}{6}$.

        \item Now let's calculate the number of elements of $S$ such that at
            least one of $x$ or $y$ is 6. When $x \in \{1, 2, 3, 4, 5\}$, $y$
            must be 6 so there are 5 such pairs. Now when $x = 6$, $y$ can be
            any number so there are 6 such pairs in this case. Therefore, the
            total number of pairs where at least one of $x$ and $y$ is 6 is $5
            + 6 = 11$. Hence $\mathbb{P}(B) = \frac{11}{36}$. 

        \item Out of 6 pairs that satisfy $A$, it is easy to see that only 2 of
            them, namely $(1, 6)$ and $(6, 1)$ also satisfy $B$. Therefore,
            $\mathbb{P}(A \cap B) = \frac{2}{36} = \frac{1}{18}$.

        \item This one is a straightforward computation; $\mathbb{P}(A) \cdot
            \mathbb{P}(B) = \frac{1}{6} \cdot \frac{11}{36} = \frac{11}{216}$. 
    \end{enumerate}

    If $A$ and $B$ are independent, then $\mathbb{P}(A \cap B) = \mathbb{P}(A)
    \cdot \mathbb{P}(B)$. But obviously this is not true in our case, so we
    conclude that $A$ and $B$ are not independent. 
\end{solution}

\begin{question}
    In $\triangle ABC$, $\angle A = 30^\circ$, $AB = 8$ cm and $BC = x$ cm. If
    $\angle C > 30^\circ$, determine the set of all possible values of $x$.
\end{question}
\begin{solution}
    Let $\ell$ be a ray originating at $A$ inclined to line $AB$ at a
    $30^\circ$ angle. Let $C'$ be the point different from $A$ on $\ell$ such
    that $\angle BC'A = 30^\circ$.
    \begin{center}
        \begin{asy}
            import olympiad;
            size(7cm);
            defaultpen(fontsize(11pt));
            pen mydash = linetype(new real[] {5,5});
            pair A = (0, 0);
            pair B = (8, 0);
            pair D = foot(B, A, dir(30));
            pair C1 = 2*D - A;
            pair C2 = C1 + 2*unit(C1);
            draw(A--C2, arrow=Arrow(size=10));
            draw(A--B);
            draw(B--C1, mydash);
            draw(B--D, mydash);
            draw(rightanglemark(B, D, A, 15));
            dot("$A$", A, dir(225));
            dot("$B$", B, dir(270));
            dot("$C'$", C1, dir(120));
            dot("$D$", D, dir(120));
        \end{asy}
    \end{center}
    Since $\angle BCA < 30^\circ$, it follows that $C$ must lie inside segment
    $AC'$. Therefore, $x < BA = 8 \text{ cm}$. It remains to find the minimum
    value of $x$. Let $D$ be the foot of perpendicular from $B$ to $l$. As $x$
    is minimum when $C = D$, 
    \[ x \geq BD = AB \sin 30^\circ = \frac{AB}{2} = 4 \text{ cm}. \]
    Therefore, the set of possible values of $x$ is $\{ x \text{ cm} \mid x \in
    \mathbb{R} \text{ and } 4 \leq x < 8 \}$.
\end{solution}

\begin{question}
    Let $f(x)$ be a polynomial with real coefficients. When $f(x)$ is divided
    by both $x - a$ and $x - b$, where $a$ and $b$ are distinct real numbers,
    the remainder is a real constant $r$. Prove that $f(x)$ has also the
    remainder $r$ when it is divided by $x^2 - (a + b)x + ab$.
\end{question}
\begin{solution}
    Let $g(x) = f(x) - r$. By the remainder theorem, $f(a) = f(b) = r$, so
    $g(a) = g(b) = 0$. This means that $x - a$ and $x - b$ are factors of
    $g(x)$. Therefore, there exists a polynomial $h(x)$ such that 
    \[ g(x) = (x - a)h(x) \Longleftrightarrow 0 = g(b) = (b - a)h(b).\]
    Since $a \ne b$, $h(b) = 0$ so $x - b$ is also a factor of $h(x)$.
    Consequently, there exists a polynomial $k(x)$ such that $h(x) = (x -
    b)k(x)$. Substituting into the above expression gives
    \[ g(x) = (x - a)(x - b)k(x) \Longleftrightarrow f(x) = (x^2 - (a + b)x +
    ab)k(x) + r.\]
    Hence $f(x)$ also leaves the remainder $r$ when divided by $x^2 - (a + b)x
    + ab$.
\end{solution}

\begin{question}
    Three circles are tangent to each other as shown. The two smaller circles
    are tangent to chord $AB$ which has length 12 at its midpoint. What is the
    area of the shaded region?
\end{question}
\begin{solution}
    Let the outer circle be $\omega$, and the inner two circles be $\omega_1$
    and $\omega_2$. Let $C$, $D$ and $E$ be points where $(\omega, \omega_1)$,
    $(\omega, \omega_2)$ and $(\omega_1, \omega_2)$ touch each other.
    \begin{center}
        \begin{asy}
            import olympiad;
            size(6cm);
            defaultpen(fontsize(11pt));
            pen mydash = linetype(new real[] {5,5});
            real s = 20;
            pair A = dir(180-s);
            pair B = dir(s);
            pair C = dir(90);
            pair D = dir(270);
            pair E = midpoint(A--B);
            pair O1 = midpoint(C--E);
            pair O2 = midpoint(D--E);
            filldraw(circle((0, 0), 1), .6*gray + .4*white);
            filldraw(circle(O1, abs(O1-E)), white);
            filldraw(circle(O2, abs(O2-E)), white);
            draw(A--B);
            dot("$A$", A, dir(A));
            dot("$B$", B, dir(B));
            dot("$C$", C, dir(C));
            dot("$D$", D, dir(D));
            dot("$E$", E, dir(90));
        \end{asy}
    \end{center}
    Then the area of the shaded region is
    \begin{align*}
        [\omega] - [\omega_1] - [\omega_2] &= \frac{\pi(CD^2 - CE^2 - DE^2)}{4}\\
        &= \frac{\pi((CE + DE)^2 - CE^2 - DE^2)}{4}\\
        &= \frac{2\pi \cdot CE \cdot DE}{4}\\
        &= \frac{\pi \cdot AE \cdot BE}{2}\\
        &= \frac{36 \pi}{2}\\
        &= 18 \pi,
    \end{align*}
    where we used the fact that $AE \cdot BE = CE \cdot DE$ since $A$, $B$, $C$,
    $D$ are concyclic.
\end{solution}

\begin{question}
    In a sequence, $u_{1} = 1$, $u_{2} = 2$ and $u_{3} = 3$. For $n \geq 4$,
    the $n$th term $u_{n}$ is calculated from the previous three terms as
    $u_{n} = u_{n - 3} + u_{n - 2} - u_{n - 1}$. For example, $u _ {4} = u_{1}
    + u_{2} - u_{3} = 0$. By using mathematical induction, prove that $u_{2n +
    1} = 2n + 1$ for all integers $n \geq 0$.
\end{question}
\begin{solution}
    This problem has the same solution as \hyperref[sol: 2015 Regional Round
    P7]{2015 Regional Round P7}. 
\end{solution}

\begin{question}
    In the diagram, $AB = BC = 1$ and $ABC$ is the diameter of the larger
    semicircle. $AB$ and $BC$ are diameters of the smaller semicircles. What is
    the diameter of the circle tangent to all three semicircles?
\end{question}
\begin{solution}
    Let $M$ be the midpoint of $AB$. Let the center of the small circle be $O$
    and let it touch the outer semicircle at $X$. Finally let its radius be
    $r$. 
    \begin{center}
        \begin{asy}
            import olympiad;
            size(7cm);
            defaultpen(fontsize(11pt));
            pen mydash = linetype(new real[] {5,5});
            pair A = dir(180);
            pair B = (0, 0);
            pair C = dir(0);
            pair O = (0, 2/3);
            pair X = dir(90);
            pair M = midpoint(A--B);
            pair N = midpoint(B--C);
            draw(O--M);
            draw(B--O);
            draw(O--X, mydash);
            draw(arc(B, 1, 0, 180)--A--C--cycle);
            draw(arc(M, 1/2, 0, 180));
            draw(arc(N, 1/2, 0, 180));
            draw(circle(O, 1/3));
            dot("$A$", A, dir(225));
            dot("$B$", B, dir(270));
            dot("$C$", C, dir(315));
            dot("$X$", X, dir(90));
            dot("$M$", M, dir(270));
            dot("$O$", O, dir(0));
        \end{asy}
    \end{center}
    Then by the Pythagoras's theorem,
    \[ r + \frac{1}{2} = OM = \sqrt{BO^2 + BM^2} = \sqrt{(1 - r)^2 +
    \frac{1}{4}}. \]
    Squaring both sides and solving for $r$ gives $r = \frac{1}{3}$, so the
    diameter of the small circle is $\frac{2}{3}$. 
\end{solution}

\begin{question}
    A six-digit number is of the form $abcabc$, where all the digits are
    nonzero. Find three different prime factors of that number.
\end{question}
\begin{solution}
    Let $N = \overline{abcabc}$. Then 
    \[ N = 100000a + 10000b + 1000c + 100a + 10b + c = 1001(100a + 10b + c). \]
    Since 7, 11 and 13 all divide 1001, they must also be prime factors of
    $N$.
\end{solution}

\begin{question}
    Real numbers $a$ and $b$ are such that $a > b > 0$, $a \ne 1$ and 
    \[a^{2016} + b^{2016} = a^{2014} + b^{2014}.\] 
    Prove that $a^2 + b^2 < 2$.
\end{question}
\begin{solution}
    Rearranging the given equation gives
    \[ a^{2014}(a^2 - 1) = b^{2014}(1 - b^2).\]
    Since $a \ne 1$,
    \[ \frac{1 - b^2}{a^2 - 1} = \left( \frac{a}{b} \right)^{2014} > 1
    \Longrightarrow 1 - b^2 > a^2 - 1 \Longrightarrow a^2 + b^2 < 2. \qedhere \]
\end{solution}

\begin{question}
    In $\triangle ABC$, $AC = \frac{1}{2}(AB + BC)$ and $BN$ is the bisector of
    $\angle ABC$. $K$ and $M$ are the midpoints of $AB$ and $BC$ respectively.
    If $\angle ABC = \beta$, prove that $\angle KNM = 90^\circ -
    \frac{1}{2}\beta$.
\end{question}
\begin{solution}
    Let $I$ be the incenter of $\triangle ABC$. By the angle bisector theorem,
    \[ \frac{AB}{BC} = \frac{AN}{NC} \Longrightarrow \frac{AB}{AN} =
    \frac{CB}{CN} = \frac{AB + CB}{AN + CN} = \frac{AB + BC}{AC} = 2. \]
    \begin{center}
        \begin{asy}
            import olympiad;
            size(7cm);
            defaultpen(fontsize(11pt));
            pen mydash = linetype(new real[] {5,5});
            usepackage("contour", "outline");
            texpreamble("\contourlength{1pt}");
            pair A = (-1, 0);
            pair C = (1, 0);
            real s = .4;
            pair N = (1-s)*A + s*C;
            pair B[] = intersectionpoints(circle(A, 4*s), circle(C, 4-4*s));
            pair B = B[0];
            pair K = midpoint(A--B);
            pair M = midpoint(C--B);
            pair I = incenter(A, B, C);
            draw(B--N);
            draw(N--K);
            draw(N--M);
            draw(A--C);
            draw(A--I, mydash);
            draw(C--I, mydash);
            draw(A--B--C--cycle, black+1);
            add(pathticks(B--K, 1, .5, 6, 3));
            add(pathticks(K--A, 1, .5, 6, 3));
            add(pathticks(A--N, 1, .5, 6, 3));
            add(pathticks(N--C, 2, .5, 1, 3));
            add(pathticks(C--M, 2, .5, 1, 3));
            add(pathticks(M--B, 2, .5, 1, 3));
            dot("$A$", A, dir(225));
            dot("$C$", C, dir(315));
            dot("$N$", N, dir(270));
            dot("$K$", K, dir(180));
            dot("$M$", M, dir(0));
            dot("$B$", B, dir(B));
            dot("\contour{white}{$I$}", I, dir(270));
        \end{asy}
    \end{center}
    Thus $AN = \frac{AB}{2} = AK$ and similarly $CM = CN$. Since $AI$ bisects
    $\angle A$, this implies that $AI \perp KN$ and similarly $CI \perp MN$.
    Therefore,
    \[ \angle KNM = 180^\circ - \angle ANK - \angle MNC = 180^\circ - \left(
    90^\circ - \frac{\alpha}{2} \right) - \left( 90^\circ - \frac{\gamma}{2}
    \right) = \frac{\alpha}{2} + \frac{\gamma}{2} = 90^\circ - \frac{\beta}{2}, \]
    where $\alpha = \angle A$ and $\gamma = \angle C$.
\end{solution}

\begin{question}
    If the $(m - n)$th and $(m + n)$th terms of a geometric progression are the
    arithmetic mean and harmonic mean of $x > 0$ and $y > 0$, prove that the
    $m$th term is their geometric mean.
\end{question}
\begin{solution}
    Let the common ratio of the geometric progression be $r$. It is easy to see
    that $u_m$ is obtained by multiplying $u_{m - n}$ with $r$ for $n$ times,
    and similarly for $u_{m + n}$ and $u_{m}$. Hence,
    \[ \frac{u_{m}}{u_{m - n}} = r^n = \frac{u_{m + n}}{u_{m}}. \]
    Therefore,
    \[ u_m = \sqrt{u_{m - n} \cdot u_{m + n}} = \sqrt{\left( \frac{x + y}{2}
    \right) \left( \frac{2xy}{x + y} \right)} = \sqrt{xy} \]
    so $u_m$ is the geometric mean of $x$ and $y$ as desired.
\end{solution}

\begin{question}
    Show that the sum of the squares of the lengths of the sides of a
    parallelogram equals the sum of the squares of the lengths of the
    diagonals.
\end{question}
\begin{solution}
    Let $\angle ABC = \theta$, and let $AB = CD = x$ and $BC = AD = y$. 
    \begin{center}
        \begin{asy}
            import olympiad;
            size(6cm);
            defaultpen(fontsize(11pt));
            pen mydash = linetype(new real[] {5,5});
            pair A = (-1, 0);
            pair D = (1, 0);
            pair B = (-1.3, -1);
            pair C = B + D - A;
            draw(A--B--C--D--cycle, black+1);
            draw(A--C);
            draw(B--D);
            dot("$A$", A, dir(135));
            dot("$B$", B, dir(225));
            dot("$C$", C, dir(315));
            dot("$D$", D, dir(45));
        \end{asy}
    \end{center}
    Since $DA \parallel BC$, $\angle BAD = 180^\circ - \theta$. Therefore, by
    law of cosines in $\triangle ABC$ and $\triangle ABD$, we have
    \[ AC^2 = x^2 + y^2 - 2xy \cos \theta, \]
    and 
    \[ BD^2 = x^2 + y^2 - 2xy \cos (180^\circ - \theta). \]
    Adding them gives
    \begin{align*}
        AC^2 + BD^2 &= 2x^2 + 2y^2 - 2xy(\cos \theta + \cos (180^\circ - \theta))\\
        &= 2x^2 + 2y^2 - 2xy(\cos \theta - \cos \theta)\\
        &= 2x^2 + 2y^2\\
        &= AB^2 + BC^2 + CD^2 + DA^2. \qedhere
    \end{align*}
\end{solution}

\begin{question}
    If $n$ is a positive even integer, prove by mathematical induction that
    $x^n - y^n$ is divisible by $x + y$.
\end{question}
\begin{solution}
    The base case $n = 2$ is true since $x^2 - y^2 = (x + y)(x - y)$. Now
    suppose that $x^{2k} - y^{2k}$ is divisible by $x - y$. We must show that
    $x^{2k + 2} - y^{2k + 2}$ is also divisible by $x - y$. Observe that
    \[ x^{2k + 2} - y^{2k + 2} = x^{2k + 2} - x^2 y^{2k} + x^2 y^{2k} - y^{2k +
    2} = x^2(x^{2k} - y^{2k}) - y^{2k}(x^2 - y^2). \]
    But both $x^{2k} - y^{2k}$ and $x^2 - y^2$ are divisible by $x + y$, so it
    follows that $x^{2k + 2} - y^{2k + 2}$ is also divisible by $x + y$. Hence
    by mathematical induction, $x^n - y^n$ is divisible by $x + y$ for all even $n$.
\end{solution}

\begin{question}
    Prime numbers $p$, $q$ and positive integers $m$, $n$ satisfy the following
    conditions:
    \[m < p, n < q \text{  and  }\frac{p}{m}+\frac{q}{n} \text{  is an integer.}\]
    Prove that $m = n$.
\end{question}
\begin{solution} 
    Since $m < p$, it follows that $m$ and $p$ are relatively prime. Similarly,
    $n$ and $p$ are also relatively prime. Now since $p/m + q/n$ is an integer,
    \[ \frac{pn}{m} = n\left(\frac{p}{m} + \frac{q}{n}\right) - q \]
    is also an integer. Since $\gcd(m, p) = 1$, by \hyperref[lem:
    euclid]{Euclid's lemma}, it follows that $m$ divides $n$, and since they're
    positive, $m \leq n$. Similarly, $n \leq m$, so it must be the case that $m
    = n$.
\end{solution}

\begin{question}
    $A$, $B$ and $C$ are three points on the circumference of a circle, and the
    tangent at $A$ meets $BC$ produced at $T$. Prove that $AB^2 : AC^2 = TB :
    TC$.
\end{question}
\begin{solution}
    Notice that $\angle TAB = \angle TCA$, so $\triangle TAB \sim \triangle TCA$. 
    \begin{center}
        \begin{asy}
            import olympiad;
            size(7cm);
            defaultpen(fontsize(11pt));
            pen mydash = linetype(new real[] {5,5});
            pair A = dir(140);
            pair B = dir(210);
            pair C = dir(330);
            pair O = circumcenter(A, B, C);
            pair T = extension(A, rotate(90, A)*O, B, C);
            draw(A--B--C--cycle, black+1);
            draw(T--A);
            draw(T--B);
            draw(anglemark(T, A, B));
            draw(anglemark(A, C, B));
            draw(circumcircle(A, B, C));
            dot("$A$", A, dir(A));
            dot("$B$", B, dir(225));
            dot("$C$", C, dir(315));
            dot("$T$", T, dir(225));
        \end{asy}
    \end{center}Hence
    \[ \frac{TB}{AB} \cdot \frac{AC}{TC} = \frac{TA}{AC} \cdot \frac{AB}{TA} =
    \frac{AB}{AC} \Longleftrightarrow \frac{TB}{TC} = \frac{AB^2}{AC^2}.
    \qedhere \]
\end{solution}

\begin{question}
    $ABCD$ is a cyclic quadrilateral. $AE$ is drawn to meet $BD$ at $E$ such
    that $\angle BAE = \angle CAD$. Prove that 
    \begin{enumerate}
        \item $\triangle ABE \sim \triangle ACD$, 
        
        \item $\triangle AED \sim \triangle ABC$,
        
        \item $AB \cdot CD + AD \cdot BC = AC \cdot BD$.
    \end{enumerate}
\end{question}
\begin{solution} 
    This is known as \hyperref[thm: ptolemy]{Ptolemy's theorem}, or more
    generally, Ptolemy's inequality.
    \begin{center}
        \begin{asy}
            import olympiad;
            size(6cm);
            defaultpen(fontsize(11pt));
            pen mydash = linetype(new real[] {5,5});
            pair A = dir(110);
            pair B = dir(210);
            pair C = dir(330);
            pair D = dir(50);
            pair M = midpoint(A--B);
            pair O = extension(B, rotate(90, B)*C, M, rotate(90, M)*B);
            pair E = 2*foot(O, B, D) - B;
            draw(A--B--C--D--cycle, black+1);
            draw(A--C);
            draw(B--D);
            draw(A--E);
            draw(anglemark(B, A, E, 5));
            draw(anglemark(C, A, D, 5));
            draw(circle((0, 0), 1));
            dot("$A$", A, dir(A));
            dot("$B$", B, dir(B));
            dot("$C$", C, dir(C));
            dot("$D$", D, dir(D));
            dot("$E$", E, dir(315));
        \end{asy}
    \end{center}
    \begin{enumerate}
        \item Since $ABCD$ is cyclic, $\angle ABE = \angle ACD$. Combined with
            the fact that $\angle BAE = \angle CAD$, this lets us deduce that
            $\triangle ABE \sim \triangle ACD$. Hence
            \[ \frac{AB}{BE} = \frac{AC}{CD} \Longleftrightarrow AB \cdot CD =
            AC \cdot BE. \]
        
        \item Observe that 
            \[ \angle BAC = \angle BAE + \angle EAC = \angle CAD + \angle EAC =
            \angle EAD, \]
            and $\angle ADE = \angle ACB$. Therefore, $\triangle AED \sim
            \triangle ABC$. Similarly to above, $AD \cdot BC = AC \cdot ED$.

        \item From the similarities,
            \[ AC \cdot BD = AC (BE + ED) = AC \cdot BE + AC \cdot ED = AB
            \cdot CD + AD \cdot BC. \qedhere \]
    \end{enumerate}
\end{solution}
