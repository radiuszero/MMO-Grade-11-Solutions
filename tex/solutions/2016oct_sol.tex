\solutionheader{2016 Regional Round}
\begin{question}
    In the sequence $u_{1}$, $u_{2}$, $\ldots$, $u_{n}$, $\ldots$, the $n$th
    term is defined by $u_{n} = 1 - \frac{1}{u_{n - 1}}$ for $n \geq 2$. If
    $u_{1} = 3$, compute $u_{2}$, $u_{3}$ and $u_{4}$. Write down $u_{2016}$.
\end{question}
\begin{solution} 
    If $u_1 = 3$,
    \begin{align*}
        u_2 &= 1 - \frac{1}{3} = \frac{2}{3}\\
        u_3 &= 1 - \frac{3}{2} = -\frac{1}{2}\\
        u_4 &= 1 - (-2) = 3.
    \end{align*}
    Here we see that $u_1 = u_4$. Moreover, since each term depends entirely
    upon the previous one, it must be the case that if $u_a = u_b$, $u_{a + 1}
    = u_{b + 1}$ as well. Therefore, $u_5 = u_2 = \frac{2}{3}$, $u_6 = u_3 =
    -\frac{1}{2}$, and so on. In general, the value of $u_n$ is 
    \[ u_n =
    \begin{cases}
        3 & \text{if $n = 3k + 1$,}\\
        \frac{2}{3} & \text{if $n = 3k + 2$,}\\
        -\frac{1}{2} & \text{if $n = 3k$.}
    \end{cases}
    \]
    for any positive integer $k$. In particular, since 2016 is divisible by 3,
    $u_{2016} = -\frac{1}{2}$.
\end{solution}

\begin{question}
    If $x$ and $y$ are positive integers such that $56 \leq x + y \leq 59$ and
    $0.9 \leq \frac{x}{y} \leq 0.91$, find the value of $y^2 - x^2$.
\end{question}
\begin{solution}
    This is pretty straightforward inequality manipulation. From the problem,
    we have the following two relations. 
    \[ 56 \leq x + y \leq 59 \quad \text{and} \quad 0.9y \leq x \leq 0.91y. \]
    This allows us to bound $y$.
    \[ 56 \leq x + y \leq 0.91y + y = 1.91y \Longrightarrow y \geq 29.31, \]
    and 
    \[ 59 \geq x + y \geq 0.9y + y = 1.9y \Longrightarrow y \leq 31.06. \]
    Since $y$ is a positive integer, the only possible values are 30 and 31. So
    we will consider two cases.
    \begin{case}{$y = 30$.}
        In this case, $27 \leq x \leq 27.3$, so it follows that $x = 27$. This
        satifies the first relation, and thus $(x, y) = (27, 30)$ is a solution
        pair.
    \end{case}
    \begin{case}{$y = 31$.}
        In this case, $27.9 \leq x \leq 28.21$ and so $x = 28$. This also
        satisifies the first relation, and so $(x, y) = (28, 31)$ is also a
        solution pair.
    \end{case}
    Finally, the possible values of $y^2 - x^2$ are 171 and 177.
\end{solution}

\begin{question}
    When 15 is added to a number $x$, it becomes the square of an integer. When
    74 is subtracted from $x$, the result is a square of another integer. Find
    the number $x$. 
\end{question}
\begin{solution}
    From the problem statement, there exist positive integers $y$ and $z$ such
    that
    \[ x + 15 = y^2 \quad \text{and} \quad x - 74 = z^2. \]
    Subtracting the two equations gives
    \[ y^2 - z^2 = (y + z)(y - z) = 89. \]
    But since 89 is a prime, the smaller one of the divisors on the left, i.e.
    $y - z$, must be equal to 1. Therefore, $y - z = 1$ and $y + z = 89$.
    Solving those two equations give $y = 45$ and $z = 44$. Substituting these
    into the first equation shows that $x = 2010$.
\end{solution}

\begin{question}
    In a rectangle $ABCD$, the points $M$, $N$, $P$, $Q$ lie on $AB$, $BC$,
    $CD$ and $DA$ respectively, such that the areas of $\triangle AQM$,
    $\triangle BMN$, $\triangle CNP$ and $\triangle DPQ$ are equal. Prove that
    $MNPQ$ is a parallelogram.
\end{question}
\begin{solution}
    Let $[\triangle AQM] = [\triangle BMN] = [\triangle CNP] = [\triangle DPQ]
    = A$. Let $AQ = a_1$, $BN = b_1$, $CN = c_1$, $DQ = d_1$, and $AM = a_2$,
    $BM = b_2$, $CP = c_2$, $DP = d_2$.
    \begin{center}
        \begin{asy}
            import olympiad;
            size(7cm);
            defaultpen(fontsize(11pt));
            pen mydash = linetype(new real[] {5,5});
            real p = .3;
            real q = .6;
            pair A = (1, 2);
            pair B = (4, 2);
            pair C = (3, 0);
            pair D = A + C - B;
            pair M = (1-p)*A + p*B;
            pair Q = (1-q)*A + p*D;
            pair N = B + D - Q;
            pair P = A + C - M;
            draw(A--B--C--D--cycle, black+1);
            draw(M--N--P--Q--cycle);
            dot("$A$", A, dir(135));
            dot("$B$", B, dir(45));
            dot("$C$", C, dir(315));
            dot("$D$", D, dir(225));
            dot("$M$", M, dir(90));
            dot("$N$", N, dir(0));
            dot("$P$", P, dir(270));
            dot("$Q$", Q, dir(180));
            label("$a_1$", midpoint(A--Q), dir(180));
            label("$a_2$", midpoint(A--M), dir(90));
            label("$b_1$", midpoint(B--N), dir(0));
            label("$b_2$", midpoint(B--M), dir(90));
            label("$c_1$", midpoint(C--N), dir(0));
            label("$c_2$", midpoint(C--P), dir(270));
            label("$d_1$", midpoint(D--Q), dir(180));
            label("$d_2$", midpoint(D--P), dir(270));
        \end{asy}
    \end{center}
    Then we can rewrite the area equality as
    \[ a_1a_2 = b_1b_2 = c_1c_2 = d_1d_2 = 2A. \]
    Since the opposide sides of a rectangle are equal, we also have
    \[ a_1 + d_1 = b_1 + c_1 \quad \text{and} \quad a_2 + b_2 = c_2 + d_2. \]
    We can transform the second equation as follows:
    \begin{align*}
        \frac{2A}{a_1} + \frac{2A}{b_1} &= \frac{2A}{c_1} + \frac{2A}{d_1}\\
        \frac{1}{a_1} + \frac{1}{b_1} &= \frac{1}{c_1} + \frac{1}{d_1}\\
        \frac{1}{a_1} - \frac{1}{c_1} &= \frac{1}{d_1} - \frac{1}{b_1}\\
        \frac{c_1 - a_1}{a_1c_1} &= \frac{b_1 - d_1}{b_1d_1}
    \intertext{From the first equation, we have $a_1 - c_1 = b_1 - d_1$. Therefore,}
        \frac{c_1 - a_1}{a_1c_1} &= -\frac{c_1 - a_1}{b_1d_1}\\
        (c_1 - a_1)\left( \frac{1}{a_1c_1} + \frac{1}{b_1d_1} \right) &= 0.
    \end{align*}
    Since the latter bracket is positive, it follows that $a_1 = c_1$.
    Similarly, we also have $b_1 = d_1$, $a_2 = c_2$ and $b_2 = d_2$.
    Therefore, $\triangle AQM \cong \triangle CNP$ and $\triangle BNM \cong
    \triangle DQP$. Thus $QM = NP$ and $NM = QP$ implying that $MNPQ$ is a
    parallelogram as desired.
\end{solution}

\begin{question}
    Draw 6 circles in the plane such that every circle passes through exactly 3
    centres of other circles. 
\end{question}
\begin{solution}
    Consider two unit equilateral triangles, separated unit distance apart.
    Next, we draw 6 circles with unit radius centered at each of the vertices.
    It is obvious that each circle passes through exactly three other vertices.
    % Insert figure
\end{solution}

\begin{question}
    There are 2016 students in a secondary school. Every student writes a new
    year card. The cards are mixed up and randomly distributed to students.
    Suppose each student gets one and only one card. Find the expected number
    of students who get back their own cards. 
\end{question}
\begin{solution}
    Label each of the students and their corresponding cards from 1 to 2016,
    and call a student \emph{fixed point} if they receive their own card. By
    the definition of expected value,
    \begin{align*}
        \Ex[\text{number of fixed points}] &= \sum_{i = 1}^{2016} i \cdot \Pr(\text{number of fixed points is $i$})\\
        &= \sum_{i = 1}^{2016} \frac{i \cdot \text{number of distributions with $i$ fixed points}}{\text{total number of distributions}}\\
        &= \frac{\text{total number of fixed points over all distributions}}{\text{total number of distributions}}.
    \end{align*}
    Consider a table, consisting of a row for each distribution of cards (An
    example for 3 students is provided below.) The total number of fixed points
    in this whole table can be found by finding the number of fixed points in
    each column, and then summing them. For each student $i$, there are 2015!
    ways to permute the other 2015 students. Therefore, the number of
    distributions where student $i$ is a fixed point is $2015!$. Since there
    are 2016 columns, the total number of fixed points in the whole table is
    $2016!$. 
    \begin{center}
        \begin{tabular}{ |c|c c c| }
            \hline
            \text{Students} & \textbf{1} & \textbf{2} & \textbf{3}\\
            \hline
            & \HL{1} & \HL{2} & \HL{3} \\
            & \HL{1} & 3 & 2 \\
            & 2 & 1 & \HL{3} \\
            & 2 & 3 & 1 \\
            & 3 & 1 & 2 \\
            & 3 & \HL{2} & 1 \\
            \hline
            \text{Number of fixed points} & 2 & 2 & 2 \\
            \hline
        \end{tabular} \\
        \vspace*{\baselineskip}
    \end{center}
    Finally, since the total number of distributions is also $2016!$, it
    follows that the expected value is 1.
\end{solution}

\begin{question}
    The points $P$, $A$, $B$ lie in that order on a circle with center $O$ such
    that $\angle POB < 180^\circ$. The point $Q$ lies inside the circle such
    that $\angle PAQ = 90^\circ$ and $PQ = BQ$. If $\angle AQB > \angle AQP$,
    prove that $\angle AQB - \angle AQP = \angle AOB$. 
\end{question}
\begin{solution}
    Let $M$ be the midpoint of $BP$.
    \begin{center}
        \begin{asy}
            import olympiad;
            size(7cm);
            defaultpen(fontsize(11pt));
            pen mydash = linetype(new real[] {5,5});
            pair P = dir(140);
            pair A = dir(90);
            pair B = dir(0);
            pair O = (0, 0);
            pair M = midpoint(B--P);
            pair Q = extension(A, rotate(90, A)*P, O, M);
            draw(circle(O, 1));
            draw(P--A--Q);
            draw(P--B);
            draw(O--Q, mydash);
            draw(P--Q--B);
            draw(A--O--B);
            draw(rightanglemark(P, A, Q, 2.5));
            draw(rightanglemark(Q, M, P, 2.5));
            dot("$P$", P, dir(P));
            dot("$A$", A, dir(A));
            dot("$B$", B, dir(B));
            dot("$Q$", Q, dir(45));
            dot("$M$", M, 1.5*dir(285));
            dot("$O$", O, dir(270));
        \end{asy}
    \end{center}
    Since $\triangle PQB$ is an isosceles triangle, $\angle QMP = 90^\circ =
    \angle QAP$ so $QAPM$ is cyclic. Moreover, $\angle PQM = \angle MQB$.
    Therefore,
    \[ \angle AQB - \angle AQP = 360^\circ - 2\angle PQM - 2\angle AQP =
    2(180^\circ - \angle AQM) = 2\angle APB = \angle AOB. \qedhere \]
\end{solution}
\begin{remark}
    There are configuration issues depending on $Q$ being inside or outside of
    $\triangle OPB$, but the proof is more or less the same so we will not
    mention them here. 
\end{remark}

\begin{question}
    A fair die is thrown three times. The results of the first, second and
    third throw are recorded as $x$, $y$ and $z$ respectively. Suppose that $x
    + y = z$. What is the probability that at least one of $x$, $y$ and $z$ is
    2. 
\end{question}
\begin{solution}
    Since the value of $z$ is automatically determined by those of $x$ and $y$,
    we only need to consider the possible outcomes of $x$ and $y$. The
    following table summarizes the possible triples of $(x,y,z)$ satisfying
    $x+y=z$. In addition, those that doesn't satisfy $x,y,z\leq 6$ are greyed
    out.
    \begin{center}
        \begin{tabular}{ |c|c|c|c|c|c|c| }
            \hline
            \diagbox[height=0.45cm]{\raisebox{-2pt}{$x$}}{\raisebox{4pt}{$y$}} & 1 & 2 & 3 & 4 & 5 & 6 \\
            \hline
            1 & (1, 1, 2) & (1, 2, 3) & (1, 3, 4) & (1, 4, 5) & (1, 5, 6) & \textcolor{mygray}{(1, 6, 7)} \\
            \hline
            2 & (2, 1, 3) & (2, 2, 4) & (2, 3, 5) & (2, 4, 6) & \textcolor{mygray}{(2, 5, 7)} & \textcolor{mygray}{(2, 6, 8)} \\
            \hline
            3 & (3, 1, 4) & (3, 2, 5) & (3, 3, 6) & \textcolor{mygray}{(3, 4, 7)} & \textcolor{mygray}{(3, 5, 8)} & \textcolor{mygray}{(3, 6, 9)} \\
            \hline
            4 & (4, 1 ,5) & (4, 2, 6) & \textcolor{mygray}{(4, 3, 7)} & \textcolor{mygray}{(4, 4, 8)} & \textcolor{mygray}{(4, 5, 9)} & \textcolor{mygray}{(4, 6, 10)} \\
            \hline 
            5 & (5, 1, 6) & \textcolor{mygray}{(5, 2, 7)} & \textcolor{mygray}{(5, 3, 8)} & \textcolor{mygray}{(5, 4, 9)} & \textcolor{mygray}{(5, 5, 10)} & \textcolor{mygray}{(5, 6, 11)} \\
            \hline
            6 & \textcolor{mygray}{(6, 1, 7)} & \textcolor{mygray}{(6, 2, 8)} & \textcolor{mygray}{(6, 3, 9)} & \textcolor{mygray}{(6, 4, 10)} & \textcolor{mygray}{(6, 5, 11)} & \textcolor{mygray}{(6, 6, 12)} \\
            \hline
        \end{tabular}
    \end{center}
    It is easy to see that the number of desired triples is 8 and the total is
    15. Thus the probability is $\frac{8}{15}$.
\end{solution}
\begin{remark}
    This problem asks for the probability that at least one of $x$, $y$, $z$ is
    2, \emph{given that} $x + y = z$. This is not the same with the probability
    that $x + y = z$ \emph{and} at least one of $x$, $y$, $z$ is 2, which would
    be $\frac{8}{6^3}$.
\end{remark}

\begin{question}
    An arithmetic progression and a harmonic progression have $a$ and $b$ for
    the first two terms. If their $n$th terms are $x$ and $y$ respectively,
    show that $(x - a) : (y - a) = b : y$. 
\end{question}
\begin{solution}
    Since $a$, $b$ are the first two terms of a HP, $\frac{1}{a}$,
    $\frac{1}{b}$ are the first two terms of an AP. The first term of this AP
    is $\frac{1}{a}$, and the common difference is $\frac{1}{b} - \frac{1}{a}$,
    so the $n$th term of this AP is
    \[ u_n = \frac{1}{a} + (n - 1)\left( \frac{1}{b} - \frac{1}{a} \right) =
    \frac{b + (n - 1)(a - b)}{ab}, \]
    which means that the $n$th term of the HP is 
    \[ y = \frac{ab}{b + (n - 1)(a - b)}. \]
    Meanwhile, the $n$th term of the AP is $x = a + (n - 1)(b - a)$. Therefore, 
    \begin{align*}
        \frac{x - a}{y - a} &= (n - 1)(b - a) \div \left( \frac{ab}{b + (n - 1)(a - b)} - a \right)\\
        &= (n - 1)(b - a) \cdot \frac{b + (n - 1)(a - b)}{a(n - 1)(b - a)}\\
        &= \frac{b + (n - 1)(a - b)}{a}\\
        &= \frac{b}{y}. \qedhere
    \end{align*}
\end{solution}

\begin{question}
    Mr. Game owns 200 custom-made dice. Each die has four sides showing the
    number 2 and two sides showing the number 5. Mr. Game is about to throw all
    200 dice together and find out the sum of all 200 results. How many
    possible values of this sum are there?
\end{question}
\begin{solution}
    Suppose that in a throw, there are $n$ dice that show 2, and $200 - n$ dice
    that show 5. Then the sum of their values is $2n + (200 - n)5 = 1000 - 3n$.
    Obviously, for different values of $n$, the sum is also different. Since
    $n$ can range from 0 to 200, this means that the sum can also have 201
    different values. 
\end{solution}

\begin{question}
    U Tet Toe wants to repair all 4 walls of his room. He has red paint, yellow
    paint and blue paint (which he cannot mix), and wants to paint his room so
    that adjacent walls are never of the same colour. In how many ways can U
    Tet Toe paint his room?
\end{question}
\begin{solution}
    We just need to find the \hyperref[def: chromaticpoly]{chromatic
    polynomial} of $C_4$. By \hyperref[teq: DC]{deletion-contraction},
    \[ P(C_4, 3) = P(P_4, 3) - P(C_3, 3). \]
    We will first find $P(P_4, 3)$. There are 3 choices for the first vertex
    and 2 choices each for the remaining vertices. Therefore, $P(P_3, 3) = 3
    \cdot 2 \cdot 2 \cdot 2 = 24$. Next, let's find the value of $P(C_3, 3)$.
    There are 3 choices for the first vertex, 2 choices for the second vertex,
    and only 1 choice for the final vertex. Therefore, $P(C_3, 3) = 3 \cdot 2
    \cdot 1 = 6$. Hence there are $24 - 6 = 18$ ways in which U Tet Toe can
    paint his room. 
\end{solution}

\begin{question}
    Given an equilateral triangle, what is the ratio of the area of its
    circumscribed circle to the area of the inscribed circle?
\end{question}
\begin{solution}
    Let $O$ be the circumcenter of the equilateral triangle $ABC$ and let the
    incircle touch side $BC$ at $D$. 
    \begin{center}
        \begin{asy}
            import olympiad;
            size(6cm);
            defaultpen(fontsize(11pt));
            pen mydash = linetype(new real[] {5,5});
            pair A = dir(90);
            pair B = dir(210);
            pair C = dir(330);
            pair O = (0, 0);
            pair D = foot(O, B, C);
            draw(A--B--C--cycle, black+1);
            draw(B--O--D);
            draw(incircle(A, B, C));
            draw(circle(O, 1));
            draw(rightanglemark(O, D, B, 2.5));
            dot("$A$", A, dir(A));
            dot("$B$", B, dir(B));
            dot("$C$", C, dir(C));
            dot("$O$", O, dir(0));
            dot("$D$", D, dir(270));
        \end{asy}
    \end{center}
    Due to the symmetry, $O$ is also the incenter of $\triangle ABC$. Since
    $\angle OBD = 30^\circ$, $\triangle OBD$ is a 30-60 right triangle, and $OB
    : OD = 2 : 1$. Therefore, if we let the area of the circumcircle by $A_1$
    and that of the incircle by $A_2$,
    \[ \frac{A_1}{A_2} = \frac{\pi OB^2}{\pi OD^2} = 4. \qedhere \]
\end{solution}

\begin{question}
    Solve the equation $\sqrt{3x^2 - 8x + 1} + \sqrt{9x^2 - 24x - 8} = 3$.
\end{question}
\begin{solution}
    Let $u = 3x^2 - 8x + 1$. Then we can rewrite the equation as
    \begin{align*}
        \sqrt{u} + \sqrt{3u - 11} &= 3 \\
    \intertext{Squaring both sides to remove the square roots gives}
        u + 3u - 11 + 2u\sqrt{3u - 11} &= 9 \\
        2u - 10 &= -\sqrt{u(3u - 11)}\\
    \intertext{Squaring again, we have}
        4u^2 - 4u + 100 &= 3u^2 - 11u\\
        u^2 - 29u + 100 &= 0\\
        (u - 25)(u - 4) &= 0\\
    \end{align*}
    which shows that $u = 4$ or $u = 25$. It is easy to check that $u = 25$
    does not satisfy the equation, so $u = 4$ must be the only solution.
    Finally,
    \[ 3x^2 - 8x + 1 = 4 \Longrightarrow (x - 3)(3x + 1) = 0. \]
    and hence $x = 3$ or $-\frac{1}{3}$.
\end{solution}

\begin{question}
    Prove by mathematical induction that 
    \[1^2 + 3^2 + 5^2 + 7^2 + \cdots + (2n - 1)^2 = \frac{1}{3}n(4n^2 - 1).\] 
\end{question}
\begin{solution}
    We will show this by induction. For the base case, when $n = 1$, both the
    left hand side and the right hand side are equal to 1, so the identity is
    true. Now suppose that for $n = k$,
    \[ 1^2 + 3^2 + \cdots + (2k - 1)^2 = \frac{k(4k^2 - 1)}{3}. \]
    Then when $n = k + 1$,
    \begin{align*}
        1^2 + 3^2 + \cdots + (2k + 1)^2 &= \frac{1}{3}k(4k^2 - 1) + (2k + 1)^2\\
        &= \frac{1}{3}(k(2k + 1)(2k - 1) + 3(2k + 1)^2)\\
        &= \frac{1}{3}(2k + 1)(2k^2 - k + 6k + 3)\\
        &= \frac{1}{3}(2k + 1)(2k + 3)(k + 1)\\
        &= \frac{1}{3}(k + 1)(4(k + 1)^2 - 1)
    \end{align*}
    and so the identity is true for $n = k + 1$. Hence by mathematical
    induction it follows that the identity is true for all $n \in \mathbb{N}$.
\end{solution}

\begin{question}
    In how many ways can 7 identical T-shirts be divided among 4 students,
    subject to the condition that each is to get at least 1 T-shirt. 
\end{question}
\begin{solution}
    This is a classic example of \hyperref[teq: starsandbars]{stars and bars}. 

    Label the students from 1 to 4. Suppose that we have all 7 T-shirts lying
    in a line, and we want to divide them up into 4 sections corresponding to
    each student. One way to do this would be to use 3 dividers, and insert
    them in the gaps between the shirts, so that the first section corresponds
    to student 1, the second section corresponds to student 2, and so on. Since
    each student gets at least one T-shirt, this means that there can only be
    at least one divider in each gap. The number of ways to insert 3 dividers
    in 6 gaps is $\binom{6}{3} = 20$. Therefore, there are 20 ways to
    distribute the T-shirts.
\end{solution}

\begin{question}
    $\triangle ABC$ has $AB > AC$. A line $DEF$, equally inclined to $AB$ and
    $AC$, is drawn, meeting $AB$ at $F$, $AC$ at $E$ and $BC$ produced at $D$.
    Prove that $BD : DC = BF : CE$.

    (\hint Draw $BG \parallel CE$ meeting $DF$ produced at $G$.) 
\end{question}
\begin{solution}
    Some angle chasing gives us
    \[ \angle BFG = \angle AFE = \angle AEF = \angle BGF, \]
    from which we get that $BF = BG$. Since $CE \parallel BG$, we see that
    \[ \frac{DB}{DC} = \frac{BG}{CE} = \frac{BF}{CE}. \qedhere \]
    \begin{center}
        \begin{asy}
            import olympiad;
            size(7cm);
            defaultpen(fontsize(11pt));
            pen mydash = linetype(new real[] {5,5});
            pair A = dir(50);
            pair B = dir(210);
            pair C = dir(330);
            real s = .4;
            pair F = (1-s)*A + s*B;
            pair E = A + abs(F-A)*unit(C-A);
            pair D = extension(E, F, B, C);
            pair G = extension(B, B + E - C, D, F);
            draw(A--B--C--cycle, black+1);
            draw(D--C);
            draw(D--G);
            draw(B--G);
            add(pathticks(A--F, 1, .5, 6, 4));
            add(pathticks(A--E, 1, .5, 6, 4));
            dot("$A$", A, dir(A));
            dot("$B$", B, dir(225));
            dot("$C$", C, dir(315));
            dot("$D$", D, dir(315));
            dot("$E$", E, dir(45));
            dot("$F$", F, dir(90));
            dot("$G$", G, dir(90));
        \end{asy}
    \end{center}
\end{solution}

\begin{question}
    The sum of the two smallest positive divisors of a positive integer $N$ is
    6, while the sum of the largest positive divisors of $N$ is 1122. Find $N$.
\end{question}
\begin{solution}
    Let the two smallest and largest positive divisors be $d_1$, $d_2$, $d_3$,
    $d_4$, with $d_1 < d_2 < d_3 < d_4$. Then since $d_1$ is the smallest
    divisor of $N$, it must be 1. Similarly, since $d_4$ is the larget divisor
    of $N$, it must be $N$ itself. Now the problem gives us $d_1 + d_2 = 6$ and
    $d_3 + d_4 = 1122$, which shows that 
    \[ d_2 = 5 \quad \text{and} \quad d_3 = 1122 - N. \]
    But since $d_2$ is the second smallest divisor and $d_3$ is the second
    biggest divisor, $d_2 d_3 = N$. Hence
    \[ 5(1122 - N) = N \Longrightarrow N = 935. \qedhere \]
\end{solution}

\begin{question}
    If $p$, $q$ and $r$ are prime numbers such that $p < q < r$ and their
    product is 19 times their sum, find $p(q + r)$.
\end{question}
\begin{solution}
    Let $x$, $y$ and $z$ be primes such that $xyz = 19(x + y + z)$. Then since
    $19 \mid xyz$ and $x$, $y$, $z$ are primes, one of them must be 19. WLOG,
    suppose that $x = 19$. Then \hyperref[teq: SFFT]{SFFT} gives us 
    \[ yz = 19 + y + z \Longrightarrow (y - 1)(z - 1) = 20. \]
    Remember that $y$ and $z$ are both positive integers. Theis means that $y -
    1$ and $z - 1$ are both positive divisors of 20. Hence
    \begin{center}
        \begin{tabular}{ |c|c| }
            \hline
            $y - 1$ & $z - 1$ \\
            \hline
            1 & 20 \\
            2 & 10 \\
            4 & 5 \\
            5 & 4 \\
            10 & 2 \\
            20 & 1 \\
            \hline
        \end{tabular}
    \end{center}
    Checking these cases, we see that $(y, z) = (11, 3)$ or $(3, 11)$. Hence
    $(x, y, z)$ can be any permutation of $(19, 11, 3)$. Adding the size
    condition, we see that $(p, q, r) = (3, 11, 19)$. Consequently, $p(q + r) =
    90$.
\end{solution}
