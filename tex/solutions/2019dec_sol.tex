\solutionheader{2019 National Round}
\begin{question}
    $\triangle ABC$ is an equilateral triangle. 
    \begin{enumerate}
        \item Prove that there are points $D$, $E$ and $F$ on $AB$, $BC$ and
            $CA$ respectively such that $DE \perp BC$, $EF\perp CA$ and $FD
            \perp AB$. 

        \item Prove that $\triangle DEF$ is also an equilateral triangle.

        \item Find the ratio of the perimeters of $\triangle DEF$ and
            $\triangle ABC$.
        
        \item Find the ratio of the areas of $\triangle DEF$ and $\triangle
            ABC$. 
    \end{enumerate}
\end{question}
\begin{solution}
    Let $D$, $E$ and $F$ be points on sides $AB$, $BC$ and $CA$ such that $3AD
    = AB$, $3BE = BC$ and $3CF = CA$. We will first show that $DE \perp BC$.
    Let $M$ be the midpoint of $BC$. Then since $\triangle ABC$ is equilateral,
    $AM \perp BC$. Moreover,
    \[ \frac{BE}{BM} = \frac{\frac{1}{3}BC}{\frac{1}{2}BC} = \frac{2}{3} =
    \frac{BD}{BA}. \]
    Therefore, $DE \parallel AM$ and hence $DE \perp BC$. Moreover, it is easy
    to see that $\triangle AFD \cong \triangle BDE \cong \triangle CEF$, and so
    $DE = EF = FD$ which implies that $\triangle DEF$ is equilateral. 
    \begin{center}
        \begin{asy}
            import olympiad;
            size(6cm);
            defaultpen(fontsize(11pt));
            pen mydash = linetype(new real[] {5,5});
            pair A = dir(90);
            pair B = dir(210);
            pair C = dir(330);
            pair D = (2/3)*A+(1/3)*B;
            pair E = foot(D, B, C);
            pair F = foot(E, A, C);
            pair M = midpoint(B--C);
            draw(A--B--C--cycle, black+1);
            draw(D--E--F--cycle);
            draw(A--M, mydash);
            dot("$A$", A, dir(A));
            dot("$B$", B, dir(225));
            dot("$C$", C, dir(315));
            dot("$E$", E, dir(270));
            dot("$M$", M, dir(270));
            dot("$F$", F, dir(30));
            dot("$D$", D, dir(120));
        \end{asy}
    \end{center}
    Now remember that $\triangle BAM \sim \triangle BDE$, and $\triangle BAM$
    is a 30-60 right triangle. Since $BA : BD = 3 : 2$,
    \[ \frac{BD}{DE} = \frac{BA}{AM} = \frac{2}{\sqrt{3}} \Longrightarrow
    \frac{BA}{DE} = \frac{BA}{BD} \cdot \frac{BD}{DE} = \frac{3}{2} \cdot
    \frac{2}{\sqrt{3}} = \sqrt{3}. \]
    Hence the ratio of their perimeters is also $\sqrt{3}$. Finally, since
    $\triangle ABC$ and $\triangle DEF$ are both equilateral triangles, they
    are similar. Therefore the ratio of their areas is the square of ratio of
    their sides, which is equal to $(\sqrt{3})^2 = 3$.
\end{solution}

\begin{question}
    \begin{enumerate}
        \item If $abc = 2ab + 2bc +2ca$ where $a$, $b$ and $c$ are integers and
            $1 \leq a \leq b \leq c$, then show that $3 \leq a \leq 6$.
        
        \item Find all possible ordered triples $(a,b,c)$ such that \[abc = 2ab
            + 2bc + 2ca\] where $a$, $b$ and $c$ are integers and $1 \leq a
            \leq b \leq c$. 
    \end{enumerate}
\end{question}
\begin{solution}
    We can divide the both sides of our given equation by $abc$ to get
    \[ 1 = \frac{2}{a} + \frac{2}{b} + \frac{2}{c}. \]
    Suppose that $6 < a$. Then as $a \leq b \leq c$, it follows that both $b$
    and $c$ are also greater than 6, so
    \[ 1 = \frac{2}{a} + \frac{2}{b} + \frac{2}{c} < \frac{2}{6} + \frac{2}{6}
    + \frac{2}{6} = 1, \]
    which is a contradiction. Therefore, $a \leq 6$. Now suppose that $a < 3$.
    Then $a \leq 2$ so
    \[ 1 = \frac{2}{a} + \frac{2}{b} + \frac{2}{c} \geq 1 + \frac{2}{b} +
    \frac{2}{c} > 1, \]
    which is also a contradiction. Therefore, $3 \leq a \leq 6$ as desired.

    Now we can plug in each value of $a$ and solve for $b$ and $c$. We will
    only show how to solve the equation for $a = 3$, as the other cases can be
    done the exact same way.
    When $a = 3$, the equation becomes
    \[ \frac{1}{6} = \frac{1}{b} + \frac{1}{c} \Longleftrightarrow (b - 6)(c -
    6) = 36. \]
    Notice that $b$ cannot be less than 6 or otherwise $c$ will be negative.
    This means that both $b - 6$ and $c - 6$ are positive divisors of 36. This
    gives us the following set of solutions pairs:
    \[ (b, c) \in \{ (7, 42) , (8, 24), (9, 18), (10, 15), (12, 12) \}. \]

    When $a = 4$, the equation becomes
    \[ \frac{1}{4} = \frac{1}{b} + \frac{1}{c} \Longleftrightarrow (b - 4)(c -
    4) = 16. \]
    In this case, the solution pairs are $(b, c) \in \{ (5, 20), (6, 12), (8,
    8) \}$. 
    
    When $a = 5$, the equation is
    \[ \frac{1}{10} = \frac{1}{3b} + \frac{1}{3c} \Longleftrightarrow (3b -
    10)(3c - 10) = 100. \]
    In this case, the solution pairs are $(b, c) \in \{(4, 20), (5, 10) \}$. 
    
    Finally, when $a = 6$, the equation is
    \[ \frac{1}{3} = \frac{1}{b} + \frac{1}{c} \Longleftrightarrow (b - 3)(c -
    3) = 9. \]
    In this case, the solution pairs are $(b, c) \in \{(4, 12), (6, 6) \}$. 
    
    The table below illustrates all the ordered pairs $(a, b, c)$ that satify
    the given equation. \qedhere
    \begin{center}
        \begin{tabular}{ |c|l| }
            \hline
            $a$ & Solution triples\\
            \hline
            3 & (3, 7, 42), (3, 8, 24), (3, 9, 18), (3, 10, 15), (3, 12, 12)\\
            4 & (4, 5, 20), (4, 6, 20), (4, 8, 8)\\
            5 & (5, 4, 20), (5, 5, 10)\\
            6 & (6, 4, 12), (6, 6, 6)\\
            \hline
        \end{tabular}
    \end{center}
\end{solution}
% Remark about SFFT

\begin{question}
    Let $\mathbb{Z}$ be the set of integers. Let $f:\mathbb{Z} \rightarrow
    \mathbb{Z}$ be a function from $\mathbb{Z}$ to $\mathbb{Z}$ such that 
    \[f(x + y) + f(x - y) = 2f(x) + 2f(y),\]
    for all integers $x$ and $y$. 
    \begin{enumerate}
        \item Show that $f(0) = 0$ and $f(2x) = 4f(x)$. 
        
        \item Show that $f(nx) = n^2f(x)$ for every positive integer $n$. 
        
        \item Show that $f(y) = f(-y)$. 
        
        \item Determine all functions $f:\mathbb{Z} \rightarrow \mathbb{Z}$ such that 
        \[f(x + y) + f(x - y) = 2f(x) + 2f(y)\] 
        for all integers $x$ and $y$. 
    \end{enumerate}
\end{question}
\begin{solution}
    Let $P(x, y)$ be the assertion $f(x + y) + f(x - y) = 2f(x) + 2f(y)$.
    \begin{align*}
        P(0, 0) &\Longrightarrow & f(0) + f(0) &= 2f(0) + 2f(0)\\
        & & 2f(0) &= 4f(0)\\
        & & f(0) &= 0\\
        P(x, x) &\Longrightarrow & f(2x) + f(0) &= 2f(x) + 2f(x)\\
        & & f(2x) &= 4x
    \intertext{We will prove the second part using strong induction. For the
    base case $n = 1$, we have to show $f(x) = f(x)$ and this is obviously
    true. So suppose that this is true for all $n \leq k$. We need to show that
    this is also true for $n = k + 1$. $P(kx, x)$ then gives}
        P(kx, x) &\Longrightarrow & f((k + 1)x) + f((k - 1)x) &= 2f(kx) + 2f(x)\\
        & & f((k + 1)x) &= 2f(kx) + 2f(x) - f((k - 1)x)\\
        & & &= 2k^2f(x) + 2f(x) - (k - 1)^2f(x)\\
        & & &= (k^2 + 2k + 1)f(x)\\
        & & &= (k + 1)^2f(x)
    \intertext{Therefore, by induction it follows that $f(nx) = n^2f(x)$ for
    all positive integers $n$. In particular if $n$ is a positive integer,
    $f(n) = n^2f(1)$.}
        P(0, y) &\Longrightarrow & f(y) + f(-y) &= 2f(0) + 2f(y)\\
        & & f(-y) &= f(y)
    \end{align*}
    Now from the second part, letting $x = -1$ gives $f(-n) = n^2f(-1) =
    (-n)^2f(1)$. Hence, we see that $f(x) = x^2f(1) = cx^2$ where $c$ is a
    constant, for all integers $x$. On the other hand, it is easy to check that
    all functions of the form $f(x) = cx^2$ satisfy the given equation.
    Therefore, it follows that this must be the only solution.
\end{solution}

\begin{question}
    Seven integers are written around a circle in a way that no two or three
    adjacent numbers have a sum divisible by 3. How many of these seven numbers
    are divisible by 3? 
\end{question}
\begin{solution}
    There are exactly 3 numbers which are divisible by 3.

    Call a number \emph{good} if it is divisible by 3, and \emph{bad}
    otherwise. Let $N$ be the number of good numbers. First, suppose that $N >
    3$. We claim that there are two adjacent good numbers. Otherwise, there has
    to be a bad number between each of those $N$ good numbers, which is not
    possible as there are only 7 numbers. Therefore, there are 2 adjacent good
    numbers, so their sum is  divisible by 3 which is a contradiction to the
    problem statement. Hence $N \leq 3$.

    Now suppose that $N < 3$. Then we claim that there are 3 consecutive bad
    numbers. Otherwise, there can only be atmost 2 bad numbers between any 2
    good numbers. This means that the total number of integers can only be at
    most 6, which is a contradiction as well. Therefore, there must be three
    consecutive bad integers. Since an integer can leave a remainder of 0, 1 or
    2 when divided by 3, a bad integer can only leave a remainder of 1 or 2
    when divided by 3. If all of those 3 consecutive bad numbers leave the same
    remainder when divided by 3, their sum is divisible by 3 which is a
    contradiction. If they leave different remainders when divided by 3, this
    means that there are 2 consecutive bad numbers which leave different
    remainders when divided by 3. Then their sum is divisible by 3, which is
    also a contradiction. Hence $N \geq 3$.

    These two parts show that $N = 3$ and we are done.
\end{solution}
\begin{remark}
    One trick that can help a lot in this problem is that we only need to care
    about the residue modulo 3 of each of the 7 numbers. That is, we can
    basically assume that each of those 7 integers are 0, 1 or 2. 
\end{remark}

\begin{question}
    \begin{enumerate}
        \item Find an ordered pair $(x,y)$ such that \[2019x + 2021y = 1\]
            where $x$ and $y$ are integers. 

        \item By using the ordered pair obtained in question 1, find all
            solutions of \[2019x + 2021y = 1\] where $x$ and $y$ are integers. 
    \end{enumerate}
\end{question}
\begin{solution}
    % Add a technique about Euclidean algorithm and Bezout's lemma.
    By the \hyperref[teq: euclideanalg]{Euclidean algorithm}, we can find a
    solution pair as follows:
    \begin{align*}
        2021 &= 2019 + 2\\
        2019 &= 2 \cdot 1009 + 1\\
        &= (2021 - 2019)(1009) + 1\\
        2019 &= 2021 \cdot 1009 - 2019 \cdot 1009 + 1\\
        1010 \cdot 2019 - 1009 \cdot 2021 &= 1.
    \end{align*}
    Therefore, $(x, y) = (1010, -1009)$ is a solution pair. Now suppose that
    $(x_0, y_0)$ is another pair that satisfies the given equation. i.e.
    \[ 2019x_0 + 2021y_0 = 1 = 1010 \cdot 2019 - 1009 \cdot 2021
    \Longleftrightarrow 2019(1010 - x_0) = 2021(y_0 + 1009). \]
    Now notice that $\gcd(2019, 2021) = 1$. Therefore, by \hyperref[lem:
    euclid]{Euclid's lemma}, $2021 \mid 1010 - x_0 \Rightarrow x_0 = 1010 -
    2021k$ for some integer $k$. Similarly, $2019 \mid y_0 + 1009 \Rightarrow
    y_0 = 2019j - 1009$ for some integer $j$. Substituting these back into the
    equation again gives
    \[ 2019 \cdot 2021 k = 2021 \cdot 2019 j \Longrightarrow k = j. \]
    Hence $(x_0, y_0) = (-2021k + 1010, 2019k - 1009)$ for some integer $k$.
    Now it is easy to check that all pairs of the form $(x, y) = (-2021k +
    1010, 2019k - 1009)$ satisfy the given equation. Therefore, it follows that
    they are the only solutions.
\end{solution}

\begin{question}
    In $\triangle ABC$, altitudes $AD$, $BE$ and $CF$ pass through the point
    $H$. Points $A'$, $B'$ and $C'$ are midpoints of $BC$, $CA$, $AB$
    respectively. Points $A''$, $B''$ and $C''$ are midpoints of $AH$, $BH$,
    $CH$ respectively. 
    \begin{enumerate}
        \item Prove that $B'C'B''C''$ is a rectangle. 
        
        \item Prove that $C'A'C''A''$ is a rectangle.
        
        \item Prove that the six points $A'$, $B'$, $C'$,$A''$, $B''$, $C''$
            are concyclic. 
        
        \item Prove that the nine points $A',B',C',A'',B'',C'',D,E,F$ are
            concyclic. 
    \end{enumerate}
\end{question}
\begin{solution}
    Since $C'$ and $B''$ are midpoints of sides $BA$ and $BH$ in $\triangle
    BAH$, it follows that $C'B'' \parallel AH$. Now $C'$ and $B'$ are midpoints
    of sides $AB$ and $AC$, so $B'C' \parallel BC$. Since $BC \perp AH$, $B'C'
    \perp B''C'$. Similarly, we can show that other three angles are also right
    angles, so $B'C'B''C''$ is a rectangle. 

    Similarly to above, we can show that $C'A'C''A''$ is also a rectangle. We
    now have
    \[ \angle CA'C'' = \angle CA''C'' = \angle CB'C'' = \angle CB''C'' = 90^\circ, \]
    so $A', B', C', A'', B'', C''$ all lie on the circle with diameter $CC''$.
    Call this circle $\omega$.
    \begin{center}
        \begin{asy}
            import olympiad;
            size(8cm);
            defaultpen(fontsize(11pt));
            pen mydash = linetype(new real[] {5,5});
            usepackage("contour", "outline");
            texpreamble("\contourlength{1pt}");
            pair A = dir(120);
            pair B = dir(210);
            pair C = dir(330);
            pair D = foot(A, B, C);
            pair E = foot(B, C, A);
            pair F = foot(C, A, B);
            pair H = A+B+C;
            pair A1 = midpoint(B--C);
            pair B1 = midpoint(C--A);
            pair C1 = midpoint(A--B);
            pair A2 = midpoint(A--H);
            pair B2 = midpoint(B--H);
            pair C2 = midpoint(C--H);
            draw(A--B--C--cycle, black+1);
            draw(A--D);
            draw(B--E);
            draw(C--F);
            draw(C1--B1--C2--B2--cycle);
            draw(C1--A2--C2--A1--cycle, mydash);
            draw(circumcircle(A1, B1, C1), mydash);
            dot("$A$", A, dir(A));
            dot("$B$", B, dir(225));
            dot("$C$", C, dir(315));
            dot("$A'$", A1, dir(270));
            dot("$B'$", B1, dir(45));
            dot("$C'$", C1, dir(165));
            dot("\contour{white}{$A''$}", A2, dir(45));
            dot("\contour{white}{$B''$}", B2, dir(270));
            dot("\contour{white}{$C''$}", C2, dir(270));
            dot("$H$", H, dir(305));
            dot("$D$", D, dir(270));
            dot("$E$", E, dir(45));
            dot("$F$", F, dir(165));
        \end{asy}
    \end{center}
    By \hyperref[sol: 2019 Regional Round P11]{2019 Regional Round problem 11},
    we see that $D$ lies on $(A'B'C')$, which is $\omega$. Similarly, $E$ and
    $F$ also lie on $\omega$, so combined with above we see that $D, E, F, A',
    B', C', A'', B'', C''$ all lie on $\omega$ as desired.
\end{solution}

\begin{question}
    Let $f:\mathbb{R} \rightarrow \mathbb{R}$ be a function such that 
    \[f(x + y) + f(x - y) = 2f(x)\cos y,\mathspace x, y \in \mathbb{R}.\]
    \begin{enumerate}
        \item Show that $f(\theta) + f(-\theta) = 2a\cos \theta$, where $a =
            f(0)$. 
        
        \item Show that $f(\theta + \pi) + f(\theta) = 0$.
        
        \item Show that $f(\theta + \pi) + f(-\theta) = -2b\sin\theta$, where
            $b = f\left(\frac{\pi}{2}\right)$.
        
        \item Find all functions $f:\mathbb{R} \rightarrow \mathbb{R}$ such
            that $$f(x + y) + f(x - y) = 2f(x)\cos y$$ where $x,y \in
            \mathbb{R}$.
    \end{enumerate}
\end{question}
\begin{solution}
    Let $P(x, y)$ be the assertion $f(x + y) + f(x - y) = 2f(x)\cos
    y,\mathspace x, y \in \mathbb{R}$. Then
    \begin{align*}
        P(0, \theta) &\Longrightarrow & f(\theta) + f(-\theta) &= 2f(0)\cos \theta\\
        & & &= 2a\cos \theta.\\
        P \left( \theta + \frac{\pi}{2}, \frac{\pi}{2} \right) &\Longrightarrow & f(\theta + \pi) + f(\theta) &= 2f \left( \theta + \frac{\pi}{2} \right) \cos \frac{\pi}{2}\\
        & & &= 0.\\
        P \left(\frac{\pi}{2}, \theta + \frac{\pi}{2} \right) & \Longrightarrow & f(\theta + \pi) + f(-\theta) &= 2f \left( \frac{\pi}{2} \right) \cos \left( \theta + \frac{\pi}{2} \right)\\
        & & &= -2b \sin \theta.
    \end{align*}
    Now subtracting the third equation from the second equation gives
    \begin{align*}
        f(\theta) - f(-\theta) &= 2b \sin \theta. \\
    \intertext{Adding this to the first equation gives}
        2f(\theta) &= 2a \cos \theta + 2b \sin \theta\\
        f(\theta) &= a \cos \theta + b \sin \theta. 
    \end{align*}
    Hence if $f$ satisfies the given equation, then $f(x) = a \cos x + b \sin
    x$ for some real constants $a$ and $b$. On the other hand, it is easy to
    check that all functions of the above form satisfy the given equation.
    Therefore, it follows that $f(x) = a \cos x + b \sin x$ is the only
    solution to the given functional equation.
\end{solution}

\begin{question}
    A student council must select a two-person welcoming committee and a
    three-person planning committee from its members. There are exactly 15 ways
    to select a two-person team for the welcoming committee. It is possible for
    students to serve on both committees. In how many different ways can a
    three-person planning committee be selected? 
\end{question}
\begin{solution}
    Let the number of students be $n$. Then
    \[ 15 = \binom{n}{2} = \frac{n(n - 1)}{2} \Longleftrightarrow (n - 6)(n +
    5) = 0. \]
    Since $n$ is a positive integer, $n = 6$. Therefore, the number of ways to
    choose a three-person planning committee is
    \[ \binom{6}{3} = \frac{6 \cdot 5 \cdot 4}{1 \cdot 2 \cdot 3} = 20.
    \qedhere \]
\end{solution}

\begin{question}
    $AB$ is a diameter of a circle $O$ with radius 10 cm. $OQ$ is a radius of a
    circle $O$. such that $QO \perp AB$. A point $P$ is on $OQ$. Draw a
    semicircle centered at $P$ with diameter $CD$ where $CD$ is the chord of
    circle $O$ and $CD \perp PQ$. $PQ$ produced meets the semicircle at $R$.
    Find the maximum possible length of $QR$.
\end{question}
\begin{solution}
    The length of $QR$ is
    \[ QR = PR - PQ = PC - OQ + OP = PC + OP - 10. \]
    By the \hyperref[thm: amgm]{AM-RMS inequality}, 
    \[ CP + OP - 10 \leq 2\sqrt{\frac{CP^2 + OP^2}{2}} - 10 =
    2\sqrt{\frac{100}{2}} - 10 = 10(\sqrt{2} - 1). \]
    Hence the maximum value of $QR$ is $10(\sqrt{2} - 1)$, achieved when $CP = OP = 5\sqrt{2}$.
\end{solution}
\begin{center}
    \begin{asy}
        import olympiad;
        size(7cm);
        defaultpen(fontsize(11pt));
        pen mydash = linetype(new real[] {5,5});
        pair A = dir(180);
        pair B = dir(0);
        pair Q = dir(90);
        real s = 50;
        pair D = dir(s);
        pair C = dir(180-s);
        pair P = midpoint(C--D);
        pair O = (0, 0);
        pair R = P+abs(P-D)*unit(P-O);
        draw(circle(O, 1));
        draw(A--B);
        draw(O--R);
        draw(C--D);
        draw(arc(P, abs(P-D), 0, 180));
        draw(rightanglemark(R, P, D, 2.5));
        draw(rightanglemark(R, O, B, 2.5));
        dot("$A$", A, dir(180));
        dot("$B$", B, dir(0));
        dot("$O$", O, dir(270));
        dot("$P$", P, dir(225));
        dot("$C$", C, dir(165));
        dot("$D$", D, dir(15));
        dot("$Q$", Q, dir(135));
        dot("$R$", R, dir(90));
    \end{asy}
\end{center}

\begin{question}
    Find all positive integers $s$ such that $\ceil*{\frac{s}{3}} - 21 =
    \ceil*{\frac{s}{5}}$ where $\ceil*{x}$ is the smallest integer greater than
    or equal to $x$. For example, $\ceil*{3.7} = 4$, $\ceil*{3} = 3$ and
    $\ceil*{3.2} = 4$. 
\end{question}
\begin{solution}
    It is easy to see that for any real number $x$,
    \[ x \leq \ceil{x} < x + 1. \]
    Therefore, from the given equation, we have
    \[ \frac{s}{3} - \frac{s}{5} - 1 < \ceil*{\frac{s}{3}} -
    \ceil*{\frac{s}{5}} = 21 < \frac{s}{3} + 1 - \frac{s}{5}. \]
    Simplifying the first part of the inequality gives us
    \[ 5s - 3s - 15 < 315 \Longrightarrow s < 165. \]
    Similarly, the second part of the inequality gives us $s > 150$. Therefore,
    $s$ must be in the set $\{ 151, 152, \ldots, 164 \}$. By checking each
    element in the set, we see that the only possible values of $s$ are 154,
    155, 157, 158, 159, 161, 162.
\end{solution}

\begin{question}
    In $\triangle ABC$, $AB = 94$ and $AC = 107$. A circle with center $A$ and
    radius $AB$ intersects $BC$ at points $B$ and $X$. Moreover, $BX$ and $CX$
    have integer lengths. What is $BC$?
\end{question}
\begin{solution}
    Since $BX$ and $CX$ are integers, $BC$ must also be an integer too. Let the
    circle intersect line $AC$ at points $D$ and $E$. Then since $B$, $X$, $D$,
    $E$ are concyclic,
    \[ CX \cdot CB = CD \cdot CE = (CA + AB)(CA - AB) = CA^2 - AB^2 = 107^2 -
    94^2 = 2613. \]
    \begin{center}
        \begin{asy}
            import olympiad;
            size(6.5 cm);
            defaultpen(fontsize(11pt));
            pen mydash = linetype(new real[] {5,5});
            pair A = dir(130);
            pair B = dir(210);
            pair C = dir(330);
            pair X = 2*foot(A, B, C)-B;
            pair D = A+abs(A-B)*unit(A-C);
            pair E = A+abs(A-B)*unit(C-A);
            draw(A--B--C--cycle, black+1);
            draw(A--X);
            draw(A--D);
            draw(circle(A, abs(A-B)), mydash);
            dot("$A$", A, 1.5*dir(90));
            dot("$B$", B, dir(225));
            dot("$C$", C, dir(315));
            dot("$X$", X, dir(270));
            dot("$D$", D, dir(135));
            dot("$E$", E, dir(10));
        \end{asy}
    \end{center}
    By the triangle inequality, $CB < CA + AB = 94 + 107 = 201$. We also have
    $CB > CX$ since $X$ lies on the side $BC$. Since the factors of 2613 are 1,
    3, 13, 39, 67, 201, 871 and 2613, we see that $BC$ must be 67.
\end{solution}

\begin{question}
    Seven people are sitting around a circular table, each holding a fair coin.
    All seven people flip their coins and those who flip heads stand while
    those flip tails seated. What is the probability that no two people
    adjacent will stand? 
\end{question}
\begin{solution}
    Label the people from 1 to 7. First let's compute the total number of
    configurations. Each person has 2 possible states, either seated or
    standing. Since there are 7 people, the total number of configurations is
    $2^7 = 128$.

    Now let's calculate the number of configurations where no two people
    standing are adjacent. If the number of people standing is greater than or
    equal to 4, there will be at least two of them which are adjacent.
    Therefore, there can only be at most 3 people who are standing. Then it is
    easy to see that the following are all the possible configurations, unique
    up to rotation\footnote{This means that any other configuration can be
    obtained by rotating one of these configurations.}. The blue blocks denote
    the people who are standing.
    \begin{center}
        \begin{asy}
            size(13cm);
            defaultpen(fontsize(9pt));

            real s = 360/7;
            real r1 = .8;
            real r2 = 1.2;
            pair S = (3, 0);
            pen filler = Cyan;

            path block(int i, pair O){
                return arc(O, r1, (-(i - 1) + 1/2)*s + 90, (-(i - 1) - 1/2)*s + 90)--arc(O, r2, (-(i - 1) - 1/2)*s + 90, (-(i - 1) + 1/2)*s + 90)--cycle;
            }

            pair O1 = (0, 0);
            pair O2 = S + O1;
            pair O3 = S + O2;
            pair O4 = S + O3;
            fill(block(1, O1), filler);
            fill(block(1, O2), filler);
            fill(block(3, O2), filler);
            fill(block(1, O3), filler);
            fill(block(4, O3), filler);
            fill(block(1, O4), filler);
            fill(block(3, O4), filler);
            fill(block(5, O4), filler);

            real t = 1;
            for (int j = 0; j < 4; ++j){
                pair O = j*S + (0, 0);
                draw(circle(O, r1), black+t);
                draw(circle(O, r2), black+t);
                for (int i = 0; i < 7; ++i){
                    pair X = O + dir(-i*s + 90);
                    pair Y = dir((-i + 1/2)*s + 90);
                    label(""+string(i + 1)+"", X);
                    draw((O + r1*Y)--(O + r2*Y), black+t);
            }
            }
        \end{asy}
    \end{center}
    Since there are 7 ways to rotate each of those configurations, and since
    there are 4 of them, the number of configurations is 28. Also counting the
    configuration where no person is standing, the final number of
    configurations is 29.

    Therefore, the probability required is $\frac{29}{128}$.
\end{solution}
\begin{remark}
    You can also use the same technique as in \hyperref[sol: 2016 National
    Round P10]{2016 National Round problem 10} to show that $f(C_7) = 29$.
\end{remark}
\begin{question}
    \begin{enumerate}
        \item Let $a$ and $b$ be positive integers. If there are integers
            $x_{0}$, $y_{0}$ such that $ax_{0} + by_{0} = 1$, then prove that
            the greatest common divisor of $a$ and $b$ is 1. 
        
        \item Prove that the fraction $\frac{12n + 5}{14n + 6}$ is in lowest
            terms for every positive integer $n$.
    \end{enumerate}
\end{question}
\begin{solution}
    Let the $\gcd(a, b) = d$. Since $d$ divides both $a$ and $b$, $d$ also
    divides $ax_0$ and $by_0$ and hence
    \[ d \mid (ax_0 + by_0) = 1. \]
    Since the only positive divisor of 1 is 1 itself, it follows that $d = 1$.
    Now notice that
    \[ 6(14n + 6) + (-7)(12n + 5) = 1 \]
    for any positive integer $n$. Therefore, $\gcd(14n + 6, 12n + 5) = 1$, and
    so $\frac{12n + 5}{14n + 6}$ is in lowest terms for every positive integer
    $n$.
\end{solution}

\begin{question}
    $\triangle ABC$ is inscribed in a circle. Altitudes $AD$, $BE$ and $CF$
    pass through the point $H$. $EF$ produced meets the circle at $P$. $BP$
    produced and $DF$ produced meet at the point $Q$. 
    \begin{enumerate}
        \item Show that $\angle ACF = \angle ADF = \angle ABE$.
        
        \item Show that $\angle AFQ = \angle ACD$. 
        
        \item Show that $AP = AQ$. 
    \end{enumerate}
\end{question}
\begin{solution}
    Since $\angle AFC = \angle ADC = 90^\circ$, $AFDC$ is cyclic, and so
    \[ \angle ACF = \angle ADF. \]
    Also $\angle HDB + \angle HFB = 90^\circ + 90^\circ = 180^\circ$, so $HFBD$
    is also cyclic and 
    \[ \angle ADF = \angle HDF = \angle HBF = \angle ABE. \]
    \begin{center}
        \begin{asy}
            import olympiad;
            size(8cm);
            defaultpen(fontsize(11pt));
            pen mydash = linetype(new real[] {5,5});
            usepackage("contour", "outline");
            texpreamble("\contourlength{1pt}");
            pair A = dir(60);
            pair B = dir(210);
            pair C = dir(330);
            pair H = orthocenter(A, B, C);
            pair D = foot(A, B, C);
            pair E = foot(B, C, A);
            pair F = foot(C, A, B);
            pair O = (0, 0);
            pair P[] = intersectionpoints(E--(F+unit(F-E)), circle(O, 1));
            pair P = P[0];
            pair Q = extension(B, P, D, F);
            draw(A--B--C--cycle, black+1);
            draw(A--D);
            draw(B--E);
            draw(C--F);
            draw(Q--B);
            draw(Q--D);
            draw(P--E);
            draw(P--A);
            draw(Q--A);
            draw(circumcircle(A, P, Q), mydash);
            draw(rightanglemark(A, D, B, 2.5));
            draw(rightanglemark(B, E, C, 2.5));
            draw(rightanglemark(C, F, A, 2.5));
            draw(circle(O, 1));
            dot("$A$", A, dir(45));
            dot("$B$", B, dir(225));
            dot("$C$", C, dir(315));
            dot("$D$", D, dir(270));
            dot("$E$", E, dir(45));
            dot("\contour{white}{$F$}", F, dir(80));
            dot("$P$", P, dir(180));
            dot("$Q$", Q, dir(135));
            dot("\contour{white}{$H$}", H, dir(315));
        \end{asy}
    \end{center}
    As $AFDC$ is cyclic, we also have $\angle AFQ = \angle ACD$. But since
    $APBC$ is cyclic, $\angle ACD = \angle APQ$ so $\angle AFQ = \angle APQ$
    and hence $AQPF$ is cyclic. Since we also have $\angle BFC = \angle BEC =
    90^\circ$, $BFEC$ is also cyclic. Therefore, finally,
    \[ \angle AQP = \angle AFE = \angle ACB = \angle APQ, \]
    and so $AP = AQ$ as desired.
\end{solution}

\begin{question}
    A car drives from town $A$ to $B$ at the average speed of 30 km/h, from
    town $B$ to town $C$ at average speed of 60 km/h; and on the way back, the
    car drives from $C$ to $B$ at average speed of 30 km/h, from town $B$ to
    $A$ at average speed of 60 km/h. The whole trip takes 6 hours. What is the
    total distance of the round trip?
\end{question}
\begin{solution}
    Let $x$ and $y$ be the distances in kilometres between town $A$ and town
    $B$, and town $B$ and town $C$. Then from the given conditions we have
    \[ \frac{x}{30} + \frac{y}{60} + \frac{y}{30} + \frac{x}{60} = 6
    \Longrightarrow (x + y)\left( \frac{1}{30} + \frac{1}{60} \right) = 6. \]
    Solving this gives $x + y = 120$, so the total distance of the round trip
    is $2(x + y) = 240 \text{ km}$. 
\end{solution}

\begin{question}
    \begin{enumerate}
        \item Prove that the square of an odd number gives the remainder 1 upon
            dividing by 8. 
        
        \item Prove that if $k$ is odd and $n$ is a positive integer, then
            $k^{2^n} - 1$ is divisible by $2^{n + 2}$. 
    \end{enumerate}
\end{question}
\begin{solution}
    Let the odd number be $2m + 1$. Then 
    \[ (2m + 1)^2 - 1 = 4m(m + 1). \]
    Since $m$ and $m + 1$ are consecutive integers, one must be even and the
    other must be odd. Therefore, their product must be even, and hence $4m(m +
    1) = (2m + 1)^2 - 1$ must be divisible by 8. Thus $(2m + 1)^2$ leaves a
    remainder of 1 when divided by 8.

    Now we will prove the general statement by induction on $n$. The base case
    $n = 1$ is just the first part. Now suppose that this is true for $n = p$.
    When $n = p + 1$,
    \[ k^{2^{p + 1}} - 1 = k^{2^p \cdot 2} - 1 = (k^{2^p})^2 - 1 = (k^{2^p} +
    1)(k^{2^p} - 1). \]
    Since $k$ is odd, $k^x$ is odd for any positive integer $x$. Therefore,
    $k^{2^p}$ is odd implying $k^{2^p} + 1$ is even. Hence the left bracket in
    the above expression is divisible by 2. Meanwhile, the right bracket is
    divisible by $2^{p + 2}$ by the induction hypothesis. Therefore, the whole
    expression is divisible by $2^{(p + 1) + 2}$, and so by induction it
    follows that $k^{2^n} - 1$ is divisible by $2^{n + 2}$ for all positive
    integers $n$.
\end{solution}
