\solutionheader{2018 Regional Round}
\begin{question}
    In the figure, $AP$, $BQ$, $CR$ are perpendicular to the straight line
    $ABC$. Prove that 
    \begin{enumerate}
        \item $\triangle PAB \sim \triangle RCB$
        
        \item $\frac{1}{BQ} = \frac{1}{AP} + \frac{1}{CR}$. 
    \end{enumerate}
\end{question}
\begin{solution}
    Notice that $\angle PAQ = \angle CRQ$ and $\angle APQ = \angle RCQ$, so
    $\triangle PAQ \sim \triangle CRQ$. Since $BQ \parallel PA$, we have
    \[ \frac{AB}{BC} = \frac{PQ}{QC} = \frac{PA}{RC} \Longrightarrow
    \frac{PA}{AB} = \frac{RC}{CB}. \]
    As $\angle PAB = \angle RCB = 90^\circ$, we see that $\triangle PAB \sim
    \triangle RCB$.
    \begin{center}
        \begin{asy}
            import olympiad;
            size(6cm);
            defaultpen(fontsize(11pt));
            pen mydash = linetype(new real[] {5,5});
            pair A = (-1, 0);
            pair B = (0, 0);
            pair C = (1.5, 0);
            pair P = (-1, 2/3);
            pair Q = extension(B, rotate(90, B)*A, P, C);
            pair R = extension(A, Q, rotate(90, C)*A, C);
            draw(A--R--C--P--cycle);
            draw(A--C);
            draw(P--B--R);
            draw(Q--B);
            dot("$A$", A, dir(225));
            dot("$C$", C, dir(315));
            dot("$B$", B, dir(270));
            dot("$P$", P, dir(135));
            dot("$Q$", Q, dir(90));
            dot("$R$", R, dir(45));
        \end{asy}
    \end{center}
    Finally, 
    \[ \frac{BQ}{AP} + \frac{BQ}{CR} = \frac{CB}{CA} + \frac{AB}{CA} = 1
    \Longrightarrow \frac{1}{AP} + \frac{1}{CR} = \frac{1}{BQ}. \qedhere \]
\end{solution}

\begin{question}
    $PA$ and $PB$ are the tangent segments at $A$ and $B$ to a circle whose
    center is $O$. $AB$ and $OP$ are intercept at $Q$. Prove that $AB \perp
    OP$. Hence show that $OQ : QP = AO^2 : AP^2$. Hence also show that
    $\alpha(\triangle OAB) : \alpha(\triangle PAB)=AO^2 : AP^2$.
\end{question}
\begin{solution}
    Since $PA = PB$ and $OA = OB$, $OAPB$ is a kite, and hence $AB \perp OP$.
    Now, $Q$ is the foot of perpendicular from $A$ in right triangle $OAP$.
    Therefore,
    \[ \frac{AO^2}{AP^2} = \frac{OQ \cdot OP}{PQ \cdot PO} = \frac{OQ}{QP}. \]
    \begin{center}
        \begin{asy}
            import olympiad;
            size(7cm);
            defaultpen(fontsize(11pt));
            pen mydash = linetype(new real[] {5,5});
            pair C = dir(120);
            pair A = dir(210);
            pair B = dir(330);
            pair O = circumcenter(A, B, C);
            pair P = 2*circumcenter(A, O, B)-O;
            pair Q = extension(A, B, O, P);
            draw(A--B--C--cycle, black+1);
            draw(P--A--O--B--cycle);
            draw(O--P);
            draw(circle(O, 1));
            draw(rightanglemark(O, Q, A, 3));
            dot("$A$", A, dir(A));
            dot("$B$", B, dir(B));
            dot("$C$", C, dir(C));
            dot("$O$", O, dir(90));
            dot("$P$", P, dir(270));
            dot("$Q$", Q, dir(45));
        \end{asy}
    \end{center}
    Since $\triangle OAB$ and $\triangle PAB$ share the same base $AB$, the
    ratio of their areas is equal to the ratio of their heights. Hence
    \[ \frac{[\triangle OAB]}{[\triangle PAB]} = \frac{OQ}{QP} =
    \frac{AO^2}{AP^2}. \qedhere \]
\end{solution}

\begin{question}
    If $T_{1}$, $T_{2}$, $T_{3}$ are the sums of $n$ terms of three series in
    AP, the first term of each being $a$ and the respective common differences
    being $d$, $2d$, $3d$, then show that $T_{1} + T_{3} = 2T_{2}$.
\end{question}
\begin{solution}
    Let $(x_k)$, $(y_k)$ and $(z_k)$ denote the three APs. First, for any $i
    \in \mathbb{N}$, we have
    \[ x_i + z_i = a + (i - 1)d + a + 3(i - 1)d = 2a + 4(i - 1)d = 2(a + 2(i -
    1)d) = 2y_i. \]
    Therefore,
    \[ T_1 + T_3 = \sum_{i = 1}^n x_i + \sum_{i = 1}^n z_i = \sum_{i = 1}^n(x_i
    + z_i) = \sum_{i = 1}^n 2y_i = 2\sum_{i = 1}^n y_i = 2T_2. \qedhere \]
\end{solution}

\begin{question}
    The positive difference between the zeros of the quadratic expression $x^2
    + kx + 3$ is $\sqrt{69}$. Find the possible values of $k$.
\end{question}
\begin{solution}
    Let the roots of this polynomial be $p$ and $q$ and WLOG assume that $p
    \geq q$. Then by \hyperref[thm: vieta]{Vieta's formulas}, $p + q = -k$ and
    $pq = 3$. We also have $p - q = \sqrt{69}$. Therefore,
    \[ 3 = pq = \frac{(p + q)^2 - (p - q)^2}{4} = \frac{k^2 - 69}{4}
    \Longrightarrow k^2 = 81. \]
    Hence $k = \pm 9$.
\end{solution}

\begin{question}
    In $\triangle PQR$, $\angle Q = 90^\circ$ and $S$ is a point on $PR$ such
    that $QS \perp PR$. If $PR = kQR$, then show that $PS = (k^2 - 1)RS$.
\end{question}
\begin{solution}
    Since $R$ is the foot of perpendicular from $Q$ in right triangle $PQR$,
    \[ \frac{PS}{RS} = \frac{PS \cdot PR}{RS \cdot PR} = \frac{PQ^2}{QR^2}. \]
    \begin{center}
        \begin{asy}
            import olympiad;
            size(7cm);
            defaultpen(fontsize(11pt));
            pen mydash = linetype(new real[] {5,5});
            pair P = dir(180);
            pair R = dir(0);
            pair Q = dir(110);
            pair S = foot(Q, P, R);
            draw(P--Q--R--cycle, black+1);
            draw(Q--S);
            draw(rightanglemark(Q, S, R, 3));
            draw(rightanglemark(P, Q, R, 3));
            dot("$P$", P, dir(225));
            dot("$R$", R, dir(315));
            dot("$Q$", Q, dir(90));
            dot("$S$", S, dir(270));
        \end{asy}
    \end{center}
    Now by Pythagoras's theorem,
    \[ PQ^2 = PR^2 - QR^2 = k^2QR^2 - QR^2 = (k^2 - 1)QR^2 \Longrightarrow
    \frac{PQ^2}{QR^2} = k^2 - 1. \]
    Therefore, $PS = (k^2 - 1)RS$.
\end{solution}

\begin{question}
    Prove the following theorem:
    
    If a ray from the vertex of an angle of a triangle divides the opposite
    side into segments that have the same ratio as the other two sides, then it
    bisects the angle.
\end{question}
\begin{solution}
    Let the triangle be $ABC$. Let the ray originate at $A$ and let it
    intersect side $BC$ at $D$. Finally, let $E$ and $F$ be the foots of
    perpendiculars from $D$ onto sides $AB$ and $AC$.
    \begin{center}
        \begin{asy}
            import olympiad;
            size(7cm);
            defaultpen(fontsize(11pt));
            pen mydash = linetype(new real[] {5,5});
            pair A = dir(120);
            pair B = dir(210);
            pair C = dir(330);
            pair I = incenter(A, B, C);
            pair D = extension(A, I, B, C);
            pair E = foot(D, A, B);
            pair F = foot(D, A, C);
            draw(A--B--C--cycle, black+1);
            draw(A--D);
            draw(D--E);
            draw(D--F);
            draw(rightanglemark(D, E, A, 2.5));
            draw(rightanglemark(D, F, A, 2.5));
            dot("$A$", A, dir(A));
            dot("$B$", B, dir(B));
            dot("$C$", C, dir(C));
            dot("$D$", D, dir(270));
            dot("$E$", E, dir(180));
            dot("$F$", F, dir(45));
        \end{asy}
    \end{center}
    The problem condition then gives us $BD : DC = BA : AC$. Since $\triangle
    ABD$ and $\triangle ACD$ share the same height, the ratio of their areas is
    the ratio of the bases. Therefore,
    \[ \frac{BD}{DC} = \frac{[ABD]}{[ACD]} = \frac{BA \cdot DE}{AC \cdot DF} \Longrightarrow DE = DF. \]
    In right triangles $ADE$ and $ADF$, we have $DE = DF$ and hypotenuse $AD$,
    so they are congruent. Thus $\angle BAD = \angle CAD$ as desired.
\end{solution}
\begin{remark}
    This is commonly known as the angle bisector theorem.
\end{remark}

\begin{question}
    The sum to $k$ terms of an AP is 21. The sum to $2k$ terms is 78. The $k$th
    term is 11. Find the first term and the common difference. 
\end{question}
\begin{solution}
    Let $u_1$ and $d$ be the first term and common difference of the AP. 
    Then we have the following three equations:
    \begin{align*}
        21 &= \frac{k(u_1 + u_k)}{2},\\
        78 &= k(u_1 + u_{2k}),\\
        11 &= u_k.
    \end{align*}
    Notice that $u_{2k}$ is obtained by adding $d$ to $u_k$ for $k$ more times.
    Therefore, $u_{2k} = u_k + kd$. From the first equation, we have $k(u_1 +
    u_k) = 42$. Therefore, the second equation gives
    \[ 78 = k(u_1 + u_k) + k^2d = 42 + k^2d \Longrightarrow k^2d = 36. \]
    We also have $u_k = u_1 + (k - 1)d$, so $u_1 = u_k - (k - 1)d$.
    Substituting this into the second equation gives
    \begin{align*}
        k(u_1 + u_{2k}) &= 78\\
        k(u_k - (k - 1)d + u_k + kd) &= 78\\
        k(2u_k + d) &= 78\\
        22k + kd &= 78\\
        22k^2 + k^2d &= 78k\\
        22k^2 - 78k + 36 &= 0\\
        11k^2 - 39k + 18 &= 0\\
        (11k - 6)(k - 3) &= 0.
    \end{align*}
    Since $k$ is a positive integer, it follows that $k = 3$ and hence $d = 4$.
    Therefore,
    \[ u_1 = u_k - (k - 1)d = 11 - 2\cdot 4 = 3. \qedhere \]
\end{solution}

\begin{question}
    Prove that if $a$, $b$, $c$ and $d$ are positive, the equation $x^4 + bx^2
    + cx - d = 0$ has one positive, one negative and two imaginary roots.
\end{question}
\begin{solution}
    Let $f(x) = x^4 + bx^2 + cx - d$. This polynomial has exactly one sign
    change, so by \hyperref[thm: ruleofsigns]{Descartes' rule of signs}, it has
    exactly one positive root. Now $f(-x) = x^4 + bx^2 - cx - d$ also has one
    sign change, so $f$ must also have exactly one negative root. Since $f$ is
    a polynomial of degree 4, by the \hyperref[thm:
    fundamentalthmofalg]{fundamental theorem of algebra}, it must have exactly
    4 complex roots. Since we know that it only has 2 real roots, the rest two
    roots must be imaginary and we are done.
\end{solution}

\begin{question}
    Show that the sum of the squares of the first $n$ odd numbers is
    $\frac{1}{3}n(4n^2 - 1)$.
\end{question}
\begin{solution}
    This is the same problem as \hyperref[sol: 2016 Regional Round P14]{2016
    Regional Round problem 14}. 
\end{solution}

\begin{question}
    If $100!$ is divisible by $7^{n}$, find the maximum value of $n$. 
\end{question}
\begin{solution}
    Since $14 \cdot 7 < 100 < 15 \cdot 7$, there are 14 numbers less than 100
    which are divisible by 7. Out of these 14 numbers, there are 2 numbers,
    namely 49 and 98 which are divisible by $7^2$. Therefore, the power of 7 in
    the prime factorization of $100!$ is $14 + 2 = 16$ and so $n = 16$.
\end{solution}
\begin{remark}
    The argument in this problem can be generalized to get what is known as
\href{https://en.wikipedia.org/wiki/Legendre%27s_formula}{Legendre's formula}.
\end{remark}

\begin{question}
    Show that $x = 10^\circ$ is a solution of $2\sin x=\frac{1 + \tan^2x}{3 -
    \tan^2x}$.
\end{question}
\begin{solution}
    We will first show the identity
    \[ \sin 3x = 3\sin x - 4\sin^3 x \]
    for any real $x$. This can be done as follows:
    \begin{align*}
        \sin 3x &= \sin(2x + x)\\
        &= \sin 2x \cos x + \cos 2x \sin x\\
        &= 2\sin x \cos^2 x + (1 - 2\sin^2 x)\sin x\\
        &= \sin x (2 - 2\sin^2 x + 1 - 2\sin^2 x)\\
        &= 3\sin x - 4\sin^3 x
    \end{align*}
    In particular, when $x = 10^\circ$,
    \begin{align*} 
        \frac{1}{2} &= 3\sin 10^\circ - 4\sin^3 10^\circ\\
        \frac{1}{2} &= \sin 10^\circ (3 - 4\sin^2 10^\circ)\\
        2 \sin 10^\circ &= \frac{1}{3 - 4\sin^2 10^\circ}\\
        &= \frac{1}{4\cos^2 10^\circ - 1}\\
        &= \frac{\sec^2 10^\circ}{4 - \sec^2 10^\circ}\\
        &= \frac{1 + \tan^2 10^\circ}{3 - \tan^2 10^\circ}
    \end{align*}
    so $x = 10^\circ$ is a solution to the given equation.
\end{solution}

\begin{question}
    Show that $\frac{1}{\sqrt{1}} + \frac{1}{\sqrt{2}} + \cdots +
    \frac{1}{\sqrt{n}} > \sqrt{n}$ for all $n > 1$.
\end{question}
\begin{solution}
    For $1 \leq i < n$, we have $\frac{1}{\sqrt{i}} > \frac{1}{\sqrt{n}}$.
    Therefore,
    \[ \frac{1}{\sqrt{1}} + \frac{1}{\sqrt{2}} + \cdots + \frac{1}{\sqrt{n}} >
    \frac{1}{\sqrt{n}} + \frac{1}{\sqrt{n}} + \cdots + \frac{1}{\sqrt{n}} =
    \frac{n}{\sqrt{n}} = \sqrt{n}. \qedhere \]
\end{solution}

\begin{question}
    \begin{enumerate}
        \item Prove the following Cauchy inequality:
            \par For any real numbers $a_1, \ldots, a_n$ and $b_1, \ldots, b_n$,
            \[(a_{1}b_{1} + a_{2}b_{2} + \cdots +a_{n}b_{n})^2\leq (a_{1}^2 +
            a_{2}^2 + \cdots + a_{n}^2)(b_{1}^2 + b_{2}^2 + \cdots + b_{n}^2).\]
        
        \item For a set of positive real numbers $x_1, \ldots, x_n$, the
            Root-Mean Square RMS is defined by the formula 
            \[\text{RMS} = \sqrt{\frac{x_{1}^2 + \cdots + x_{n}^2}{n}}\] 
            and the Arithmetic Mean AM is defined by the formula 
            \[\text{AM} = \frac{x_{1} + \cdots + x_{n}}{n}.\] 
            Prove that $\text{RMS} \geq \text{AM}$.
    \end{enumerate}
\end{question}
\begin{solution}
    Consider the following quadratic polynomial in $x$:
    \[ P(x) = (a_1x + b_1)^2 + (a_2x + b_2)^2 + \cdots + (a_nx + b_n)^2 \geq 0.
    \]
    Letting $A = a_1^2 + \cdots + a_n^2$, $B = 2(a_1b_1 + \cdots + a_nb_n)$ and
    $C = b_1^2 + \cdots + b_n^2$, the quadratic can be rewritten as
    \[ P(x) = Ax^2 + Bx + C \geq 0. \]
    Since this quadratic is non-negative, it can only have at most one real
    root (i.e. the parabola cannot intersect the $x$ axis at two distinct
    points.) Therefore, its discriminant is less than or equal to zero; which
    yields the desired inequality.
    \[ B^2 - 4AC \leq 0 \Longrightarrow (a_1b_1 + \cdots + a_nb_n)^2 \leq
    (a_1^2 + \cdots + a_n^2)(b_1^2 + \cdots +b_n^2). \]
    Now letting $a_i = x_i$ and $b_i = 1$ for all $1 \leq i \leq n$, by the
    \hyperref[thm: cs]{Cauchy-Schwarz inequality},
    \[ (x_1 + x_2 + \cdots + x_n)^2 \leq (x_1^2 + x_2^2 + \cdots + x_n^2)n. \]
    Rearranging the equation gives
    \[ \frac{x_1 + x_2 + \cdots + x_n}{n} \leq \sqrt{\frac{x_1^2 + x_2^2 +
    \cdots + x_n^2}{n}}. \qedhere \]
\end{solution}

\begin{question}
    \begin{enumerate}
        \item Prove that if $p$ is a prime number, then the coefficient of
            every term in the expansion of $(a + b)^p$ except the first and
            last is divisible by $p$. 
        
        \item Hence show that if $p$ is a prime number and $N$ is a positive
            integer, then $N^{p} - N$ is a multiple of $p$. 
        
        \item Hence also show that if $p$ is a prime number, then $10^p - 7^p -
            3$ is divisible by $p$. 
    \end{enumerate}
\end{question}
\begin{solution}
    This problem is a step-by-step proof of \hyperref[thm: FLT]{Fermat's little theorem}.
    \begin{enumerate}
        \item By the binomial theorem,
            \[ (a + b)^p = a^p + \binom{p}{1}a^{p - 1}b + \cdots + \binom{p}{p -
            1}ab^{p - 1} + b^p. \]
            For $1 \leq r \leq p - 1$,
            \[ \binom{p}{r} = \frac{p(p - 1) \cdots (p - r + 1)}{r!}. \]
            Since $r < p$, there are no factors of $p$ in the denominator. However,
            $p$ obviously divides the numerator. Since $\binom{p}{r}$ is an
            integer, during the cancellation, there was no number in the
            denominator that could cancel out the $p$ in the denominator.
            Therefore, $p$ divides $\binom{p}{r}$. 

        \item We will show that $n^p - n$ is divisible by $p$, by induction on
            $n$. For the base case, when $n = 1$, this is equal to 0 so it is
            divisible by $p$. Now suppose that for $n = k$, $k^p - k$ is
            divisible by $p$. Then when $n = k + 1$,
            \[ (k + 1)^p - (k + 1) = (k^p - k) + \left( \binom{p}{1}k^{p - 1} +
            \binom{p}{2}k^{p - 2} + \cdots + \binom{p}{p - 1}k \right).\]
            Since $p$ divides the first bracket by the induction hypothesis and
            the second bracket by the first part, it follows that $p$ also
            divides $(k + 1)^p - (k + 1)$. Therefore, by induction it follows
            that $p$ divides $n^p - n$ for all positive integers $n$.

        \item Notice that $10^p - 7^p - 3 = (10^p - 10) - (7^p - 7)$. Since
            both brackets are divisible by $p$, we are done. \qedhere
    \end{enumerate}
\end{solution}
