\solutionheader{2019 Regional Round}
\begin{question}
    \begin{enumerate}
        \item Show that $x^2 + y^2 \geq 2xy$ for any two real numbers $x$ and
            $y$, and that the sign of the equality holds if and only if $x=y$. 
        
        \item By using similar triangles, prove that $h = \frac{ab}{c}$ for the
            given figure.
            \begin{center}
                \begin{asy}
                    import olympiad;
                    size(7cm);
                    defaultpen(fontsize(11pt));
                    pen mydash = linetype(new real[] {5,5});
                    real s = -.16;
                    pair A = dir(180);
                    pair B = dir(0);
                    pair A1 = (xpart(A), ypart(A)+s);
                    pair B1 = (xpart(B), ypart(B)+s);
                    pair M = midpoint(A1--B1);
                    real k = .1;
                    pair MA = M + k*unit((-1, 0));
                    pair MB = M + k*unit((1, 0));
                    pair C = dir(110);
                    pair D = foot(C, A, B);
                    draw(A--B--C--cycle, black+1);
                    draw(C--D);
                    draw(MA--A1, arrow=Arrow(size=5));
                    draw(MB--B1, arrow=Arrow(size=5));
                    draw(rightanglemark(A, C, B, 3));
                    draw(rightanglemark(B, D, C, 3));
                    dot("$A$", A, dir(225));
                    dot("$B$", B, dir(315));
                    dot("$C$", C, dir(90));
                    dot("$D$", D, dir(270));
                    label("$b$", midpoint(C--A), dir(135));
                    label("$a$", midpoint(C--B), dir(45));
                    label("$h$", midpoint(C--D), dir(0));
                    label("$c$", M);
                \end{asy}
            \end{center}
        \item By using (1) and (2), show that $h \leq \frac{c}{2}$, and that the
            sign of equality only holds if and only if $a = b$. 
        
        \item Show that of all right triangles having the same length of
            hypotenuse, the isosceles right triangle maximizes the area. 
    \end{enumerate}
\end{question}
\begin{solution}
    Since the square of a real number is always non-negative,
    \[ (x - y)^2 \geq 0 \Longrightarrow x^2 + y^2 \geq 2xy, \]
    and the equality holds if and only if $x - y = 0 \Longleftrightarrow x = y$.

    In $\triangle BDC$ and $\triangle BCA$, we have $\angle BDC = \angle BCA =
    90^\circ$ and $\angle B$ as a common angle, so these two triangles are
    similar. Therefore, 
    \[ \frac{BC}{CD} = \frac{BA}{AC} \Longrightarrow \frac{a}{h} = \frac{c}{b}
    \Longrightarrow h = \frac{ab}{c}. \]
    Now by the first part and the Pythagoras's theorem,
    \[ h = \frac{1}{c} \cdot ab \leq \frac{1}{c} \cdot \frac{a^2 + b^2}{2} =
    \frac{c^2}{2c} = \frac{c}{2}. \]
    Finally, as the equality holds if and only if $a = b$, we see that the $h$
    is maximum when the right triangle is isosceles. Since the area of
    $\triangle ABC$ is $hc/2$ and $c$ is constant, it is maximized when $h$ is
    maximized, so we're done.
\end{solution}

\begin{question}
    Find all functions over the reals such that $f(x) + 2f(1 - x) = x(1 - x)$.
\end{question}
\begin{solution}
    Substituing $x$ with $1 - x$ in the equation, we have
    \[ f(1 - x) + 2f(x) = (1 - x)x. \]
    Multiplying this equation by 2 and subtracting the original equation gives
    us
    \[ 3f(x) = x(1 - x) \Longrightarrow f(x) = \frac{x(1 - x)}{3}. \]
    It is easy to check that this function satisfies the given equation, so
    $f(x) = \frac{x(1 - x)}{3}$ is the only solution to the given equation.
\end{solution}

\begin{question}
    In the given figure, $ABCD$ is a rectangle. $P$ and $Q$ are points on $BC$
    such that $AB = BP = PQ = QC$. Find two similar but not congruent triangles
    and prove their similarity. 
\end{question}
\begin{solution}
    Let $x = AB = BP = PQ = QC$. We claim that $\triangle PAQ \sim \triangle
    PCA$. First by Pythagoras's theorem, we have 
    \[ PA^2 = 2x^2 = x(2x) = PQ \cdot PC \Longrightarrow \frac{PA}{PQ} =
    \frac{PC}{PA}. \]
    \begin{center}
        \begin{asy}
            import olympiad;
            size(7cm);
            defaultpen(fontsize(11pt));
            pen mydash = linetype(new real[] {5,5});
            pair A = (0, 1);
            pair B = (0, 0);
            pair P = (1, 0);
            pair Q = (2, 0);
            pair C = (3, 0);
            pair D = (3, 1);
            draw(A--B--C--D--cycle, black+1);
            draw(A--P);
            draw(A--Q);
            draw(A--C);
            add(pathticks(A--B, s=4));
            add(pathticks(B--P, s=4));
            add(pathticks(P--Q, s=4));
            add(pathticks(Q--C, s=4));
            dot("$A$", A, dir(135));
            dot("$B$", B, dir(225));
            dot("$C$", C, dir(315));
            dot("$D$", D, dir(45));
            dot("$P$", P, dir(270));
            dot("$Q$", Q, dir(270));
        \end{asy}
    \end{center}
    We also have $\angle APQ = \angle CPA$, so $\triangle PAQ \sim \triangle
    PCA$.
\end{solution}

\begin{question}
    Let $f(x) = ax^2 + bx + c$, $a \ne 0$. 
    \begin{enumerate}
        \item Fill in the following steps ($\square$) to show that $f(x) = a(x
            + \frac{b}{2a})^2 - \frac{b^2 - 4ac}{2a}$. 
            \begin{align*}
                f(x) &= a(x^2 + \square x) + c\\
                &= a\left(x^2 + \square x + \square - \left(\frac{b}{2a}\right)^2\right) + c\\
                &= a(x^2 + \square x + \square)- \frac{b^2}{\square} + c\\
                &= a(x + \square)^2 - \frac{b^2}{\square} + c\\
                &= a\left(x + \frac{b}{2a}\right)^2 - \frac{b^2 - 4ac}{4a}
            \end{align*}
        
        \item Hence show that if $a > 0$ and $b^2 - 4ac < 0$, then $f(x) > 0$
            for all $x \in \mathbb{R}$. 
        
        \item Show that if $a < 0$ and $b^2 - 4ac < 0$, then $f(x) < 0$ for all
            $x \in \mathbb{R}$.
        
        \item If $b^2 - 4ac > 0$, find the roots of the equation $f(x) = 0$ in
            terms of $a$, $b$ and $c$. 
    \end{enumerate}
\end{question}
\begin{solution}
    This is just completing the square.
    \begin{align*}
        f(x) &= a \left( x^2 + \frac{b}{a} \cdot x \right) + c\\
        &= a \left( x^2 + \frac{b}{a} \cdot x + \left( \frac{b}{2a} \right)^2 - \left( \frac{b}{2a} \right)^2 \right) + c\\
        &= a \left( x^2 + \frac{b}{a} \cdot x + \left( \frac{b}{2a} \right)^2 \right) - \frac{b^2}{4a} + c\\
        &= a \left( x + \frac{b}{2a} \right)^2 - \frac{b^2}{4a} + c\\
        &= a \left( x + \frac{b}{2a} \right)^2 - \frac{b^2 - 4ac}{4a}.
    \end{align*}
    If $a > 0$ and $b^2 - 4ac < 0$, we must have $\frac{b^2 - 4ac}{4a} < 0$.
    Since squares are non-negative,
    \[ f(x) = a \left( x + \frac{b}{2a} \right)^2 - \frac{b^2 - 4ac}{4a} \geq
    -\frac{b^2 - 4ac}{4a} > 0. \]
    Now if $a < 0$ and $b^2 - 4ac < 0$, we must have $\frac{b^2 - 4ac}{4a} >
    0$. Therefore,
    \[ f(x) = a \left( x + \frac{b}{2a} \right)^2 - \frac{b^2 - 4ac}{4a} \leq
    -\frac{b^2 - 4ac}{4a} < 0. \]
    Finally, let's find the roots of $f(x) = 0$.
    \begin{align*}
        && f(x) &= 0\\
        &\Longleftrightarrow& a \left( x + \frac{b}{2a} \right)^2 - \frac{b^2 - 4ac}{4a} &= 0\\
        &\Longleftrightarrow& \left( x + \frac{b}{2a} \right)^2 &= \frac{b^2 - 4ac}{4a^2}\\
        &\Longleftrightarrow& x + \frac{b}{2a} &= \pm \frac{\sqrt{b^2 - 4ac}}{2a}\\
        &\Longleftrightarrow& x = &= \frac{-b \pm \sqrt{b^2 - 4ac}}{2a}. 
    \end{align*}
    Since $b^2 - 4ac > 0$, the square root will evaluate to a positive real
    number, and we will have two distinct real roots.
\end{solution}

\begin{question}
    In the given figure, $DC = PB$ and $DC \parallel PB$. $M$ and $N$ are
    midpoints of $AC$ and $BD$ respectively. 
    \begin{enumerate}
        \item Prove that the points $P$, $N$, $C$ are collinear. 
        
        \item Prove that $AP \parallel MN$ and $AP = 2MN$. 
    \end{enumerate}
\end{question}
\begin{solution}
    As $DC = PB$ and $DC \parallel PB$, it follows that $DCBP$ is a
    parallelogram. Therefore, the diagonals bisect each other, and hence the
    midpoint $N$ of segment $BD$ must also be the midpoint of segment $PC$,
    which means that $P$, $N$, $C$ are collinear. 
    \begin{center}
        \begin{asy}
            import olympiad;
            size(7cm);
            defaultpen(fontsize(11pt));
            pen mydash = linetype(new real[] {5,5});
            pair B = (2, -.5);
            pair D = -B;
            pair C = (.5, 1.5);
            pair P = -C;
            pair N = (0, 0);
            pair A = (-2.5, -.7);
            pair M = midpoint(A--C);
            draw(D--C--B--P--cycle, black+1);
            draw(D--A);
            draw(A--P, black+1);
            draw(A--C);
            draw(D--B--A);
            draw(P--C);
            draw(N--M, black+1);
            dot("$A$", A, dir(180));
            dot("$B$", B, dir(0));
            dot("$C$", C, dir(90));
            dot("$D$", D, dir(135));
            dot("$P$", P, dir(270));
            dot("$M$", M, dir(135));
            dot("$N$", N, dir(45));
        \end{asy}
    \end{center}
    Now $M$ is also the midpoint of $AC$, so $MN \parallel AP$ and 
    \[ \frac{MN}{AP} = \frac{CN}{CP} = \frac{1}{2} \Longrightarrow AP = 2MN.
    \qedhere \]
\end{solution}

\begin{question}
    In the given figure, $\angle BAC = 60^\circ$, $AB = 24$ cm, $BD \perp AC$
    and $DC = 3$ cm. Find the diameter of the circle. 
\end{question}
\begin{center}
    \begin{asy}
        import olympiad;
        size(7cm);
        defaultpen(fontsize(11pt));
        pen mydash = linetype(new real[] {5,5});
        real s = 30;
        pair A = dir(170);
        pair B = dir(s);
        pair C = dir(s-120);
        pair D = foot(B, C, A);
        pair O = (0, 0);
        pair E = -C;
        draw(A--B--C--cycle, black+1);
        draw(B--D);
        draw(C--E);
        draw(E--B);
        draw(circumcircle(A, B, C));
        draw(rightanglemark(B, D, A, 2.5));
        draw(rightanglemark(C, B, E, 2.5));
        dot("$A$", A, dir(A));
        dot("$B$", B, dir(B));
        dot("$C$", C, dir(C));
        dot("$D$", D, dir(225));
        dot("$E$", E, dir(E));
        dot("$O$", O, dir(180));
        label("24", midpoint(A--B), dir(100));
        label("3", midpoint(D--C), dir(225));
    \end{asy}
\end{center}
\begin{solution}
    Let $O$ be the center of the circle and let line $CO$ meet the circle again
    at $E$. Since $\triangle BAD$ is a 30-60 right triangle, we have $AD = 12$.
    Therefore, $AC = AD + DC = 15$. Now by law of cosines in $\triangle ABC$,
    \[ BC^2 = AC^2 + AB^2 - 2AB \cdot AC \cdot \cos 60^\circ = 15^2 + 24^2 - 15
    \cdot 24 = 441 \Longrightarrow BC = 21. \]
    Now notice that $\angle CEB = \angle CAB = 60^\circ$ and $\angle CBE =
    90^\circ$. Therefore,
    \[ CE = \frac{BC}{\sin 60^\circ} = \frac{42}{\sqrt{3}} = 14\sqrt{3}, \]
    and hence the diameter of the circle is $14 \sqrt{3} \text{ cm}$.
\end{solution}

\begin{question}
    In an AP, the $k$th term is 11. The sum of the first $k$ terms is 26. The
    sum of the next $k$ terms is 74. Find the first term and the common
    difference. 
\end{question}
\begin{solution}
    Let $u_1$ and $d$ be the first term and common difference of the sequence.
    It is easy to see that the sum to $2k$ terms is $26 + 74 = 100$. Therefore,
    we have the following three equations:
    \begin{align*}
        26 &= \frac{k(u_1 + u_k)}{2},\\
        100 &= k(u_1 + u_{2k}),\\
        11 &= u_k.
    \end{align*}
    Since $u_{2k}$ is obtained by adding $d$ to $u_k$ for $k$ more times,
    $u_{2k} = u_k + kd$. From the first equation, we also have $k(u_1 + u_k) =
    52$. Substituting all of this into the second equation gives
    \[ 100 = k(u_1 + u_{2k}) = k(u_1 + u_k) + k^2d = 52 + k^2d \Longrightarrow
    k^2d = 48. \]
    Now from the second equation, $u_k = u_1 + (k - 1)d$, so $u_1 = u_k - (k -
    1)d$. Substituting again into the second equation,
    \begin{align*}
        100 &= k(u_k - (k - 1)d + u_k + kd)\\
        &= k(2u_k + d)\\
        &= 22k + kd\\
        100k &= 22k^2 + k^2d\\
        100k &= 22k^2 + 48\\
        22k^2 - 100k + 48 &= 0\\
        11k^2 - 50k + 24 &= 0\\
        (11k - 6)(k - 4) &= 0
    \end{align*}
    Since $k$ is a positive integer, $k$ must be 4 and so $d = 3$. Finally, 
    \[ u_1 = u_k - (k - 1)d = 11 - 3 \cdot 3 = 2. \qedhere \]
\end{solution}
\begin{remark}
    This problem is basically the same as \hyperref[sol: 2018 Regional Round
    P7]{2018 Regional Round Problem 7}. 
\end{remark}

\begin{question}
    What is the radius of the inscribed circle of a 3-4-5 right triangle? 
\end{question}
\begin{solution}
    We will first derive a formula to find the inradius of a general triangle.
    Suppose that we have $\triangle ABC$ with incenter $I$. Let its inradius be
    $r$, then the area of $\triangle IBC$ is 
    \[ [\triangle IBC] = \frac{r \cdot BC}{2}. \]
    Similarly, we can also compute the areas of $\triangle ICA$ and $\triangle
    IAB$. Therefore summing all of them gives
    \[ [\triangle ABC] = [\triangle IBC] + [\triangle ICA] + [\triangle IAB] =
    \frac{r(AB + BC + CA)}{2} \Longrightarrow r = \frac{2[\triangle ABC]}{AB +
    BC + CA}. \]
    \begin{center}
        \begin{asy}
            import olympiad;
            size(7cm);
            defaultpen(fontsize(11pt));
            pen mydash = linetype(new real[] {5,5});
            pair A = dir(120);
            pair B = dir(210);
            pair C = dir(330);
            pair I = incenter(A, B, C);
            pair D = foot(I, B, C);
            pair E = foot(I, C, A);
            pair F = foot(I, A, B);
            draw(A--B--C--cycle, black+1);
            draw(I--D, mydash);
            draw(I--E, mydash);
            draw(I--F, mydash);
            draw(I--A);
            draw(I--B);
            draw(I--C);
            draw(circle(I, abs(I-D)));
            dot("$A$", A, dir(A));
            dot("$B$", B, dir(B));
            dot("$C$", C, dir(C));
            dot("$D$", D, dir(270));
            dot("$E$", E, dir(45));
            dot("$F$", F, dir(135));
            dot("$I$", I, dir(70));
            label("$r$", midpoint(I--D), dir(0));
        \end{asy}
    \end{center}
    Now let's apply this formula to our 3-4-5 right triangle. Its area is
    10, so its inradius is
    \[ r = \frac{20}{3 + 4 + 5} = \frac{20}{12} = \frac{5}{3}. \qedhere \]
\end{solution}

\begin{question}
    A bag contains 3 red balls and 2 green balls. Balls are drawn at random,
    one at a time but not replaced, until all 3 of red balls are drawn or until
    both green balls are drawn. What is the probability that the 3 reds are
    drawn? 
\end{question}
\begin{solution}
    Suppose that we continue drawing after we are supposed to stop until there
    are no balls left. Note that 3 reds will be drawn before both greens are
    drawn if and only if the fifth ball is green. Therefore, we just need to
    calculate the probability that the fifth ball is green. Note that by
    symmetry, the number of draws where the fifth ball is green is exactly the
    same as the number of draws where the first ball is green. Therefore, the
    probability is just that of drawing a green ball at the start, which is
    $2/5$.
\end{solution}

\begin{question}
    If 75! is divisible by $5^n$, find the maximum value of $n$. 
\end{question}
\begin{solution}
    Since $75 = 5 \cdot 15$, there are 15 numbers less than or equal to $75$
    which are divisible by 5. Out of those 15 numbers, there are 3 numbers
    which are divisible by $5^2 = 25$. Therefore, the total power of 5 in the
    prime factorization of $75!$ is $15 + 3 = 18$. Hence $n = 18$.
\end{solution}
\begin{remark}
    This problem uses the same idea as \hyperref[sol: 2018 Regional Round P10]{2018 Regional Round Problem 10}. 
\end{remark}

\begin{question}
    If $L$, $M$, $N$ are the midpoints of the sides of $\triangle ABC$, and $P$
    is the foot of perpendicular from $A$ to $BC$, prove that $L$, $P$, $M$,
    $N$ are concyclic.
\end{question}
\begin{solution}
    First observe that $ML \parallel AC$ and $MN \parallel AB$ due to the
    midpoints. Therefore, $ALMN$ is a parallelogram, and so $\angle LMN =
    \angle LAN$. Now since $L$ is the midpoint of the hypotenuse in right
    triangle $ABP$, we see that $LA = LP$. Similarly, $NA = NP$. Therefore,
    $\triangle LAN \cong \triangle LPN$. This implies that $\angle LMN =
    \angle LAN = \angle LPN$ so $L$, $M$, $N$, $P$ are concyclic as desired.
\end{solution}
\begin{center}
    \begin{asy}
        import olympiad;
        size(7cm);
        defaultpen(fontsize(11pt));
        pen mydash = linetype(new real[] {5,5});
        pair A = dir(120);
        pair B = dir(210);
        pair C = dir(330);
        pair M = midpoint(B--C);
        pair N = midpoint(C--A);
        pair L = midpoint(A--B);
        pair P = foot(A, B, C);
        draw(A--B--C--cycle, black+1);
        draw(A--P);
        draw(L--P--N);
        draw(L--M--N);
        draw(L--N);
        draw(circumcircle(L, M, N), mydash);
        draw(rightanglemark(A, P, C, 2.5));
        dot("$A$", A, dir(A));
        dot("$B$", B, dir(225));
        dot("$C$", C, dir(315));
        dot("$M$", M, dir(270));
        dot("$N$", N, dir(45));
        dot("$L$", L, dir(135));
        dot("$P$", P, dir(270));
    \end{asy}
\end{center}

\begin{question}
    In a GP, the $k$th term is 864. The sum of the first $k$ terms is 2080. The
    sum of the first $2k$ terms is 12610. Find the first term and the common
    ratio.
\end{question}
\begin{solution}
    Let the first term of the GP be $a$ and common ratio be $r$. We can view
    $S_{2k}$ in the following way.
    \begin{align*}
        S_{2k} &= a + ar + ar^2 + \cdots + ar^{k - 1} + ar^{k} + \cdots + ar^{2k - 1}\\
        &= (a + ar + ar^2 + \cdots + ar^{k - 1}) + r^k(a + ar + ar^2 + \cdots + ar^{k - 1})\\
        &= S_k + r^k S_k\\
        &= S_k (1 + r^k)\\
        1 + r^k &= \frac{97}{16}\\
        r^k &= \frac{81}{16}.
    \end{align*} 
    We will now use the condition $u_k = ar^{k - 1} = 864$.
    \begin{align*}
        S_k &= \frac{a - ar^k}{1 - r}\\
        (1 - r)2080 &= r \left( \frac{a}{r} - ar^{k - 1} \right)\\
        &= r \left( \frac{ar^{k - 1}}{r^k} - ar^{k - 1} \right)\\
        &= r \left( \frac{864 \cdot 16}{81} - 864 \right)\\
        (r - 1)2080 &= \frac{56160r}{81}\\
        r - 1 &= \frac{r}{3}\\
        r &= \frac{3}{2}.
    \end{align*}
    Therefore, $k = 4$. Finally,
    \[ a \left( \frac{3}{2} \right)^3 = 864 \Longrightarrow a = 256. \qedhere \]
\end{solution}

\begin{question}
    Prove that $\binom{n}{m}\binom{m}{k} = \binom{n}{k}\binom{n - k}{m - k}$ if
    all variables are integers and $n \geq m \geq k \geq 0$.
\end{question}
\begin{solution}
    Suppose that there are $n - m$ blue balls, $m - k$ green balls, and $k$ red
    balls. The total number balls is $n - m + m - k + k = n$. We will count the
    number of ways to arrange these balls in two different ways. First suppose
    that we place the blue balls first. There are $\binom{n}{n - m}$ ways to
    choose $n - m$ positions out of $n$ positions. After this, there will be
    $m$ places left for red and green balls. The number of ways to choose $k$
    places from these $m$ places for red balls is $\binom{m}{k}$. The green
    balls are placed in the places left, so there is no choice. Therefore,
    the total number of arrangements is
    \[ \binom{n}{n - m}\binom{m}{k} = \binom{n}{m}\binom{m}{k}. \]
    Now suppose that we place the red balls first. There are $\binom{n}{k}$
    ways to choose $k$ positions out of $n$ positions. Then there will be $n -
    k$ positions remaining for blue and green balls. Out of those $n - k$
    positions, there are $\binom{n - k}{m - k}$ ways to choose $m - k$
    positions for the green balls. Again, the remaining blue balls are placed in
    the places left, so there is no choice. Therefore, the total number of
    arrangements counted this way is
    \[ \binom{n}{k}\binom{n - k}{m - k}. \]
    Since the total number of arrangements is the same, we have our desired
    identity.
\end{solution}

\begin{question}
    \begin{enumerate}
        \item Prove that $n \leq -k^2 + nk + k \leq \frac{(n + 1)^2}{4}$ for $1
            \leq k \leq n$. 
        
        \item Consider $k(n + 1 - k) = -k^2 + nk + k$ and by using the
            inequalities in (1), prove that 
            \[n^n \leq (n!)^2 \leq \frac{(n + 1)^{2n}}{4^n}.\]
        
        \item Hence prove that 
            \[n^{\frac{n}{2}} \leq n! \leq \frac{(n + 1)^n}{2^n}.\]
    \end{enumerate}
\end{question}
\begin{solution}
    Let's show the first inequality.
    \begin{align*}
        && k &\leq n &&\\
        &\Longleftrightarrow& k(k - 1) &\leq n(k - 1) &&\\
        &\Longleftrightarrow& k^2 - k &\leq nk - n && \\
        &\Longleftrightarrow& n &\leq -k^2 + nk + k. &&
    \end{align*}
    The second inequality follows from the \hyperref[thm: amgm]{AM-GM inequality}. 
    \[ \frac{(n + 1)^2}{4} = \left( \frac{(n + 1 - k) + k}{2} \right)^2 \geq
    k(n + 1 - k) = -k^2 + nk + k. \]
    Now consider the following $n$ inequalities, obtained when $k$ ranges from
    1 to $n$.
    \begin{align*}
        n &\leq 1 \cdot n \leq \frac{(n + 1)^2}{4}\\
        n &\leq 2 \cdot (n - 1) \leq \frac{(n + 1)^2}{4}\\
        &\vdotswithin{\leq}\\
        n &\leq n \cdot 1 \leq \frac{(n + 1)^2}{4}.
    \end{align*}
    Multiplying all of them gives
    \[ n^n \leq (n!)^2 \leq \frac{(n + 1)^{2n}}{4^n}. \]
    Taking the square roots of every term, we finally have
    \[ n^{\frac{n}{2}} \leq n! \leq \frac{(n + 1)^n}{2^n}. \qedhere \]
\end{solution}
