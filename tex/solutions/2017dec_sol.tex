\solutionheader{2017 National Round}
\begin{question}
    $x = \overline{ABCDE}$ is a five digit number. If $(A + C + E) - (B + D) =
    11k$ where $k = -1, 0, 1, 2$, prove that $x$ is divisible by 11.
\end{question}
\begin{solution}
    Notice that $x$ can be rewritten as
    \begin{align*}
        x
        &= 10000A + 1000B + 100C + 10D + E\\
        &= 9999A + 1001B + 99C + 11D + (A - B + C - D + E)\\
        &= 11(909A + 91B + 9C + D + k).
    \end{align*}
    and so $x$ is divisible by 11.
\end{solution}

\begin{question}
    A palindrome is a number that remains the same when its digits are
    reversed. For example, 252 is a three-digit palindrome and 3663 is a
    four-digit palindrome. If the numbers $x - 22$ and $x$ are three-digit and
    four-digit palindromes, respectively, find the value of $x$.
\end{question}
\begin{solution}
    Let's look at the sequence of all palindromes in ascending order, near
    1000.
    \[ \ldots, 969, 979, 989, 999, 1001, 1111, 1221, \ldots \]
    Notice that $x$ cannot be greater than 1001 or otherwise $x - 22$ will not
    be a three digit number. Therefore, $x$ can only be 1001. It is also easy
    to see that $x - 22 = 979$ is also a three-digit palindrome, so $x = 1001$
    must be the only solution.
\end{solution}

\begin{question}
    \begin{enumerate}
        \item To find the exact value of $\sqrt{4 + 2\sqrt{3}}$, let $\sqrt{4 +
            2\sqrt{3}} = a + b\sqrt{3}$, where $a$ and $b$ are integers and $a
            > b\sqrt{3} > 0$, and compute the exact values of $a$ and $b$.

        \item Length of a side of an equilateral triangle $ABC$ is 2. If $AD
            \perp BC$, and the angle bisector of $\angle BAD$ meets $BC$ at
            $E$, show that $\angle AED = 75^\circ$. Using the angle bisector
            theorem, find the exact length of $ED$. Using the diagram, compute
            exact values of $\sin 75^\circ$ and $\cos 75^\circ$.
    \end{enumerate}
\end{question}
\begin{solution}
    \begin{enumerate}
        \item Let $\sqrt{4 + 2\sqrt{3}} = a + b\sqrt{3}$. Then
            \[ 4 + 2\sqrt{3} = a^2 + 3b^2 + 2ab\sqrt{3}. \]
            Therefore, we will be done if we can find integers such that $a^2 +
            3b^2 = 4$ and $ab = 1$. Substituting $b = \frac{1}{a}$ into the
            first equation,
            \[ a^2 - 4a^2 + 3 = 0 \Longrightarrow (a^2 - 1)(a^2 - 3) = 0. \]
            and hence we have the following four solutions pairs $(a, b)$: $(1,
            1)$, $(-1, -1)$, $(\sqrt{3}, \frac{1}{\sqrt{3}})$, and $(-\sqrt{3},
            -\frac{1}{\sqrt{3}})$. Out of all these four pairs, we see that
            only one pair $(1, 1)$ satisfies $a > b\sqrt{3} > 0$. Hence $a = 1$
            and $b = 1$.
    \end{enumerate}
    \begin{center}
        \begin{asy}
            import olympiad;
            size(6cm);
            defaultpen(fontsize(11pt));
            pen mydash = linetype(new real[] {5,5});
            pair A = dir(90);
            pair B = dir(210);
            pair C = dir(330);
            pair D = midpoint(B--C);
            pair E = (sqrt(3)/(2+sqrt(3)))*B + (2/(2+sqrt(3)))*D;
            draw(A--B--C--cycle, black+1);
            draw(A--D);
            draw(A--E);
            draw(rightanglemark(A, D, C, 2.5));
            dot("$A$", A, dir(90));
            dot("$B$", B, dir(225));
            dot("$C$", C, dir(315));
            dot("$D$", D, dir(270));
            dot("$E$", E, dir(270));
            label("2", midpoint(A--B), dir(150));
            label("2", midpoint(A--C), dir(30));
        \end{asy}
    \end{center}
    \begin{enumerate}[resume]
        \item Since $\triangle ABC$ is equilateral, $D$ must be the midpoint of
            $BC$. By Pythagoras's theorem,
            \[ AD = \sqrt{AB^2 - BD^2} = \sqrt{4 - 1} = \sqrt{3}. \]
            Therefore, by the angle bisector theorem,
            \[ \frac{ED}{DA} = \frac{EB}{BA} = \frac{ED + EB}{DA + BA} =
            \frac{1}{\sqrt{3} + 2} \Longrightarrow ED =
            \frac{\sqrt{3}}{\sqrt{3} + 2}. \]
            Now we can calculate the length of $AE$ as follows:
            \[ AE = \sqrt{DE^2 + DA^2} = \sqrt{\left( \frac{\sqrt{3}}{\sqrt{3}
            + 2} \right)^2 + 3} = \sqrt{\frac{12(2 + \sqrt{3})}{(2 +
            \sqrt{3})^2}} = \sqrt{\frac{12}{2 + \sqrt{3}}}. \]
            The right hand side can be written as
            \[ AE = \sqrt{\frac{12}{2 + \sqrt{3}}} = \sqrt{\frac{24}{4 +
            2\sqrt{3}}} = \frac{\sqrt{24}}{1 + \sqrt{3}}. \]
            Therefore, finally,
            \[ \sin 75^\circ = \frac{AD}{AE} = \frac{1 + \sqrt{3}}{2 \sqrt{2}}, \]
            and 
            \[ \cos 75^\circ = \frac{ED}{AE} = \frac{\sqrt{3}(1 +
            \sqrt{3})}{\sqrt{24}(2 + \sqrt{3})} = \frac{1 +
            \sqrt{3}}{\sqrt{2}(4 + 2\sqrt{3})} = \frac{1 + \sqrt{3}}{\sqrt{2}(1 +
            \sqrt{3})^2} = \frac{\sqrt{3} - 1}{2\sqrt{2}}. \qedhere \]
    \end{enumerate}
\end{solution}
\begin{remark}
    It is way easier to calculate these using law of sines or angle addition
    formulas.
\end{remark}

\begin{question}
    Find the $n$th term of a harmonic progression whose first two terms are $a$
    and $b$.
\end{question}
\begin{solution}
    Suppose that $a, b, u_3, u_4, \ldots$ is a harmonic progression. Then
    $\frac{1}{a}, \frac{1}{b}, \frac{1}{u_3}, \frac{1}{u_4}, \ldots$ is an
    arithmetic progression. Since the first term of this arithmetic progression
    is $\frac{1}{a}$ and the common difference is $\frac{1}{b} - \frac{1}{a}$,
    the $n$th term of this arithmetic progression is
    \[ \frac{1}{u_n} = \frac{1}{a} + (n - 1)\left( \frac{1}{b} - \frac{1}{a}
    \right) = \frac{(n - 1)(a - b) + b}{ab}. \]
    Hence the $n$th term of the harmonic progression must be
    \[ u_n = \frac{ab}{(n - 1)(a - b) + b}.\qedhere \]
\end{solution}

\begin{question}
    Find the relationship between $a$, $b$ and $c$ if the system 
    \begin{align*}
        x + y &= a\\
        x^2 + y^2 &= b\\
        x^3 + y^3 &= c
    \end{align*}
    has solutions.
\end{question}
\begin{solution}
    By the sum of cubes identity,
    \[ c = x^3 + y^3 = (x + y)(x^2 - xy + y^2) = a(b - xy). \]
    Therefore, we just need to find the value of $xy$ in terms of $a$, $b$ and
    $c$. This can be done as follows:
    \[ xy = \frac{(x + y)^2 - x^2 - y^2}{2} = \frac{a^2 - b}{2}. \]
    Hence the relationship between $a$, $b$ and $c$ is
    \[ c = a \left( b + \frac{b - a^2}{2} \right) = \frac{a(3b - a^2)}{2}
    \Longrightarrow a^3 - 3ab + 2c = 0. \qedhere \]
\end{solution}

\begin{question}
    A function $f$ is defined on the positive integers, $f(1) = 1009$ and 
    \[f(1) + f(2) + \cdots + f(n) = n^2 f(n).\]
    \begin{enumerate}
        \item By expressing $f(1) + f(2) + \cdots + f(n - 1)$ in two ways, find
            $\frac{f(n)}{f(n - 1)}$. 
        
        \item By using the result in part(a), find the formula for $f(n)$.
            Calculate $f(2018)$. 
    \end{enumerate}
\end{question}
\begin{solution}
    Observe that
    \begin{align*}
    n^2f(n) &= f(1) + f(2) + \cdots + f(n - 1) + f(n)\\
    &= (n - 1)^2f(n - 1) + f(n)\\
    (n^2 - 1)f(n) &= (n - 1)^2f(n - 1)\\
    \frac{f(n)}{f(n - 1)} &= \frac{n - 1}{n + 1}.
    \end{align*}
    Therefore,
    \[ f(n) = f(1) \cdot \frac{f(2)}{f(1)} \cdot \frac{f(3)}{f(2)} \cdot
    \frac{f(4)}{f(3)}\cdot\cdots\cdot\frac{f(n)}{f(n - 1)} = 1009 \cdot
    \frac{1}{3} \cdot \frac{2}{4} \cdot \frac{3}{5} \cdot\cdots\cdot \frac{n - 1}{n
    + 1} = \frac{2018}{n(n + 1)}.\]
    Finally, $f(2018) = \frac{2018}{2018 \cdot 2019} = \frac{1}{2019}$.
\end{solution}

\begin{question}
    \begin{enumerate}
        \item $f(x) = \frac{1}{(2x - 1)(2x - 3)}$ can be expressed as $f(x) =
            \frac{A}{2x - 1}+\frac{B}{2x - 3}$, where $A$ and $B$ are
            constants. Find the values of $A$ and $B$. 
        
        \item If $f(3) + f(4) + f(5) + \cdots + f(n) = c - g(n)$, where $c$ is
            a constant and $g(n)$ is a function, determine $c$ and $g(n)$. 
    \end{enumerate}
\end{question}
\begin{solution}
    Since 
    \[ \frac{1}{(2x - 1)(2x - 3)} = \frac{A}{2x - 1} + \frac{B}{2x - 3} =
    \frac{A(2x - 3) + B(2x - 1)}{(2x - 1)(2x - 3)}, \]
    Clearing the denominators gives\footnote{We can do this since the domain of
    $f$ does not include $\frac{1}{2}$ or $\frac{3}{3}$.}
    \[ A(2x - 3) + B(2x - 1) = 1 \Longrightarrow (2A + 2B)x - 3A - B = 1, \]
    By equating the coefficients, we see that $2A + 2B = 0$ and $-3A - B = 1$.
    Solving these two equations give $A = -\frac{1}{2}$ and $B = \frac{1}{2}$.
    Therefore, $f(x)$ can be represented as
    \[ f(x) = \frac{1}{2}\left( \frac{1}{2x - 3} - \frac{1}{2x - 1} \right). \]
    Consequently, the sum in the second part is
    \begin{align*}
        f(3) + f(4) + \cdots + f(n)
        &= \frac{1}{2}\left( \frac{1}{3} - \frac{1}{5} + \frac{1}{5} - \frac{1}{7} + \cdots + \frac{1}{2n - 3} - \frac{1}{2n - 1} \right)\\
        &= \frac{1}{2}\left( \frac{1}{3} - \frac{1}{2n - 1} \right)\\
        &= \frac{1}{6} - \frac{1}{4n - 2}.
    \end{align*}
    Therefore, $c = \frac{1}{6}$ and $g(n) = \frac{1}{4n - 2}$.
\end{solution}

\begin{question}
    Find the number of ways in which 5 men, 3 women and 2 children can sit at a
    round table, if 
    \begin{enumerate}
        \item there are no restrictions,
        
        \item each child is seated between 2 women.
    \end{enumerate}
\end{question}
\begin{solution}
    \begin{enumerate}
        \item The number of circular permutations of $n$ different people is
            $(n - 1)!$. Since there are $5 + 3 + 2 = 10$ people in total, the
            number of circular permutations is $9! = 362880$.

        \item Since each child is seated between 2 women, this means that the 3
            women and 2 children form a block like so:
            \[ W - C - W - C - W \]
            There are $(6 - 1)! = 120$ ways to permute the block and 5 men around
            the table. Inside the block, there are $3! = 6$ ways to permute the
            women, and 2 ways to permute the children. Therefore, the total number
            of permutations is $120 \cdot 6 \cdot 2 = 1440$. \qedhere
    \end{enumerate}
\end{solution}

\begin{question}
    $ABCD$ is a rectangle with $AB = x$, $AD = y$ and $y > x$, and $AXYZ$ is a
    square. If the area of $AXYZ$ is the same as the area of $ABCD$, show that
    $\sqrt{xy} - x\leq BX\leq \sqrt{xy} + x$ and $y - \sqrt{xy} \leq DZ\leq y +
    \sqrt{xy}$.
\end{question}
\begin{solution}
    Let the side length of the square be $z$. Then since $ABCD$ and $AXYZ$ have
    the same area, we have $z = \sqrt{xy}$. By the triangle inequality in
    $\triangle AXB$,
    \[ AX \leq AB + BX \Longrightarrow \sqrt{xy} - x \leq BX
    \quad\text{and}\quad BX \leq BA + AX = x + \sqrt{xy} \]
    which gives the first inequality. We can obtain the second inequality
    similarly by applying triangle inequality on $\triangle AZD$.
\end{solution}
\begin{center}
    \begin{asy}
        import olympiad;
        size(7cm);
        defaultpen(fontsize(11pt));
        pen mydash = linetype(new real[] {5,5});
        pair A = (-1.5, 1);
        pair B = (-1.5, -1);
        pair C = (1.5, -1);
        pair D = (1.5, 1);
        pair Z = (-.5, .5);
        pair X = rotate(90, A)*Z;
        pair Y = rotate(-90, Z)*A;
        draw(A--B--C--D--cycle);
        draw(A--X--Y--Z--cycle);
        draw(A--X--B--cycle, black+1);
        draw(A--Z--D--cycle, black+1);
        dot("$A$", A, dir(135));
        dot("$B$", B, dir(225));
        dot("$C$", C, dir(315));
        dot("$D$", D, dir(45));
        dot("$X$", X, dir(90));
        dot("$Y$", Y, dir(0));
        dot("$Z$", Z, dir(270));
        label("$x$", midpoint(A--B), dir(180));
        label("$y$", midpoint(B--C), dir(270));
        label("$\sqrt{xy}$", midpoint(A--X), dir(135));
    \end{asy}
\end{center}

\begin{question}
    The average of the numbers $1, 2, 3, \ldots, 99$ and $x$ is $100x$. Find
    the value of $x$. 
\end{question}
\begin{solution}
    By the AP summation formula,
    \[ 1 + 2 + \cdots + 99 = \frac{99 \cdot 100}{2} = 4950. \]
    Therefore,
    \[ \frac{4950 + x}{100} = 100x \Longrightarrow x = \frac{4950}{9999} =
    \frac{50}{101}. \qedhere \]
\end{solution}

\begin{question}
    Each of 240 boxes in a line contains a single red marble, and for $1 \leq k
    \leq 240$, the box in the $k$th position also contains $k$ white marbles.
    Phyu Phyu begins at the first box and successively draws a single marble at
    random from each box. She stops when she first draws a red marble. Let
    $\mathbb{P}(n)$ be the probability that Phyu Phyu stops after drawing
    exactly $n$ marbles. What is the smallest value of $n$ for which
    $\mathbb{P}(n) < \frac{1}{240}$?
\end{question}
\begin{solution}
    Let's first calculate $\mathbb{P}(n)$ for general $n$. It is easy to see
    that the $k$th box contain a total of $k + 1$ marbles.
    \begin{align*}
        \mathbb{P}(n) &= \mathbb{P}(\text{First marble is white}) \cdot \mathbb{P}(\text{Second marble is white}) \cdot \cdots \cdot \mathbb{P}(\text{$n$th marble is red})\\
        &= \frac{1}{2} \cdot \frac{2}{3} \cdot \cdots \cdot \frac{n - 1}{n} \cdot \frac{1}{n + 1}\\
        &= \frac{1}{n(n + 1)}.
    \end{align*}
    Therefore,
    \[ \mathbb{P}(n) = \frac{1}{n(n + 1)} < \frac{1}{240} \Longrightarrow (n -
    15)(n + 16) > 0. \]
    This shows that $n > 15$ or $n < -16$. As $n$ is positive, the smallest
    value of $n$ is 16.
\end{solution}

\begin{question}
    For all positive integers $n$, define 
    \[f(n) = 1 - \frac{1}{2} + \frac{1}{3} - \cdots + \frac{1}{2n-1} -
    \frac{1}{2n},\]
    \[g(n) = \frac{1}{n + 1} + \frac{1}{n + 2} + \frac{1}{n + 3} + \cdots +
    \frac{1}{2n}.\]
    By using mathematical induction, prove that $f(n) = g(n)$ for all positive
    integers $n$.
\end{question}
\begin{solution}
    For the base case $n = 1$, both $f(1)$ and $g(1)$ are equal to
    $\frac{1}{2}$. Now suppose that $f(k) = g(k)$ for $n = k$. Then $f(k + 1)$
    is equal to
    \begin{align*}
        f(k + 1) &= 1 - \frac{1}{2} + \frac{1}{3} - \cdots + \frac{1}{2k - 1} - \frac{1}{2k} + \frac{1}{2k + 1} - \frac{1}{2k + 2}\\
        &= f(k) + \frac{1}{2k + 1} - \frac{1}{2k + 2}\\
        &= - \frac{1}{k + 1} + g(k) + \frac{1}{2k + 1} + \frac{1}{2k + 2}\\
        &= \frac{1}{k + 2} + \frac{1}{k + 3} + \cdots + \frac{1}{2k + 2}\\
        &= g(k + 1)
    \end{align*}
    and hence by mathematical induction, $f(n) = g(n)$ for all natural numbers
    $n$.
\end{solution}

\begin{question}
    $ABCD$ is a cyclic quadrilateral with $AB = AC$. The line $PQ$ is tangent
    to the circle at the point $C$, and is parallel to $BD$. Diagonals $BD$ and
    $AC$ intersect at $E$. If $AB = 18$ and $BC = 12$, find the length of $AE$.
\end{question}
\begin{solution}
    Since $PQ \parallel BD$, 
    \[ \angle EAB = \angle CAB = \angle PCB = \angle DBC = \angle EBC, \]
    and hence by tangent-chord theorem, segment $BC$ is tangent to $(ABE)$.
    \begin{center}
        \begin{asy}
            import olympiad;
            size(7cm);
            defaultpen(fontsize(11pt));
            pen mydash = linetype(new real[] {5,5});
            real s1 = 1;
            real s2 = 1;
            pair A = dir(90);
            pair B = dir(230);
            pair C = dir(310);
            pair D = 2*foot(B, C, (0, 0)) - B;
            pair E = extension(B, D, A, C);
            pair P = C + s1*(rotate(90, C)*(0, 0) - C);
            pair Q = C + s2*(rotate(-90, C)*(0, 0) - C);
            draw(A--B--C--cycle, black+1);
            draw(A--D--C);
            draw(B--D);
            draw(P--Q);
            draw(circle((0, 0), 1));
            draw(circumcircle(A, B, E), mydash);
            dot("$A$", A, dir(A));
            dot("$B$", B, dir(B));
            dot("$C$", C, dir(C));
            dot("$D$", D, dir(D));
            dot("$E$", E, dir(350));
            dot("$P$", P, dir(225));
            dot("$Q$", Q, dir(45));
        \end{asy}
    \end{center}
    Therefore,
    \[ CE \cdot CA = CB^2 \Longrightarrow CE = 8. \]
    Finally, $AE = AC - CE = 18 - 8 = 10$.
\end{solution}

\begin{question}
    Let $a > b > 0$. Define two sequences $a_{n}$ and $b_{n}$ as follows:
    \[a_{1} = a,\mathspace b_{1} = b,\mathspace a_{n + 1} = \frac{a_{n} +
    b_{n}}{2},\mathspace b_{n + 1} = \sqrt{a_{n}b_{n}}.\]
    \begin{enumerate}
        \item Prove that $a_{n + 1} < a_{n}$ and $b_{n + 1} > b_{n}$ for $n >
            1$. 
        
        \item Prove that $a_{n + 1} - b_{n + 1} = \frac{(a_{n} -
            b_{n})^2}{8a_{n + 2}}$. 
        
        \item If $a = 4$ and $b = 1$, find the first four terms of each
            sequence of $a_{n}$ and $b_{n}$.
    \end{enumerate}
\end{question}
\begin{solution}
    When $n = 1$, $a_1 = a > b = b_1$. Now suppose that $a_k > b_k$ for some $k
    \geq 1$. Then by the \hyperref[thm: amgm]{AM-GM inequality} we have 
    \[ a_{k + 1} = \frac{a_{k} + b_{k}}{2} < \sqrt{a_{k}b_{k}} =
    b_{k + 1}. \]
    Hence by induction, $a_n > b_n$ for all $n \in \mathbb{N}$. Therefore, for $n > 1$,
    \[ a_{n + 1} = \frac{a_n + b_n}{2} < \frac{a_n + a_n}{2} = a_n. \]
    Similarly, we can show that $b_{n + 1} > b_n$.
    Now observe that
    \begin{align*}
        a_{n + 2}(a_{n + 1} - b_{n + 1}) &= \frac{1}{2}(a_{n + 1} + b_{n + 1})(a_{n + 1} - b_{n + 1})\\
        &= \frac{1}{2}(a_{n + 1}^2 - b_{n + 1}^2)\\
        &= \frac{1}{8}(a_n^2 + b_n^2 + 2a_nb_n - 4a_nb_n)\\
        &= \frac{(a_n - b_n)^2}{8}\\
        a_{n + 1} - b_{n + 1} &= \frac{(a_n - b_n)^2}{8a_{n + 2}}
    \end{align*} 
    Finally when $a = 4$ and $b = 1$, the first four terms of $a_n$ are $4,
    \frac{5}{2}, \frac{9}{4}, \frac{9 + 4\sqrt{5}}{8}$ and those of $b_n$ are
    $1, 2, \sqrt{5}, \frac{3
    \sqrt[4]{5}}{2}$.
\end{solution}

\begin{question}
    If $\cos\alpha, \cos\beta$ and $\cos\gamma$ are the roots of the equation
    $x^3 + ax^2 + bx + c = 0$, where $\alpha$, $\beta$ and $\gamma$ are angles
    of a triangle, prove that $a^2 = 2b + 2c + 1$. 
\end{question}
\begin{solution}
    By \hyperref[thm: vieta]{Vieta's formulas},
    \begin{align*}
        -a &= \cos\alpha + \cos\beta + \cos\gamma\\
        b &= \cos\alpha \cos\beta + \cos\beta \cos\gamma + \cos\gamma \cos\alpha\\
        -c &= \cos\alpha \cos\beta \cos\gamma
    \end{align*}
    Therefore, we just need to show that
    \[ \cos^2\alpha + \cos^2\beta + \cos^2\gamma = a^2 - 2b = 1 + 2c = 1 -
    2\cos\alpha \cos\beta \cos\gamma. \]
    Remember that we have $\cos 2\theta = 2\cos^2\theta - 1$. Therefore,
    \begin{align*}
        \cos^2\alpha + \cos^2\beta + \cos^2\gamma &= \frac{1}{2}(2 + \cos 2\alpha + \cos 2\beta + 2\cos^2\gamma)\\
        &= \frac{1}{2}(2 + 2\cos(\alpha + \beta)\cos(\alpha - \beta) + 2\cos^2\gamma)\\
        &= 1 - \cos\gamma \cos(\alpha - \beta) + \cos^2 \gamma\\
        &= 1 - \cos\gamma(\cos(\alpha - \beta) - \cos \gamma)\\
        &= 1 - \cos\gamma(\cos(\alpha - \beta) + \cos(\alpha + \beta))\\
        &= 1 - 2\cos\alpha \cos\beta \cos\gamma. \qedhere
    \end{align*}
\end{solution}

\begin{question}
    The two circles $C_{1}$ and $C_{2}$ intersect at the points $A$ and $B$.
    The tangent to $C_{1}$ at $A$ intersects $C_{2}$ at $P$ and the line $PB$
    intersects $C_{1}$ at $Q$. The tangent to $C_{2}$ drawn from $Q$ intersects
    $C_{1}$ and $C_{2}$ at the points $X$ and $Y$ respectively. The points $A$
    and $Y$ lie on the different sides of $PQ$. Show that $AY$ bisects $\angle
    XAP$.
\end{question}
\begin{solution}
    In fact, we don't even need the fact that segment $AP$ is tangent to $C_1$.
    Note that
    \[ \angle XAY = \angle XAB + \angle BAY = \angle XQB + \angle BYQ = \angle YBP = \angle YAP. \qedhere \]
    \begin{center}
        \begin{asy}
            import olympiad;
            size(8cm);
            defaultpen(fontsize(11pt));
            pen mydash = linetype(new real[] {5,5});
            pair A = (0, 1);
            pair B = (0, -1);
            pair O1 = (-1, 0);
            pair O2 = (2, 0);
            pair P = 2*foot(O2, A, rotate(90, A)*O1)-A;
            pair Q = 2*foot(O1, B, P)-B;
            pair Y[] = intersectionpoints(circle(midpoint(Q--O2), .5*abs(Q-O2)), circle(O2, abs(O2-A)));
            pair Y = Y[1];
            pair X = 2*foot(O1, Q, Y)-Q;
            pair R = 2*foot(O2, A, X)-A;
            draw(A--P);
            draw(P--B);
            draw(B--Q);
            draw(Q--Y);
            draw(A--X);
            draw(A--B);
            draw(A--Y);
            draw(Y--B);
            draw(circle(O1, abs(O1-A)));
            draw(circle(O2, abs(O2-A)));
            dot("$A$", A, 2*dir(100));
            dot("$B$", B, dir(260));
            dot("$P$", P, dir(315));
            dot("$Q$", Q, dir(180));
            dot("$Y$", Y, dir(225));
            dot("$X$", X, dir(270));
        \end{asy}
    \end{center}
\end{solution}
