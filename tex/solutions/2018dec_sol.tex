\solutionheader{2018 National Round}
\begin{question}
    To construct a triangle $ABC$, only given that $AB = 10$ and $\angle ABC$
    is $30^\circ$. Find all values of $AC$ for which 
    \begin{enumerate}
        \item there are two possible triangles $ABC$.
        
        \item there is only one triangle $ABC$. 
        
        \item there is no triangle $ABC$. 
    \end{enumerate}
\end{question}
\begin{solution}
    Let $\ell$ be the ray originating from $B$ which is inclined to segment
    $AB$ at a $30^\circ$ angle. Obviously $C$ must lie on this ray. Now let $H$
    be the foot of perpendicular from $A$ onto $\ell$. Let $x = AC$ be the
    length of segment $AC$, and let $\omega(x)$ be the circle centered at $A$
    with radius $x$. It is easy to see that $AH = AB \sin 30^\circ = 5$.
    \begin{center}
        \begin{asy}
            import olympiad;
            size(7cm);
            defaultpen(fontsize(11pt));
            pen mydash = linetype(new real[] {5,5});
            real s = 1/3;
            pair A = (0, 0);
            pair B = (1, 0);
            pair C1 = 1.5*dir(30);
            pair H = foot(B, A, C1);
            pair X = B + s*dir(75);
            draw(A--C1, arrow=Arrow(size=10));
            draw(A--B);
            draw(B--H, mydash);
            draw(B--X, arrow=Arrow(size=5));
            draw(rightanglemark(A, H, B, 2));
            draw(arc(B, s, 45, 195), mydash);
            dot("$A$", A, dir(225));
            dot("$B$", B, dir(315));
            dot("$H$", H, dir(120));
            label("$x$", midpoint(B--X), dir(-15));
            label("$\omega(x)$", B + s*dir(45), dir(45));
        \end{asy}
    \end{center}
    \begin{case}{$x < AH$}
        In this case, $\omega(x)$ does not intersect $\ell$ at all, so there is
        no triangle $ABC$.
    \end{case}
    \begin{case}{$x = AH$}
        In this case, $\omega(x)$ is tangent to $\ell$ at $H$. Therefore, there
        is only one place $C$ can be, namely $H$ and hence there is only one
        triangle $ABC$.
    \end{case}
    \begin{case}{$AH < x < AB$}
        In this case, $\omega(x)$ intersects the ray $\ell$ at two points.
        Therefore, $C$ can be at any of those two points, which gives two
        possible triangles $ABC$.
    \end{case}
    \begin{case}{$AB \leq x$}
        Finally in this case, $\omega(x)$ intersects ray $\ell$ at only one
        point, so similarly to case 2 there is only one triangle $ABC$.
    \end{case}
        Therefore, the final answers are 
        \begin{enumerate}
            \item $\{ x \in \mathbb{R} \mid 5 < x < 10 \}$.
            \item $\{ x \in \mathbb{R} \mid x = 5 \text{ or } x \geq 10 \}$.
            \item $\{ x \in \mathbb{R} \mid 0 < x < 5 \}$. \qedhere
        \end{enumerate}
\end{solution}

\begin{question}
    In the given figure, $O$ is the center of the circle. $OA$ and $OC$ are
    radii with $OA = OC = 5$ units. Find the area of region $ABCD$, bounded by
    arc $ABC$, line segments $CD$ and $DA$, in terms of $\theta$, if the line
    segment $DA = 8$ units and $\angle AOC$ is $\theta^\circ$. 
\end{question}
\begin{solution}
    First, let's calculate the area of sector $AOC$. It subtends an angle of
    $\theta$ degrees at the center, so its area $A_1$ is
    \[ A_1 = \frac{\theta}{360}\cdot \pi(5)^2 = \frac{5 \pi \theta}{72}. \] 
    Therefore we just need to calculate the area of $\triangle DOC$. The length
    of $OD$ is $8 - 5 = 3$ units.
    \begin{center}
        \begin{asy}
            import olympiad;
            size(6cm);
            defaultpen(fontsize(11pt));
            pen mydash = linetype(new real[] {5,5});
            usepackage("contour", "outline");
            texpreamble("\contourlength{1pt}");
            real s = 35;
            pair A = dir(s);
            pair C = dir(-s);
            pair B = dir(0);
            pair D = (3/5)*dir(180+s);
            pair O = (0, 0);
            pair H = foot(C, A, D);
            draw(A--O--C--cycle);
            draw(O--D--C);
            draw(circle(O, 1));
            draw(anglemark(C, O, A, 5));
            draw(C--H, mydash);
            dot("$A$", A, dir(A));
            dot("$C$", C, dir(C));
            dot("$D$", D, dir(210));
            dot("$O$", O, dir(135));
            dot("$H$", H, dir(135));
            label("$B$", B, dir(0));
            label("\contour{white}{$\theta^\circ$}", 0.15*dir(0), dir(0));
        \end{asy}
    \end{center}
    Let $H$ be the foot of perpendicular from $C$ onto line $AD$. Then the area
    $A_2$ of $\triangle DOC$ is
    \[ A_2 = \frac{CH \cdot OD}{2} = \frac{CO \cdot \sin \angle HOC \cdot
    OD}{2} = \frac{CO \cdot \sin \angle AOC \cdot OD}{2} = \frac{5 \cdot \sin
    \theta \cdot 3}{2} = \frac{15 \sin \theta}{2}. \]
    Therefore, the total area of the region is 
    \[ A_1 + A_2 = \frac{5 \pi \theta}{72} + \frac{15 \sin\theta}{2}. \qedhere \]
\end{solution}

\begin{question}
    Prove that
    \begin{enumerate}
        \item $\cos 20^\circ$,
        
        \item $\log 21$, and 
        
        \item $\sqrt{3}$ 
    \end{enumerate}
    are irrational.
\end{question}
\begin{solution}
    We will use proof by contradiction in all the following subproblems.
    \begin{enumerate}
        \item Suppose in contrary that $\cos 20^\circ$ is rational. Then $\cos
            20^\circ = \frac{a}{b}$ for some relatively prime integers $a$ and
            $b$. Then since $\cos(3x) = 4\cos^3 x - 3\cos x$, we see that
        \[ \frac{1}{2} = \cos 60^\circ = 4 \cos^3 20^\circ - 3\cos 20^\circ =
        \frac{4a^3}{b^3} - \frac{3a}{b}. \]
        Multiplying both sides of the equation by $2b^3$ gives $b^3 = 8a^3 -
        6ab^2$, and since $a$ divides the right hand side it follows that $a
        \mid b^3$. However, since we assumed that $a$ and $b$ are relatively
        prime, this is a contradiction. Therefore, $\cos 20^\circ$ must be
        irrational.

        \item Similarly, assume that $\log 21$ is rational. Then $\log 21 =
            \frac{a}{b}$ for some relatively prime integers $a$ and $b$. Hence
            \[ \log 21 = \frac{a}{b} \Longrightarrow 21 = 10^\frac{a}{b}
            \Longrightarrow 21^b = 10^a. \]
            However, the left hand side is even and the right hand side is odd
            which is clearly not possible unless $a = b = 0$. Therefore, $\log
            21$ must be irrational. 

        \item Finally, suppose that $\sqrt{3}$ is irrational. Then $\sqrt{3} =
            \frac{a}{b}$ for some relatively prime integers $a$ and $b$. Then 
            \[ \sqrt{3} = \frac{a}{b} \Longrightarrow 3b^2 = a^2. \]
            Since 3 divides the left hand side, $3 \mid a^2 \Rightarrow 3 \mid
            a$. Therefore, $a = 3k$ for some $k$. Substituting gives
            \[ 3b^2 = 9k^2 \Longrightarrow 3k^2 = b^2. \]
            Similarly to above, this means that 3 also divides $b$. However,
            since we assumed that $a$ and $b$ are relatively prime, this is not
            possible. Hence $\sqrt{3}$ is irrational. \qedhere
    \end{enumerate}
\end{solution}

\begin{question}
    Triangle $ABC$ with $AB = c$, $BC = a$ and $CA = b$ is inscribed in a
    circle. Find the radius of the circle in terms of $a$, $b$ and $c$.
\end{question}
\begin{solution}
    Let $H$ be the foot of perpendicular from $A$ onto side $BC$, and let $A'$
    be the diametrically opposide point of $A$ on circle $(ABC)$. Then $\angle
    AHB = \angle ACA' = 90^\circ$. We also have $\angle ABH = \angle ABC =
    \angle AA'C$ so $\triangle ABH \sim \triangle AA'C$. Let $S$ be the area of
    $\triangle ABC$ and $R$ be the radius of $(ABC)$. Then
    \[ 2S = AH \cdot BC = \frac{AB \cdot AC \cdot BC}{AA'} = \frac{abc}{2R}. \]
    \begin{center}
        \begin{asy}
            import olympiad;
            size(6cm);
            defaultpen(fontsize(11pt));
            pen mydash = linetype(new real[] {5,5});
            pair A = dir(120);
            pair B = dir(210);
            pair C = dir(330);
            pair O = (0, 0);
            pair A1 = 2*O-A;
            pair H = foot(A, B, C);
            draw(A--B--C--cycle);
            draw(A--H);
            draw(A--A1);
            draw(A1--C);
            draw(circle(O, 1));
            draw(rightanglemark(A, H, B, 3));
            draw(rightanglemark(A, C, A1, 3));
            dot("$A$", A, dir(A));
            dot("$B$", B, dir(B));
            dot("$C$", C, dir(C));
            dot("$H$", H, dir(270));
            dot("$A'$", A1, dir(A1));
            dot("$O$", O, dir(180));
        \end{asy}
    \end{center}
    Meanwhile, by Heron's formula, 
    \[ S = \sqrt{s(s - a)(s - b)(s - c)}, \]
    where $s = \frac{a + b + c}{2}$ is the semiperimeter of $\triangle ABC$.
    Therefore,
    \[ R = \frac{abc}{4S} = \frac{abc}{4\sqrt{s(s - a)(s - b)(s - c)}}. \qedhere \]
\end{solution}

\begin{question}
    Prove the following Ptolemy's theorem: 
    \par In a cyclic quadrilateral $PQRS$, 
    $$PQ \cdot SR + PS \cdot QR = PR \cdot SQ,$$ 
    That is, the sum of the products of the opposite sides is equal to the
    product of the diagonals.
\end{question}
\begin{solution}
    This is \hyperref[thm: ptolemy]{Ptolemy's theorem} which has appeared before as \hyperref[sol: 2015 National Round P18]{2015 National Round P18}. 
\end{solution}

\begin{question}
    $f(x)$ is a real-valued function defined on $a < x < b$ such that 
    \[f\left( \frac{x_{1} + x_{2}}{2} \right) \leq \frac{1}{2}(f(x_{1}) + f(x_{2})),\] 
    for all $a < x_{1}, x_{2} < b$. Prove that 
    \[f\left(\frac{x_{1} + x_{2} + \cdots + x_{n}}{n}\right)\leq \frac{1}{n}(f(x_{1}) + f(x_{2}) + \cdots + f(x_{n}))\] 
    for all $a < x_{1}, x_{2}, \ldots, x_{n} < b$.
\end{question}
\begin{solution}
    This is one of the rare problems which need \hyperref[teq:
    f.b.induction]{forward-backward induction} to be solved. The base case $n
    = 2$ is true. Now suppose that the inequality is true for $n = k$. We must
    first show that it is true for $n = 2k$:
    \begin{align*}
        \MoveEqLeft
    f\left( \frac{x_1 + x_2 + \cdots + x_k + x_{k + 1} + \cdots + x_{2k}}{2k} \right)\\
        &= f\left( \frac{1}{2} \left( \dfrac{x_1 + x_2 + \cdots + x_k}{k} + \dfrac{x_{k + 1} + x_{k + 2} + \cdots + x_{2k}}{k} \right) \right)\\
        &\leq \frac{1}{2}\left( f\left( \frac{x_1 + x_2 + \cdots + x_k}{k} \right) + f\left( \frac{x_{k + 1} + x_{k + 2} + \cdots + x_{2k}}{k} \right) \right) \\
        &\leq \frac{1}{2} \left( \frac{1}{k} \left( f(x_1) + f(x_2) + \cdots + f(x_k) \right) + \frac{1}{k} \left( f(x_{k + 1}) + f(x_{k + 2}) + \cdots + f(x_{2k}) \right) \right) \\
        &= \frac{f(x_1) + f(x_2) + \cdots + f(x_{2k})}{2k}
    \end{align*}
    where the third inequality is because of the base case and the fourth is
    because of the induction hypothesis. Hence the inequality is true for $n =
    2k$. Now observe that
    \[
        f \left( \frac{1}{k} \left( x_1 + x_2 + \cdots + x_{k - 1} + \frac{x_1 + x_2 + \cdots + x_{k - 1}}{k - 1} \right) \right) = f \left( \frac{x_1 + x_2 + \cdots + x_{k - 1}}{k - 1} \right)
    \]
    Since the inequality is true for any $k$ real numbers between $a$ and $b$,
    it is also true when $x_k = \frac{x_1 + x_2 + \cdots + x_{k - 1}}{k - 1}$.
    As $a < x_1, x_2, \ldots, x_{k - 1} < b$, it is guaranteed that $x_k$ lies
    between $a$ and $b$ as well.
    Therefore, we have
    \begin{align*}
        f \left( \frac{x_1 + x_2 + \cdots + x_{k - 1}}{k - 1} \right) &\leq \frac{1}{k} \left( f(x_1) + f(x_2) + \cdots + f\left( \frac{x_1 + x_2 + \cdots + x_{k - 1}}{k - 1} \right) \right)\\
        (k - 1) f \left( \frac{x_1 + x_2 + \cdots + x_{k - 1}}{k - 1} \right) &\leq f(x_1) + f(x_2) + \cdots + f(x_{k - 1})\\
        f \left( \frac{x_1 + x_2 + \cdots + x_{k - 1}}{k - 1} \right) &\leq \frac{f(x_1) + f(x_2) + \cdots + f(x_{k - 1})}{k - 1}.
    \end{align*}
    Therefore, we have proved that this inequality is also true for $n = k -
    1$, and we are done by forward-backward induction.
\end{solution}

\begin{question}
    Point $A$ has position vector $\overrightarrow{OA}$ and point $B$ has
    position vector $\overrightarrow{OB}$. Prove that for $0\leq \lambda \leq
    1$, the vector $\lambda \overrightarrow{OA} + (1 - \lambda)
    \overrightarrow{OB}$ is a position vector of a point on the line segment
    $AB$. Prove also that any point on the line segment $AB$ has position
    vector $\lambda \overrightarrow{OA} + (1 - \lambda)\overrightarrow{OB}$ for
    some $0\leq \lambda \leq 1$.
\end{question}
\begin{solution}
    Let $P$ be a point such that $P$ has position vector $\lambda
    \overrightarrow{OA} + (1 - \lambda) \overrightarrow{OB}$. This means that
    \[ \overrightarrow{AP} = \overrightarrow{OP} - \overrightarrow{OA} =
    \lambda \overrightarrow{OA} + (1 - \lambda) \overrightarrow{OB} -
    \overrightarrow{OA} = (1 - \lambda)(\overrightarrow{OB} - \overrightarrow{OA})
    = (1 - \lambda)\overrightarrow{AB}, \]
    which shows that $A$, $P$ and $B$ are collinear. Also, as $AP = (1 -
    \lambda)AB$, it follows that $AP \leq AB$ and hence $P$ lies on segment
    $AB$.
    
    Now suppose that $P$ is a point on segment $AB$. We will show that this
    point $P$ has the position vector $\lambda \overrightarrow{OA} + (1 -
    \lambda) \overrightarrow{OB}$ for some $0\leq \lambda \leq 1$. Since
    $\overrightarrow{BP}$ and $\overrightarrow{BA}$ have the same direction, we
    must have $\overrightarrow{BP} = \lambda\overrightarrow{BA}$ for some
    positive constant $\lambda$. As $P$ lies on segment $AB$, we also have $0
    \leq \lambda \leq 1$. Therefore, 
    \[ \overrightarrow{OP} = \overrightarrow{OB} + \overrightarrow{BP} =
        \overrightarrow{OB} + \lambda \overrightarrow{BA} = \overrightarrow{OB}
        + \lambda \overrightarrow{OA} - \lambda \overrightarrow{OB} = \lambda
        \overrightarrow{OA} + (1 - \lambda)\overrightarrow{OB}. \qedhere \]
\end{solution}

\begin{question}
    $f(x)$ is a real-valued function defined on the interval $a < x < b$ such
    that 
    \[f(tx_{1} + (1 - t)x_{2}) \leq tf(x_{1}) + (1 - t)f(x_{2})\] 
    for all $a < x_{1}, x_{2} < b$ and for all $0 \leq t \leq 1$. Prove that
    for each triple $x_1$, $x_2$, $x_3$ of distinct numbers in the interval, 
    \[\frac{(x_{3} - x_{2})f(x_{1}) + (x_{2} - x_{1})f(x_{3}) + (x_{1} -
    x_{3})f(x_{2})}{(x_{1} - x_{2})(x_{2} - x_{3})(x_{3} - x_{1})}\geq 0.\]
\end{question}
\begin{solution}
    WLOG, we can assume that $x_1 < x_2 < x_3$. Then $(x_1 - x_2)$ and $(x_2 -
    x_3)$ are negative while $(x_3 - x_1)$ is positive, so $(x_1 - x_2)(x_2 -
    x_3)(x_3 - x_1)$ must be positive. Therefore, we just need to show the
    numerator is also positive, i.e.,
    \[ (x_3 - x_2)f(x_1) + (x_2 - x_1)f(x_3) + (x_1 - x_3)f(x_2) \geq 0. \]
    Let $P_1 = (x_1, f(x_1))$ and $P_3 = (x_3, f(x_3))$ be points in the plane.
    Since $x_2$ lies between $x_1$ and $x_3$, by the above problem we must have 
    \[ x_2 = tx_1 + (1 - t)x_3 \]
    for some real number $0 < t < 1$. Now define $P_2$ to be the point which
    has position vector 
    \[ \overrightarrow{P_2} = t\overrightarrow{P_1} + (1 - t)\overrightarrow{P_3}. \]
    From the above problem, $P_2$ lies on segment $P_1P_2$. The coordinates of
    this point must be $(x_2, y_2) = (tx_1 + (1 - t)x_3, tf(x_1) + (1 -
    t)f(x_3))$.
    From the given condition, we have
    \[ f(x_2) = f(tx_1 + (1 - t)x_3) \leq tf(x_1) + (1 - t)f(x_3) = y_2. \]
    Therefore,
    \begin{align*}
        \MoveEqLeft
    (x_3 - x_2)f(x_1) + (x_2 - x_1)f(x_3) + (x_1 - x_3)f(x_2)\\
        &\geq (x_3 - x_2)f(x_1) + (x_2 - x_1)f(x_3) + (x_1 - x_3)y_2\\
        &= (x_3 - x_2)f(x_1) + (x_2 - x_1)f(x_3) - ((x_3 - x_2) + (x_2 - x_1))y_2\\
        &= (x_3 - x_2)(f(x_1) - y_2) + (x_2 - x_1)(f(x_3) - y_2). 
    \end{align*}
    \begin{center}
        \begin{asy}
            size(6cm);
            import graph;
            real f(real x) {
                return 0.2 * (x - 1) * (x - 1) + 2;
            }
            pen mydash = linetype(new real[] {5,5});
            real xmin = -0.1;
            real xmax = 5;
            real ymin = -0.1;
            real ymax = 5;
            draw((xmin, 0)--(xmax, 0), arrow=Arrow);
            draw((0, ymin)--(0, ymax), arrow=Arrow);
            draw(graph(f, xmin, 4.5, operator..));
            path tick = (0, 0)--(0, -0.2cm);
            real x1 = 1;
            real x2 = 2;
            real x3 = 3.5;
            draw((x1, 0), tick, L=Label("$x_1$", position=EndPoint));
            draw((x2, 0), tick, L=Label("$x_2$", position=EndPoint));
            draw((x3, 0), tick, L=Label("$x_3$", position=EndPoint));
            pair P1 = (x1, f(x1));
            pair P = (x2, f(x2));
            pair P3 = (x3, f(x3));
            dot("$P_1$", P1, dir(225));
            dot("$P_3$", P3, dir(315));
            pair P2 = (x2 - x1)/(x3 - x1) * P3 + (x3 - x2)/(x3 - x1) * P1;
            dot("$P_2$", P2, dir(90));
            draw(P1--P3);
            draw((x1, 0)--P1, mydash);
            draw((x2, 0)--P2, mydash);
            draw((x3, 0)--P3, mydash);
        \end{asy}
    \end{center}

    However, since $P_1$, $P_2$ and $P_3$ are collinear, lines $P_1P_2$ and
    $P_2P_3$ have the same slope. Therefore,
    \[ \frac{y_2 - f(x_1)}{x_2 - x_1} = \frac{f(x_3) - y_2}{x_3 - x_2}
    \Longrightarrow (x_3 - x_2)(f(x_1) - y_2) + (x_2 - x_1)(f(x_3) - y_2) = 0.
    \]
    and we are done. 
\end{solution}
\begin{remark}
    The reason for constructing the point $P_2$ is due to the following
    observation. Let $A_1$ be the area of the rectangle with base length $x_3 - x_2$
    and height $f(x_1)$. Similarly define $A_2$ and $A_3$ with base lengths $x_3
    - x_1$ and $x_2 - x_1$. Then the complicated inequality is equivalent to
    much simpler $A_1 + A_3 \geq A_2$. However, it can be shown that the sum on
    the left is the same as the area of a rectangle with length $x_3 - x_1$,
    and height equal to that of $P_2$. Therefore, the inequality amounts to
    showing that the height of $P_2$ is greater than or equal to $f(x_2)$ which
    follows readily from the given convexity condition. The above proof is just
    an algebraic translation of this geometric argument.
\end{remark}
% Add in area motivation? 
\begin{question}
    Equilateral triangle $ABC$ is inscribed in a circle. $P$ is a point on arc
    $AB$. Prove that $PA + PB = PC$.
\end{question}
\begin{solution}[1]
    Let $p$ be the side length of the equilateral triangle $ABC$. Then by
    \hyperref[thm: ptolemy]{Ptolemy's theorem} on quadrilateral $ACBP$,
    \[ PA \cdot p + PB \cdot p = PC \cdot p \Longrightarrow PA + PB = PC. \qedhere \]
\end{solution}
\begin{center}
    \begin{asy}
        import olympiad;
        size(6cm);
        defaultpen(fontsize(11pt));
        pen mydash = linetype(new real[] {5,5});
        pair A = dir(90);
        pair B = dir(210);
        pair C = dir(330);
        pair P = dir(295);
        draw(A--B--C--cycle);
        draw(circle((0, 0), 1));
        draw(P--A);
        draw(P--B);
        draw(P--C);
        add(pathticks(A--B, s=3));
        add(pathticks(B--C, s=3)); 
        add(pathticks(C--A, s=3));
        dot("$C$", A, dir(A));
        dot("$A$", B, dir(B));
        dot("$B$", C, dir(C));
        dot("$P$", P, dir(P));
    \end{asy}
\end{center}
\begin{solution}[2]
    Let segment $PC$ and segment $AB$ intersect at $E$. Then $\angle PCA =
    \angle PBA = \angle PBE$, and $\angle CPA = \angle CBA = \angle CAB = \angle
    CPB = \angle BPE$. Therefore, $\triangle PCA \sim \triangle PBE$. Similarly,
    $\triangle PCB \sim \triangle PAE$. Consequently,
    \[ \frac{PA}{PC} + \frac{PB}{PC} = \frac{AE}{BC} + \frac{EB}{AC}
    = \frac{AE + EB}{AB} = \frac{AB}{AB} = 1. \]
    Rearranging gives our desired identity.
\end{solution}

\begin{question}
    $\triangle ABC$ is inscribed in a circle. The tangent at $C$ and the line
    through $B$ parallel to $AC$ meet at $D$. The tangent at $B$ and the line
    through $C$ parallel to $AB$ meet at $E$. Prove that $BC^2 = BE \cdot CD$. 
\end{question}
\begin{solution}
    Since $AB \parallel CE$ and $BE$ is tangent to $(ABC)$, we have $\angle ABC
    = \angle BCE$ and $\angle CAB = \angle EBC$, so $\triangle ABC \sim
    \triangle BCE$. Analogously, as $AC \parallel BD$ and $CD$ is tangent to
    $(ABC)$, we have $\angle BCA = \angle DBC$ and $\angle CAB = \angle BCD$.
    Hence $\triangle BCE \sim \triangle ABC \sim \triangle CDB$.
    \begin{center}
        \begin{asy}
            import olympiad;
            size(7cm);
            defaultpen(fontsize(11pt));
            pen mydash = linetype(new real[] {5,5});
            pair A = dir(120);
            pair B = dir(210);
            pair C = dir(330);
            pair O = circumcenter(A, B, C);
            pair A1 = B + C - A;
            pair A2 = extension(B, rotate(90, B)*O, C, rotate(90, C)*O);
            pair D = extension(C, A2, B, A1);
            pair E = extension(C, A1, B, A2);
            draw(A--B--C--cycle, black+1);
            draw(B--D--C);
            draw(B--E--C);
            draw(circle(O, 1));
            dot("$A$", A, dir(A));
            dot("$B$", B, dir(B));
            dot("$C$", C, dir(C));
            dot("$D$", D, dir(270));
            dot("$E$", E, dir(270));
        \end{asy}
    \end{center}
    This shows that
    \[ \frac{BC}{CD} = \frac{EB}{BC} \Longrightarrow BC^2 = BE \cdot CD. \qedhere \]
\end{solution}

\begin{question}
    $\triangle ABC$ is isosceles with $AB = AC$. $D$ is the midpoint of $BC$.
    $AB$ and $AC$ are produced to $X$ and $Y$ respectively such that $\angle
    XDY = \angle DCY$. Prove that $\triangle XBD$, $\triangle XDY$ and
    $\triangle DCY$ are similar triangles. 
\end{question}
\begin{solution}
    Since $AB = AC$, we must have $\angle XBD = \angle DCY = \angle XDY$. Also,
    \[ \angle BDX = 180^\circ - \angle XDY - \angle YDC = 180^\circ - \angle
    DCY - \angle YDC = \angle CYD, \]
    so it follows that $\triangle XBD \sim \triangle DCY$. This means that
    $\frac{XD}{DY} = \frac{XB}{DC} = \frac{XB}{BD}$. As $\angle XDY = \angle
    XBD$, we also see that $\triangle XBD \sim \triangle XDY$ as desired.
\end{solution}
\begin{center}
    \begin{asy}
        import olympiad;
        size(7cm);
        defaultpen(fontsize(11pt));
        pen mydash = linetype(new real[] {5,5});
        real s = 110;
        real k = 40;
        pair A = dir(s);
        pair X = dir(-s-k);
        pair Y = dir(-s+k);
        pair D = incenter(A, X, Y);
        pair B = extension(A, X, D, rotate(90, D)*A);
        pair C = extension(A, Y, D, rotate(90, D)*A);
        pair O = (0, 0);
        draw(A--B--C--cycle);
        draw(X--D--Y--cycle, black+1);
        draw(X--B--D, black+1);
        draw(Y--C--D, black+1);
        add(pathticks(A--B, s=3));
        add(pathticks(A--C, s=3));
        add(pathticks(B--D, n=2, spacing=1, s=3));
        add(pathticks(C--D, n=2, spacing=1, s=3));
        draw(anglemark(X, B, D, 3));
        draw(anglemark(X, D, Y, 3));
        draw(anglemark(D, C, Y, 3));
        dot("$A$", A, dir(90));
        dot("$B$", B, dir(180));
        dot("$C$", C, dir(0));
        dot("$X$", X, dir(225));
        dot("$Y$", Y, dir(315));
        dot("$D$", D, dir(90));
    \end{asy}
\end{center}
\begin{question}
    $a_{1}$, $b_{1}$, $c_{1}$, $a_{2}$, $b_{2}$, $c_{2}$ are positive real
    numbers with $a_{1}c_{1} - b_{1}^2 \geq 0$ and $a_{2}c_{2} - b_{2}^2 \geq
    0$. Show that $(a_{1} + a_{2})(c_{1} + c_{2}) - (b_{1} + b_{2})^2 \geq 0$.
\end{question}
\begin{solution}
    From the given conditions we have $a_1 c_1 \geq b_1^2$ and $a_2 c_2 \geq
    b_2^2$. By \hyperref[thm: cs]{Cauchy-Schwarz} inequality,
    \[ (a_1 + a_2)(c_1 + c_2) = ((\sqrt{a_1})^2 +
    (\sqrt{a_2})^2)((\sqrt{c_1})^2 + (\sqrt{c_2})^2) \geq (\sqrt{a_1c_1} +
    \sqrt{a_2c_2})^2 = (b_1 + b_2)^2. \qedhere \]
\end{solution}

\begin{question}
    How many permutations of the word `TRIANGLE' have none of the vowels together?
\end{question}
\begin{solution}
    First notice that there are 5 consonants and 3 vowels. Consider the
    following template; blue boxes are for consonants and red boxes are for
    vowels. 
    \begin{center}
        \begin{asy}
            size(10cm);
            pair S = (2, 0);
            real t = 1;
            for (int i = 0; i < 11; i = i + 2){
                draw(shift(i*S)*polygon(4), red+t);
            }
            for (int i = 1; i < 10; i = i + 2){
                draw(shift(i*S)*polygon(4), Cyan+t);
            }
        \end{asy}
    \end{center}
    Since no two vowels can be together, this means that there must be at most
    1 vowel in each red box. There are 6 red boxes, so the number of ways to
    arrange 3 vowels in 6 red boxes is $6 \times 5 \times 4 = 120$. Now the
    number of ways to arrange the 5 consonants in 5 blue boxes is $5! = 120$.
    Therefore, the total number of words is $120^2 = 14400$.
\end{solution}

\begin{question}
    From the group of $2n + 1$ people, how many ways to choose a group of $n$
    people or less?
\end{question}
\begin{solution}[1]
    Label the people from 1 to $2n + 1$, and let $S$ be the set $\{ 1, 2,
    \ldots, 2n + 1 \}$. Let $A$ be the set of subsets of $S$ which have $n$
    elements or less, and let $B$ be the set of subsets of $S$ which has $n +
    1$ or more elements. It is easy to see that for each element $X$ in $A$, we
    can pair it up with $S \setminus X$, which belongs in $B$. This means that
    $A$ and $B$ have the same number of elements. Meanwhile, the number of
    subsets of a set of $2n + 1$ elements is $2^{2n + 1}$, so 
    \[ |A| + |B| = 2^{2n + 1} \Longrightarrow |A| = 2^{2n} = 4^n. \]
    One thing to be careful is that we also included the empty set in $A$.
    Subtracting it from the total, we see that the number of ways to choose a
    group of $n$ people or less from a group of $2n + 1$ people is $4^n - 1$.
\end{solution}
\begin{solution}[2]
    The number of ways to choose a group of $i$ people from a group of $2n + 1$
    people is $\binom{2n + 1}{i}$. Therefore, the quantity we seek is
    \[N = \binom{2n + 1}{1} + \binom{2n + 1}{2} + \cdots + \binom{2n + 1}{n} \]
    where $N$ is the total number of ways. Now by the binomial theorem, we have
    \[ 2^{2n + 1} = (1 + 1)^{2n + 1} = \binom{2n + 1}{0} + \binom{2n + 1}{1} +
    \cdots + \binom{2n + 1}{2n + 1}. \]
    However, remember that $\binom{2n + 1}{i} = \binom{2n + 1}{2n + 1 - i}$ for
    $0 \leq i \leq 2n + 1$. Therefore, the quantity on the right hand side can
    be rewritten as
    \begin{align*}
        \MoveEqLeft
    \binom{2n + 1}{0} + \binom{2n + 1}{1} + \cdots + \binom{2n + 1}{n} + \binom{2n + 1}{n + 1} + \cdots + \binom{2n + 1}{2n + 1}\\ 
        &= 2\left( \binom{2n + 1}{0} + \binom{2n + 1}{1} + \cdots + \binom{2n + 1}{n} \right)\\
        &= 2\left( N + \binom{2n + 1}{0} \right)\\
        &= 2N + 2.
    \end{align*}
    Therefore, we finally have
    \[ N = \frac{2^{2n + 1} - 2}{2} = 4^n - 1. \qedhere \]
\end{solution}

\begin{question}
    Show that the product of any positive integer and its $k - 1$ successors is
    divisible by $k!$. 
\end{question}
\begin{solution}
    Let the $k$ consecutive integers be $n + 1, n + 2, \ldots, n + k$. Then 
    \[ N = \frac{(n + 1)(n + 2) \cdots (n + k)}{k!} = \frac{n!(n + 1)(n + 2)
    \cdots (n + k)}{n!k!} = \frac{(n + k)!}{n!(n + k - n)!} = \binom{n + k}{n}.
    \]
    On the other hand, the right hand side is the number of ways to choose $n$
    objects from $n + k$ objects, so this is clearly an integer. Hence $N$ is
    an integer and we are done.
\end{solution}

\begin{question}
    How many integers between 1 and 1,000,000 have the sum of digits equal to 10?
\end{question}
\begin{solution}
    Note that this number is just the number of $6$-tuples $(x_1, x_2, \ldots,
    x_6)$ such that $x_1 + x_2 + \cdots + x_6 = 10$, with the restriction that
    $0 \leq x_i \leq 9$ for all $1 \leq i \leq 6$ since they are supposed to
    represent digits. If we remove this restriction, then by \hyperref[teq:
    starsandbars]{stars and bars}, the number of such tuples is $\binom{15}{5} =
    72072$. Since such a tuple already satisfies $0 \leq x_i \leq 10$ for all $1
    \leq i \leq 6$, the only tuples that violate the restriction are $(10, 0, 0,
    0, 0, 0)$ and its variants. There are 6 of them, so the total number of
    integers between 1 and 1,000,000 with digit sum equal to 10 is $72072 - 6 =
    72066$.
\end{solution}

\begin{question}
    Each side and diagonal of a regular hexagon is coloured either red or blue.
    Show that there is a triangle with all three sides of the same colour. 
\end{question}
\begin{solution}
    Take any vertex $v$ of the hexagon. Since there are 5 edges connected to
    $v$, by the \hyperref[thm: pigeonhole]{pigeonhole principle}, there are at
    least three edges of the same colour. WLOG, suppose that these three edges
    are blue, and suppose that they connect $v$ to three other vertices $x$,
    $y$ and $z$. Then if one of the edges $(x, y)$, $(y, z)$ or $(z, x)$ are
    blue, we will have a completely blue triangle, so suppose that none of them
    are blue. This means that those three edges form a completely red triangle,
    and hence we have proved that there is always a triangle with all three
    sides of the same colour.
\end{solution}

\begin{question}
    Find the value of $2018^2 - 2017^2 + 2016^2 - 2015^2 + \cdots + 2^2 - 1$.
\end{question}
\begin{solution} 
    Differences of squares!
    \begin{align*}
        \MoveEqLeft
    2018^2 - 2017^2 + 2016^2 - 2015^2 + \cdots + 2^2 - 1\\
        &= (2018 + 2017)(2018 - 2017) + (2016 + 2015)(2016 - 2015) + \cdots + (2 + 1)(2 - 1)\\
        &= 2018 + 2017 + 2016 + 2015 + \cdots + 2 + 1\\
        &= \frac{2018 \cdot 2019}{2}\\
        &= 2037171. \qedhere
    \end{align*}
\end{solution}
